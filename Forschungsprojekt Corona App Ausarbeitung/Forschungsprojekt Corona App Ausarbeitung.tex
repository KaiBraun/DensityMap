\documentclass[conference]{IEEEtran}
\IEEEoverridecommandlockouts
% The preceding line is only needed to identify funding in the first footnote. If that is unneeded, please comment it out.
\usepackage{cite}
\usepackage{amsmath,amssymb,amsfonts}
\usepackage{algorithmic}
\usepackage{graphicx}
\usepackage{textcomp}
\usepackage{xcolor}
\usepackage[ngerman]{babel}
\def\BibTeX{{\rm B\kern-.05em{\sc i\kern-.025em b}\kern-.08em
    T\kern-.1667em\lower.7ex\hbox{E}\kern-.125emX}}
\begin{document}

\title{Entwurf und Analyse von Crowd-Sensing Mechanismen für mobile Corona-Warn-Applikationen\\
{\footnotesize \textsuperscript{*}Note: Sub-titles are not captured in Xplore and
should not be used}
}

\author{\IEEEauthorblockN{Gabriel Bonnet}
\IEEEauthorblockA{\textit{Bsc. Science Softwaretechnik} \\
\textit{Universität Stuttgart}\\
Stuttgart, Germany}
\and
\IEEEauthorblockN{Kai Braun} 
\IEEEauthorblockA{\textit{Bsc. Science Softwaretechnik} \\
\textit{Universität Stuttgart}\\
Stuttgart, Germany}
\and
\IEEEauthorblockN{Hannes Deichmann}
\IEEEauthorblockA{\textit{Bsc. Science Softwaretechnik} \\
\textit{Universität Stuttgart}\\
Stuttgart, Germany}
\and
\IEEEauthorblockN{Timm Marquardt}
\IEEEauthorblockA{\textit{Bsc. Science Informatik} \\
\textit{Universität Stuttgart}\\
Stuttgart, Germany}
\and
\IEEEauthorblockN{\textbf{Betreuer:} Dr. rer. Nat. Frank Dürr}
\IEEEauthorblockA{\textit{Institut für Parallele und Verteilte Systeme} \\
\textit{Universität Stuttgart}\\
Stuttgart, Germany \\
frank.duerr@ipvs.uni-stuttgart.de}
\and
\IEEEauthorblockN{\textbf{Prüfer: } Prof. Dr. rer. Nat. Kurt Rothermel}
\IEEEauthorblockA{\textit{Institut für Parallele und Verteilte Systeme} \\
\textit{Universität Stuttgart}\\
Stuttgart, Germany \\
kurt.rothermel@ipvs.uni-stuttgart.de}
}

\maketitle

\begin{abstract}
TODO
\end{abstract}

\begin{IEEEkeywords}
Corona, Corona-Warn-App, Density Map, Risikoberechnung
\end{IEEEkeywords}

\section{Einleitung}


\section{Bestehende Arbeiten}

\subsection{Corona Lage}
\subsection{Corona Warn App}
Die deutsche "Corona Warn App" (CWA) wurde im Auftrag des Robert Koch Institutes von SAP, der deutschen Telekom und weiteren Unternehmen entwickelt. 
Ihr Hauptziel ist die automatisierte und anonyme Kontaktverfolgung mittels Smartphones, um möglichst viele Infektionsketten zu unterbrechen und die Ausbreitung des Coronavirus weitmöglichst zu verlangsamen. 
Zum aktuellen Zeitpunkt ist die App seit etwa einem dreiviertel Jahr aktiv und  wurde knapp 30millionen Mal heruntergeladen (Stand Mai 2021).

Die Funktion der Corona Warn App ist relativ simpel. Ein Nutzer lädt sich die App auf sein Smartphone und aktiviert seine Bluetooth Funktion. 
Falls er nun mit einer weiteren Person, die ebenfalls die CWA installiert hat in Kontakt kommt, werden anonym Daten ausgetauscht und die Begegnung auf den jeweiligen Endgeräten gespeichert. 
Sofern man sich nun mit dem Coronavirus Infiziert hat, erhält man von einer Teststelle zusammen mit dem positiven Testergebnis einen Code, welchem man in seine App eintragen kann. 
Alle Kontaktpersonen werden dementsprechend informiert, ohne zu wissen von wem diese Mitteilung letztendlich stammt. 
Die Corona Warn App unterscheidet prinzipiell zwischen zwei Risikostufen, die mithilfe eines grünen bzw. roten Bildschirms auf der Startseite der App angezeigt werden. 
Ein grüner Bildschirm bedeutet hierbei keine/kaum Risikobegegnungen und damit ein vermindertes Risiko, während ein roter Bildschirm ein erhöhtes Risiko anzeigt.

Der Corona Warn App zugrunde liegendes Feature ist das "Exposure Notification Framework", welches in Kooperation von Google und Apple für Corona Apps entwickelt wurde und das Bluetooth Low Energy (BLE) Feature nutzt, um die Begegnungsdaten der Smartphones anonym untereinander auszutauschen. \\
\\
\textit{Exposure Notification Framework:}\\
Jeden tag wird vom Framework ein sogenannter Temporary Exposure Key (TEK) generiert. Dieser hält 24 Stunden und ist ein zufällig ausgewählter Schlüssel. Aus diesem TEK werden Rolling Proximity Identifiers (RPI) abgeleitet. 
Diese RPI's haben eine Lebensdauer von 10-15 Minuten, bevor sie gelöscht und ein neuer RPI aus dem TEK erstellt wird. Wichtig ist hierbei, dass sich aus den RPI's nicht der TEK ableiten lässt. 
Die RPI's werden nun unter den Smartphones im Falle einer Begegnung mithilfe von BLE ausgetauscht. Neben den RPI werden zudem verschlüsselt Informationen über die Begegnungsdauer, sowie dem ungefähren Abstand ausgetauscht.
Falls sich nun eine Person mit dem Coronavirus infiziert, werden dessen TEK (nun Diagnosis Keys genannt) der letzten 14 Tage auf einen zentralen Server hochgeladen. 
Jedes Smartphone hat Zugriff auf diesen zentralen Server und kann die Diagnosis Keys aller Infizierten Personen  herunterladen. 
Mithilfe der Diagnosis Keys können dann die gespeicherten RPI's aus den Begegnungen abgeglichen und im Falle einer Übereinstimmung die Begegnungsdauer und der Abstand entschlüsselt werden. 
Aus diesen Informationen bestimmt die Corona Warn App das Infektionsrisiko der Person. 

\subsection{Bluetooth Low Energy und Density Maps}

\section{Lösungsvorschläge}


\subsection{Übersicht aller Erweiterungen}

\textit{Impfpass:} - In der aktuellen Situation ist es wichtig, nachweisen zu können, dass man geimpft wurde. Dies könnte man entweder dadurch regeln, einen analogen Impfnachweis immer bei sich zu tragen, oder durch einen digiatalen Impfpass in der Corona-Warn-App.\\
Größter Vorteil eines in der App integrierten Nachweises wäre die Digitalisierung des Verwaltungswesens und dadurch eine Vereinfachung des Organisationsaufwands.\\ Anstatt mehrere Unterlagen mit sich herumzutragen, hat man diese stattdessen gesammelt in einer App.\\
Digitalisierung ist zudem ein positiv konnotierter Begriff, welcher einen Anreiz für die Bevölkerung schafft, die App zu nutzen. Dies ist immer ein relevanter Nutzen, da es das Ziel ist, die Corona-Warn-App von 100\% der Bevölkerung nutzen zu lassen. \\
Implementiert man zusätzlich dazu eine Art Krankenakte, in der man Merkmale wie Gewicht und Vorerkrankungen einträgt, liese sich dadurch die Impfpriorisierung und die Reihenfolge in der Bevölkerung besser organisieren.\\
Benötigte Daten für einen generellen Impfpass wären:\\
- Name \\
- Krankenkasse\\
- Durchgeführte Impfungen\\
Für die Einschätzung der Impfpriorisierung wäre zusätzlich erforderlich:\\
- Alter\\
- Vorerkrankungen \\
- Gewicht \\
- Beruf/Arbeitsumfeld\\
Bei einem unverbindlichen Impfpass wären diese Angaben nicht wirklich nutzbar außer für den Endnutzer, da man auf die Eigenverantwortung des einzelnen Nutzers angewiesen wäre, die Daten also einfach gefälscht werden können.\\
Stattdessen müsste man, um die Daten als sicher zu garantieren, die Bearbeitung des Impfpass beschränken, sodass nur offizielle Stellen wie Krankenhäuser oder Arztpraxen diesen bearbeiten können.\\
Somit wird Richtigkeit der Daten gewährt, welche dann für das Krankenwesen genutzt werden können.\\ \\

\textit{Gamification:}\\
Eine Möglichkeit, das Problem einer zu niedrigen Nutzung der App zu adressieren, ist Gamification. Dadurch versuchen wir, den Anreiz zu steigern, die App herunterzuladen und auch aktiv zu benutzen.\\ 
Gamification der App kann es in verschiedenen Varianten geben. Zum Beispiel lassen sich simple Spielmechaniken wie persönliche Highscores einbauen, welche man mit anderen Nutzern der App vergleichen kann.  Dafür bieten sich Statistiken an, die sich bereits aus der Nutzung der App erstellen lassen. Dazu zählen die getroffenen Personen an einem Tag, die durchschnittliche Dauer einer Begegnung oder der ermittelte Risikofaktor. Ziel der Highscores wäre es dann, diese Werte so möglich wie gering zu halten oder im Vergleich zu anderen Nutzern bessere Werte zu haben.\\
Vorstellbar sind auch daraus resultierende optische Veränderungen der App durch neue Farben oder ein anderes Design.\\
Dieses Feature richtet sich vermutlich an eine jüngere Zielgruppe, die mit solchen Mechaniken meist mehr vertraut ist als ältere Generationen.\\
Es könnte sogar sein, dass dieses Feature als Verharmlosung der ernsten Thematik interpretiert wird, was in einem gegensätzlichen Effekt zur eigentlichen Intention resultieren würde.\\
Eine Alternative zu Highscores und anderen virtuellen Errungenschaften wäre eine Art Belohnung in der Realität, zum Beispiel durch eine in der App durchgeführte Lotterie.\\
Dies ist vermutlich attraktiver für alle Generationen, jedoch schwieriger umzusetzen, da hier Betrug oder mehrmalige Teilnahme verhindert werden muss. Vorstellbar wäre auch eine Verbindung mit den ermittelten Statistiken, sodass Leute mit guten Werten eine höhere Gewinnchance haben.\\
 
\subsection{Bewertungskriterien}

Um einen möglichst einheitlichen Vergleich der Features zu garantieren, haben wir sechs Bereiche festgelegt, nach denen die Features beurteilt werden.\\
Diese Bereiche sind wie folgt aufgeteilt:\\
\textit{Privatsphäre:} Wie stark greift das Feature in die Privatsphäre ein?\\
\textit{Nutzen:} Welchen effektiven Nutzen bringt das Feature und lohnt sich der Aufwand für den effektiven Nutzen?\\
\textit{Umsetzbarkeit:} Wie realistisch ist es, dieses Feature zu implementieren in Anbetracht von Datenschutz, Programmierung oder anderen Hindernissen?\\
\textit{Akzeptanz:} Da wir eine möglichst hohe Nutzung der App in der Bevölkerung erzielen wollen, ist es wichtig, dass das Feature entweder einen Anreiz zur Nutzung schafft oder zumindest diesen nicht schwächt. Die Frage ist also, ob die neue Funktion Skepsis oder Missbilligung beim Nutzer auslöst.\\
\textit{Sicherheit:} Wie gut lässt sich das implementierte Feature vor Missbrauch und Angriffen schützen?\\
\textit{Daten:} Welche Daten werden mindestens benötigt, dass das Feature funktioniert?\\
\subsection{Bewertung Nebenfeatures}

\section{Hauptfeatures}
\subsection{Erweiterte Risikoberechnung}
\subsubsection{Definitionen}

Im Folgenden werden die Begrifflichkeiten genauer definiert: 
Generell wird angenommen, dass wegen der Inkubationszeit, 
Tests beziehungsweise Begegnungen innerhalb der letzten 14 Tage stattgefunden haben. 
Man unterscheidet zwischen zwei Gruppen: 
CWA-User, die von der CWA somit aufgezeichnet werden und Not-CWA-User, die die CWA nicht nutzen und somit nicht aufgezeichnet werden.
CWA-User werden in drei Untergruppen aufgeteilt: CWA-Positive sind alle Personen, die einen positiven Test in die CWA eingetragen haben, 
CWA-Likely-Positive sind alle Personen, die mindestens eine Begegnung mit CWA-Positive hatten und einen roten Bildschirm auf der CWA angezeigt bekommen und 
CWA-Likely-Negative sind alle Personen, die meist keine Begegnung mit CWA-Positive hatten und somit einen grünen Bildschirm auf der CWA angzeigt bekommen.
Ebenso wird die NOT-CWA-User in drei Untergruppen aufgeteilt: 
Not-CWA-Positive sind alle Personen, die ein positives Test Ergebnis haben, dieses aber nicht in die CWA eingetragen haben (Es ist möglich, dass sie die CWA trotzdem nutzen), 
Hidden-Positive sind alle Personen, die infiziert wurden, jedoch kein positives Testergebnis nachweisen können und Hidden-Negative sind alle Personen, die nicht infiziert sind.

\subsubsection{Familienfeature}

\subsubsection{Feature Potenziell Kranke}

\subsubsection{Feature TRL Update}

\subsubsection{Berechnung}

Die Risikoberechnung der aktuellen Corona Warn App zeigt einem Nutzer zwei Risiken an, das erhöhte Risiko mit einem roten Bildschirm und das geringe Risiko mit einem grünen Bildschirm. 
Um dieses zu berechnen wird das Übertragunspotenzial aus Begegnungsdauer und die Begegnungsnähe mit der CWA-Positive Person mit einbezogen und die Wahrscheinlichkeit, 
dass eine Person infiziert ist, Positiv-Begegnungs-Wahrscheinlichkeit, mit einberechnet. 
Wenn das Übertragunspotenzial sich insgesamt zu einer Begegnungsdauer von 13 Minuten in weniger als 8 Metern Abstand aufsummiert, wird der Bildschirm rot, 
ein erhöhtes Risiko angezeigt und die Person zählt zur CWA-Likely-Positive Gruppe. 
Die Positiv-Begegnungs-Wahrscheinlichkeit ist in der aktuellen CWA immer 1, da es sich in jedem Fall um Begegnungen mit CWA-Positive Personen handelt.
Diese Berechnung hätte einen hohen Nutzen, wenn mehr Personen die App herunterladen würden und mehr Tests zur Verfügung gestellt werden könnten. (Quelle 123)
Je mehr Tests durchgeführt werden und auch in die CWA eingetragen werden, umso genauer wird die momentane Risikoberechnung. 
Aufgrund der aktuellen Test-Möglichkeiten und Downloadzahlen sollte die Risikoberechnung jedoch angepasst werden.

Ein Problem der Risikoberechnung der CWA ist, dass diese zu wenige relevante Faktoren mit einberechnet, denn nur CWA-Positive Personen werden mit einbezogen.
Alle anderen Personengruppen werden ignoriert. Der Risikowert ist somit geringer als er eigentlich sein sollte. Im Folgenden werden weitere Faktoren beschrieben, die die angepasste Risikoberechnung beachtet:
Mit der Dunkelziffer lässt sich die Anzahl an tatsächlich Infizierten abschätzen und sie sorgt dafür, dass CWA-Likely-Negative Personen, unter einbezug der positiven Not-CWA-User, 
zu einem Risiko führen können, welches in der neuen Risikoberechnung nicht ignoriert wird.
Die Downloadzahl wird genutzt, um den Risikowert nach oben zu korrigieren, da die Not-CWA-User nicht als Begegnung aufgezeichnet werden.
Begegnungen mit CWA-Likely-Positive Personen, die eine höhere Wahrscheinlichkeit haben infiziert zu sein, da sie direkten Kontakt zu Infizierten hatten, 
werden in der neuen Berechnung stärker als die CWA-Likely-Negative gewichtet.
Außerdem ist die Risikoberechnungsanzeige nicht transparent genug, zwar ist die Anzahl an Begegnungen mit CWA-Positive Personen sichtbar, 
jedoch sieht man weder den Grad des Übertragungspotenzials, noch wie der Risikowert zustande gekommen ist.

Zudem sollte es eine geringere Abstufung der Risikostufen geben, denn nur grün und rot wecken den Eindruck, dass man bei einem grünen nicht gefährdet wäre. 
Ab einem gewissen Risiko sollte auch ein gelber sowie ein orangener Bildschirm angezeigt werden, damit die Nutzer ihr Verhalten besser regulieren können. 

Dazu wurde eine angepasste Risikoberechnung ausgearbeitet, die Daten des RKIs und aus anderen Features, sowie Daten aus Umfragen, nutzt, 
um die fehlenden Daten mit einzubeziehen oder zu beschaffen. Die neue Risikoberechnung wird transparent am Ende neben einer feineren Granulierung des Risikowerts angezeigt. 
Die neue Risikoberechnung behandelt drei Punkte die zu einem neuen Risikowert zusammenaddiert werden.

\centerline{\text{ }}
\centerline{\text{Neuer Risikowert := }}
\centerline{\text{R\textsubscript{CWA-Positive} + R\textsubscript{CWA-Likely-Positive} + R\textsubscript{CWA-Likely-Negative}}}
\centerline{\text{ }}

Der erste Summand ist R\textsubscript{CWA-Positive} und berechnet das Risiko, indem nur Begegnungen mit CWA-Positive Personen einbezogen werden. 
Dabei wird jede Begegnung gewichtet, je nach dem wie hoch das Übertragungspotenzial der Begegnung ist. 
Der zweite Summand R\textsubscript{CWA-Likely-Positive} schätzt das Risiko von Begegnungen mit Likely-Positive Personen, sprich roten Bildschirmen bzw. 
CWAs mit einem Risikowert von über 13, aber keinem positiven Test Ergebnis, ab.
Außerdem gibt es noch den dritten Summanden R\textsubscript{CWA-Likely-Negative}, der die Begegnungen mit Personen aus der Personengruppe CWA-Likely-Negativ mit einberechnet,
wobei auch die Personengruppe der Not-CWA-User mit betrachtet wird.

Der Neue Risikowert wird anschließend wird in eine Bildschirmfarbe umgewandelt. 
Dabei werden anstatt der ursprünglichen zwei Farben nun vier Farben Rot, Orange, Gelb und Grün verwendet.


Im Folgenden werden die drei Summanden näher beschrieben, wobei die angepasste Grafik Abbildung Risikoberechnung(Unter der Risikoberechnung) dies veranschaulicht.



\paragraph{R\textsubscript{CWA-Positive}}

Das Risiko R\textsubscript{CWA-Positive} entspricht der aktuellen Risikoberechnung der CWA (Stand 26.04.2021). 
In dieses Risiko fließen alle Begegnungen mit der Personengruppe CWA-Positive ein. 
Da CWA-Positive Personen ein positives Testergebnis in der CWA eingetragen haben, ist die Positiv-Begegnungs-Wahrscheinlichkeit hier konstant 1. 
Es wird also nur das Übertragungspotential betrachtet. Dafür wird zunächst zu einer Begegnung die Weighted Exposure Time ausgewertet. 
Die Weighted Exposure Time bestimmt die Gewichtung der Begegnungszeit t anhand der geringsten Dämpfung (attenuation) att  wie folgt: 

\centerline{\text{ }}
\centerline{$\text{WeightedExposureTime} := t \cdot \text{attFactor}$}
\centerline{\text{ }}

Wobei der attFactor durch folgende Intervalle aus der Distanz att bestimmt wird:

\centerline{\text{ }}

\text{$att < 63dB:attFactor=0.8$}

\text{$63dB \le att<73dB:attFactor=1.0$}

\text{$73dB \le att<79dB:attFactor=0.1$}

\text{$79dB \le att:attFactor=0$}
\centerline{\text{ }}

Dabei kann man die Werte 63dB bzw. 73dB, als Signalstärke von einem Handy interpretieren, das 3 Meter bzw 8 Meter weit entfernt ist. 
Dabei schwächen Kleidung, Körper oder andere Gegenstände das Signal ab und sorgen dafür, dass die meisten Begegnungen erst auf kleinerer Distanz aufgezeichnet werden (Quelle cwa-risk-assessment).
Aus der Weighted Exposure Time wird dann die Normalised Exposure Time t­n mithilfe des Transmission Risk Levels TRL ermittelt und des daraus resultierenden Transmission Risk Values TRV. 
Das Transmission Risk Level wird in Abschnitt TRL genauer beschrieben und bezieht die Wahrscheinlichkeit mit ein, dass eine Person die Krankheit auch wirklich überträgt. 
Der TRV ergibt sich aus diesem nach der folgenden Vorschrift:

\centerline{\text{ }}

\text{$TRL=I \rightarrow TRV=0.0$}

\text{$TRL=II\rightarrow TRV=0.0$}

\text{$TRL=III\rightarrow TRV=0.6$}

\text{$TRL=IV\rightarrow TRV=0.8$}

\text{$TRL=V\rightarrow TRV=1.0$}

\text{$TRL=VI\rightarrow TRV=1.2$}

\text{$TRL=VII\rightarrow TRV=1.4$}

\text{$TRL=VIII\rightarrow TRV=1.6$}
\centerline{\text{ }}

Mit dem TRV lässt sich dann t\textsubscript{n} bestimmen:

\centerline{\text{ }}
\centerline{$t_n := \text{Weighted Exposure Time}\cdot TRV$}
\centerline{\text{ }}

Zuletzt wird aus den Normalised Exposure Times der einzelnen Begegnungen die Summe gebildet und man erhält damit das Risiko R\textsubscript{CWA-Positive}:

\centerline{\text{ }}
\centerline{ \text{R\textsubscript{CWA-Positive}} $ := \sum t_n$}
\centerline{\text{ }}

\paragraph{R\textsubscript{CWA-Likely-Positive}}
Das Risiko R\textsubscript{CWA-Likely-Positive} ergibt sich aus den Begegnungen mit CWA-Likely-Positive Personen, 
die eine erhöhte Wahrscheinlichkeit aufweisen, infiziert zu sein und deswegen in der Berechnung gesondert behandelt werden. 
Hierbei werden Begegnungen mit Gruppenangehörigen aus dem Familienfeature (Abschnitt Familienfeature) nicht mit einbezogen, 
da diese sonst aufgrund der potenziell langen Begegnungsdauer und 
kurzen Distanz ein hohes R\textsubscript{CWA-Likely-Positive} Risiko hervorrufen könnten und man im Regelfall diesen Personen ohnehin sofort mitteilt, wenn man ansteckend sein könnte. 
Zudem werden Begegnungen mit Diagnosis Keys nicht mit einberechnet, da diese bereits im Summand R\textsubscript{CWA-Positive} betrachtet werden. Dies geschieht, 
indem bei der Risikoberechnung die Begegnungs RPIs mit den Diagnosis Keys auf dem CWA Server abgeglichen werden und nur die RPIs mit einbezogen werden, 
die keinen passenden Diagnosis Keys auf dem CWA Server liegen haben.
Zuerst wird, wie beim Summand R\textsubscript{CWA-Positive}, das Übertragungspotenzial einer Begegnung bestimmt, indem die Weighted Exposure Time ausgewertet wird. 
Aus der Weighted Exposure Time wird danach allerdings nicht die Normalised Exposure Time bestimmt, wie dies beim Summand R\textsubscript{CWA-Positive} der Fall ist, 
da dafür ein positives Testergebnis vorausgesetzt wird, welches CWA-Likely-Positive Personen nicht vorweisen können.
Anstelle davon wird als nächstes aus der Weighted Exposure Time jeder Begegnung die Standardized Exposure Time $t_s$ ermittelt. 
Da die Begegnung der CWA-Likely-Positive Personengruppe angehört, ist es nicht sicher, dass diese auch Corona hat. 
Aus diesem Grund wird die Positiv-Begegnungs-Wahrscheinlichkeit benötigt, welche die Wahrscheinlichkeit angibt, dass die Begegnung infiziert ist. 
Um die Positiv-Begegnungs-Wahrscheinlichkeit zu erhalten wird jedoch erst eine Funktion f benötigt, die diese Wahrscheinlichkeit angibt. 
Die Funktion $f$ nimmt als Input den Risikowert der Begegnung und gibt die Wahrscheinlichkeit zurück, dass diese infiziert ist. 
Der Risikowert wurde zuvor mit dem TEK zusammen vom PKServer (Abschnitt Feature Potenziell Kranke) heruntergeladen. Um f zu bestimmen wird folgendes Verfahren angewandt:
Zunächst werden Risikowert-Intervalle gebildet, da Risikowerte nicht nur natürliche Zahlen als Werte annehmen können. 
Anhand der gesammelten Daten (Abschnitt Feature Potenziell Kranke) wird für jedes Risikowert-Intervall $x_i mit i \in N$ und einer Intervallgröße von 1 die Wahrscheinlichkeit bestimmt, 
dass eine Begegnung, mit einem Risikowert innerhalb $x_i$, infiziert ist. Dabei gilt für jedes Intervall:

\centerline{\text{ }}
\centerline{$x_i := [13+i, 14+i[$}
\centerline{\text{ }}

Es sind somit nur Risikowerte über 13 relevant, da die Risikowerte auf dem PKServer alle $\le 13$ sind. 
Damit die Wahrscheinlicht bestimmt werden kann, dass eine Person mit einem Risikowert innerhalb eines bestimmten Intervalls $x_i$ infiziert ist, 
wird die Anzahl an Personen benötigt, die einen Risikowert innerhalb des Intervalls $x_i$ hat und die Anzahl an Personen, 
die sowohl einen Risikowert innerhalb des Intervalls $x_i$ hat, als auch später an Corona erkrankt ist. 
Dafür werden die TEKs auf dem PKServer und die Diagnosis Keys auf dem CWA Server verwendet. Aus den TEKs auf dem PKServer kann die Anzahl an Personen bestimmt werden,
die einen Risikowert innerhalb des Intervalls $x_i$ hat.
Werden zusätzlich zu den TEKs auf dem PKServer die Diagnosis Keys auf dem CWA Server dazu genommen, 
kann damit die Anzahl an Personen bestimmt werden, die sowohl einen Risikowert innerhalb des Intervalls $x_i$ hat, als auch später an Corona erkrankt ist, 
indem die TEKs auf dem PKServer und den zugehörigen Risikowerten mit den Diagnosis Keys auf dem CWA Server abgeglichen werden. 
Daraus lässt sich für jedes Intervall $x_i$ die Wahrscheinlichkeit, infiziert zu sein, wie folgt bestimmen:

\centerline{\text{ }}
\centerline{$\frac{\text{Anzahl TEKs auf dem PKServer und CWA Server mit Risikowert} \in x_i}{\text{Anzahl TEKs auf dem PKServer mit Risikowert} \in x_i}$}
\centerline{\text{ }}

Zur Anzahl TEKs auf dem PKServer und CWA Server mit Risikowert $\in x_i$ werden nur die TEKs gezählt, 
die auf PKServer und den CWA Server hochgeladen wurden und einen Risikowert innerhalb $x_i$ aufweisen. 

Zur Anzahl TEKs auf dem PKServer mit Risikowert  $\in x_i$  werden alle TEKs auf dem PKServer gezählt, die einen Risikowert innerhalb $x_i$ aufweisen.
Da in der Regel zwischen dem Hochladen eines TEKs auf den TEK Server und dem Hochladen des gleichen TEKs als Diagnosis Key auf den CWA Server einige Tage vergehen und 

Keys auf beiden Servern nur 14 Tage lang gespeichert werden, muss für Bestimmung der Funktion $f$ die richtige Datenmenge genommen werden. 
Die Anzahl TEKs auf dem PKServer mit Risikowert  $\in x_i$ werden ab dem Startzeitpunkt $s_0$, der 14 Tage in der Vergangenheit liegt, bis zu einem Endzeitpunkt $s_1$, 
der 7 Tage nach $s_0$ liegt, gezählt. Die Anzahl TEKs auf dem PKServer und
CWA Server mit Risikowert $\in x_i$ werden ab dem gleichen Startzeitpunkt $s_0$ bis zum Tag der Bestimmung der Funktion $f$ gezählt.
Dabei muss allerdings der TEK im Zeitraum $s_0$ bis $s_1$ auf den PKServer hochgeladen worden sein. 
Verwendet man zur Bestimmung der Funktion f zum Beispiel die TEKs der letzten 7 Tage auf dem PKServer, 
so werden viele dieser TEKs nicht als Diagnosis Keys auf den CWA Server hochgeladen, 
obwohl diese infiziert sind, da noch kein positives Testergebnis aufgrund der geringen zeitlichen Distanz vorliegt. 
Die davon betroffenen TEKs würden die Funktion $f$ negativ beeinflussen.

Nachdem für jedes Intervall $x_i$ die Wahrscheinlichkeit, infiziert zu sein, bestimmt wurde, wird durch ein Regressionsverfahren die Funktion $f$ bestimmt, 
die jedem Risikowert die Positiv-Begegnungs-Wahrscheinlichkeit zuordnet. Hierbei ist besonders zu beachten, 
dass die Funktion $f$ etwa alle 7 Tage mit aktuelleren Daten neu berechnet und für die CWA aktualisiert werden muss. 
Beim Herunterladen von Keys (vom CWA Server oder PKServer) innerhalb der CWA muss dann gleichzeitig die aktualisierte Funktion $f$ heruntergeladen werden. 
Mithilfe der Funktion $f$ kann dann $t_s$ bestimmt werden: 

\centerline{\text{ }}
\centerline{$t_s := \text{Weighted Exposure Time} \cdot f(ContactRS)$}
\centerline{\text{ }}

Hierbei ist ContactRS der Risikowert des Begegneten.
Als nächstes wird aus den Standardized Exposure Times der einzelnen Begegnungen mit CWA-Likely-Positive Personen die Summe gebildet und das Ergebnis auf 13 begrenzt, 
um daraus das Risiko RCWA-Likely-Positive zu erhalten:

\centerline{\text{ }}
\centerline{$R_{\text{CWA-Likely-Positive}}$ := min($\sum t_s$, 13)}
\centerline{\text{ }}

Die Begrenzung des Summanden R\textsubscript{CWA-Likely-Positive} auf 13 hat zur Folge, 
dass CWA-Likely-Positive Personen nur einen Risikowert $\le 13$ bei anderen Personen hervorrufen können. 
Dies verhindert, dass sich eine Kaskade von CWA-Likely-Positive Personen bilden kann. Nimmt man diese Begrenzung nicht vor, 
ist es möglich, dass eine einzelne CWA-Likely-Positive Person bei einer großen Anzahl an Personen einen roten Bildschirm hervorrufen kann, 
die wiederum bei weiteren Personen rote Bildschirme hervorrufen können und so weiter. 
Durch die Begrenzung kann zwar eine CWA-Likely-Positive Person mit einem hohen Risikowert immer noch rote Bildschirme bei anderen hervorrufen, 
jedoch können die Personen, die dadurch einen roten Bildschirm erhalten haben, diesen nur durch sehr lange Begegnungen, z.B. einen ganzen Tag, bei weiteren Personen hervorrufen. 


\paragraph{R\textsubscript{CWA-Likely-Negative}}
Das Risiko R\textsubscript{CWA-Likely-Negative} (Abbildung 3, grüner Kasten) berechnet alle Begegnungen mit CWA-Likely-Negative und Not-CWA-User Personen mit ein.
Denn aktuell haben nur ein Bruchteil der Infizierten ein positives Test Ergebnis in die App eingetragen. 
Somit muss das Risiko mit den folgenden Hauptpunkten angepasst werden:

\begin{itemize}
	\item Dunkelziffer: Mithilfe der Dunkelziffer ist eine genauere Abschätzung der Anzahl der mit Corona infizierten deutschen Bürger möglich. Denn Covid-19 hat die Eigenschaft nicht immer symptomatisch aufzutreten, es gibt somit viele Infizierte, die nicht aufgezeichnet wurden. Dazu kommt noch, dass nicht alle Personen ihr Testergebnis in die App eintragen. Die Dunkelziffer ist somit  ein Faktor, der beschreibt wie viele Personen mehr erkrankt sind, als positiv getestet wurden. Mithilfe der Dunkelziffer kann somit die größe der Gruppe Hidden-Positive geschätzt werden.
	Die Dunkelziffer wurde erstmals durch eine Studie in Kupferzell (Quelle) nachgewiesen und bestimmt. Da die Datenerhebung dieser Studie vom 20. Mai bis 9. Juni 2020 durchgeführt wurde und sich die Testmöglichkeiten seitdem stark entwickelt haben, ist die damals bestimmte Dunkelziffer von 6 möglicherweise ungenau. 
	\item  AppUsage: Eine Begegnung mit anderen Personen kann nur dann aufgezeichnet werden, wenn diese die Corona Warn App auch installiert haben. Begegnungen mit der Personengruppen Not-CWA-User werden somit nicht aufgezeichnet, sowie auch der Anteil an Infizierten aus dieser Gruppe. Da die derzeitige Nutzung deutlich unter 50% liegt, muss dieser Faktor in die Risikoberechnung mit einfließen, indem die Anzahl der aufgezeichneten Begegnungen nach oben korrigiert wird. Dazu wird die Downloadzahl der CWA (DownloadCount) und die Einwohnerzahl Deutschlands (Population) genutzt. (Quelle)
\end{itemize}

\centerline{\text{ }}
\centerline{$AppUsage := \frac{DownloadCount}{Population}$}
\centerline{\text{ }}

Wie auch beim Risiko R\textsubscript{CWA-Positive} werden beim Risiko R\textsubscript{CWA-Likely-Negative} die Begegnungen der letzten 14 Tage beachtet.
Denn obwohl sich jemand nur mit Personen aus den Gruppen CWA-Likely-Negative und Not-CWA-User getroffen hat
ist es bei einer hohen Anzahl an Begegnungen wahrscheinlicher infiziert zu werden, als bei einer niedrigen, denn in der Not-CWA-User Gruppe gibt es Personen, die infektiös sind. 
Deswegen wird hierfür die Personengruppe Likely-Negative benötigt. 
Hierbei werden, wie beim Risiko R\textsubscript{CWA-Likely-Positive}, 
Begegnungen mit Gruppenangehörigen aus dem Familienfeature (Abschnitt Familienfeature) und Begegnungen mit Diagnosis Keys, 
sowie zusätzlich Begegnungen mit Likely-Positive (Abschnitt Potenziell Kranke)  nicht mit einberechnet, 
da diese bereits in die Risiken R\textsubscript{CWA-Positive} und R\textsubscript{CWA-Likely-Positive} mit einfließen.

Für dieses Risiko wird hauptsächlich mit einer Positiv-Begegnungs-Wahrscheinlichkeit gerechnet,
da man mit dieser die Gruppen Not-CWA-Positive und Hidden-Positive mit einfließen lassen kann.
Dabei wird das Übertragungspotenzial nicht wie bei den anderen Risiken R\textsubscript{CWA-Positive} und R\textsubscript{CWA-Likely-Positive}  berechnet, 
da die Positiv-Begegnungs-Wahrscheinlichkeit einer CWA-Likely-Negative Person zu gering ist. 
Die Positiv-Begegnungs-Wahrscheinlichkeit mit Begegnungen aus der CWA-Likely-Negative Gruppe wird dafür genutzt, 
um den Einfluss der Gruppen Not-CWA-Positive und Hidden-Positive zu berechnen und wird unter anderem durch die Dunkelziffer und die AppNutzung bestimmt.

Um das Übertragungspotenzial trotzdem miteinfließen zu lassen werden alle Begegnungen gezählt bei denen die Signalstärke att Aunter 73 Dezibel ist,
der Abstand somit kleiner als 8 Meter ist und die Begegnungsdauer zwischen einem LowerBound und einem UpperBound ist. 

Damit nicht jede noch so kurze Begegnungen aufgezeichnet wird,
wird eine LowerBound auf 2 Minuten gesetzt. Bei einer einzigen langen Begegnung werden viele Begegnungen mit einer Begegnungsdauer von ca.
15 Minuten aufgezeichnet, da der RPI alle 15 Minuten gewechselt wird. Um dies zu verhindern wird noch einen UpperBound benötigt.
Dafür soll jede Signalstärke mit einberechnet werden, damit kurze Distanzierungen keinen großen Einfluss haben.
Für diesen Wert wird 14 Minuten geschätzt, damit Verbindungsfehler keinen großen Einfluss auf die Begegnungsanzahl haben.
Jede Begegnung mit Begegnungsdauer zwischen Lower- und UpperBound wird somit gezählt.

\centerline{\text{ }}
\text{$if(att<73dB) \le lowerLimit$\{}

\text{\&\&$t(att > 0dB) < upperLimit)$}

\text{$Contacts++$}

\text{\}}

\centerline{\text{ }}

Um die Wahrscheinlichkeit zu berechnen, bei welcher Begegnungsanzahl man sich ansteckt, wird zuerst die aktuelle Anzahl infizierter ausgerechnet, 
um damit die Positiv-Begegnungs-Wahrscheinlichkeit auszurechnen. 
Die aktuelle Anzahl positiv getesteter wird mit der Dunkelziffer multipliziert, 
um die Anzahl aktuell Infizierter zu berechnen. Damit wird currentCWAPositive subtrahiert, da das Risiko bereits in R\textsubscript{CWA-Positive} mit einfließt.  

\centerline{\text{ }}
\centerline{$\text{PersonIsInfected} :=$}
\centerline{$\frac{((\text{currentPositiveTested} \cdot \text{Dunkelziffer}) - \text{currentCWAPositive}}{\text{Population}}$}
\centerline{\text{ }}

R\textsubscript{CWA-Likely-Negative}setzt sich aus dieser Positiv-Begegnungs-Wahrscheinlichkeit zusammen und wird auf 13 beschränkt, 
da die Person gewarnt werden soll, allerdings nicht als hoch Infektiös gilt. 
Dabei wird die Anzahl Contacts mit der AppUsage nach oben korrigiert, um die Anzahl wirklicher Begegnungen CorrectContacts besser abzuschätzen. 
Aktuell beträgt die AppUsage ca. $1/3$ (Quelle), somit werden die Contacts um das Dreifache erhöht. 

\centerline{\text{ }}
\centerline{$\text{CorrectContacts} := \frac{Contacts}{AppUsage}$}
\centerline{\text{ }}

Da mit einem Beschränkten Wachstum gerechnet wird und somit die Positive-Begegnungs-Wahrscheinlichkeit subtrahiert werden muss, 
muss mit der Gegenwahrscheinlichkeit gerechnet werden. Denn umso höher die  Infektionswahrscheinlichkeit, 
desto geringer ist die Gegenwahrscheinlichkeit und desto näher ist R\textsubscript{CWA-Likely-Negative} an der Schranke 13. 
Um die Wahrscheinlichkeit zu berechnen infiziert worden zu sein, wird die Wahrscheinlichkeit, dass eine Person nicht infektiös ist, 
mit der Anzahl an getroffenen Personen CorrectContacts exponiert.

\centerline{\text{ }}
\centerline{\text{R\textsubscript{CWA-Likely-Negative}$=$}}
\centerline{\text{$(1 - ( 1-\text{PersonIsInfected} )\text{CorrectContacts}) \cdot 13$}}
\centerline{\text{ }}

Somit ist es möglich ein erhöhtes Risiko in der App angezeigt zu bekommen, obwohl man keinen Diagnosis Key getroffen hat. 


\subsection{Density-Map}
Eine Density-Map zeigt an, wo sich bestimmte Datenpunkte auf einer Karte häufen. 
Häufig findet sie Anwendung bei Bevölkerungsaufzeichnungen. 
Im Sinne der erweiterten Corona Warn App möchten wir dieses Feature benutzen, um in nahezu Echtzeit die Besucherzahl an öffentlichen Orten oder Geschäften aufzuzeichnen und grafisch auf einer Karte darstellen. 
Damit kann eine Nutzer stark besuchte Gegenden identifizieren und meiden. 
Ein ähnliches Feature wird bereits vom Unternehmen Google angeboten. 
Wer in dessen Suchmaschine nach Geschäften in der Umgebung sucht kann neben den Öffnungszeiten meist auch die live Stoßzeiten und durchschnittliche Aufenthaltsdauer ansehen. 
Google verwendet hierfür jedoch personalisierte Daten seiner Nutzer, wie z.B. den Standort um diese Aufzeichnung vorzunehmen. 
Um die Sicherheit und Privatsphäre der Nutzer zu schützen schlagen wir deshalb eine alternative Lösung mithilfe des BLE-Features der Corona Warn App vor. \\
\subsubsection{Lösungsansatz} 
\text{ }\\
Das zentrale Problem, mit dem wir uns beim Erstellen einer Density Map beschäftigen ist, wie die anonymen Daten gebündelt und auf einen zentralen Server geladen werden können, ohne die Sicherheits-/Privacy Aspekte signifikant einzuschränken/verletzen. 
Ein naiver Ansatz wäre beispielsweise, dass jedes Smartphone seine gesammelte Begegnungsanzahl auf einen zentralen Server hochlädt und dieser diese Daten auswertet. 
Jedoch müssten dafür neben der Standorte der einzelnen Smatphones auch die RPI's aller aufgezeichneten Geräte mit hochgeladen werden, um etwa eine doppelte Zählung bestimmter Endgeräte zu verhindern. 
Eine derartige Sammlung aller Kontaktdaten (auch wenn es sich bei diesen Daten um anonyme RPI's handelt) auf einem zentralen Server würde jedoch der grundsätzlichen Philosophie der Corona Warn App widersprechen.
Diese verfolgte in ihrer Kontaktverfolgung einen möglichst dezentralen Ansatz, um die Privatsphäre der Nutzer zu garantieren.
Dementsprechend suchen wir ebebfalls gezielt nach Verfahren, welche weitmöglichst eine dezentrale Lösung verfolgen. 
Hierfür haben wir uns für die Umsetzung 2 Verfahren überlegt, mit welchen wir das Problem angehen: Eine infrastrukuturbasierte Lösung mittels \textbf{zentraler Masten}, welche in Echtzeit die BLE-Geräte in ihrem festen Empfangsgebiet aufzeichnen, sowie eine infrastrukturlose Lösung \textbf{ohne Masten}. \\ \\
\textit{Infrastrukturbasierte Lösung mit Masten: }\\
Die Idee zentraler Masten besteht darin, dass diese an stark besuchten Orten/Plätzen, wie z.B. Supermärkten oder Haltestellen, ihre Umgebung scannen und für ein gewisses Zeitintervall alle erfassten Smartphones aufzeichnen. 
Nach Ablauf des Zeitintervalls wird die Anzahl der Begegnungen auf einen zentralen Server hochgeladen.
Anders als beim naiven Ansatz, bei dem der zentrale Server die Daten aller Smartphones erfassen würde, ist hier die Aufzeichung auf tausende bis zehntausende einzelne Masten verteilt. 
Dadurch gäbe es keinen "single point of failure", bei dem beispielsweise die Standorte aller Smartphones preisgegeben werden würde. 
Da außerdem die Daten dezentral bei den Masten ausgewertet werden und dann erst auf einen Server hochgeladen, reduziert dies den Datenstrom und die Last des znetralen Servers erheblich. 

Die Begegnungsaufzeichnung erfolgt analog zur Aufzeichnung mobiler Endgeräte untereinander per BLE. 
Somit sind in diesem Aspekt Sicherheit und Privatsphäre gleichermaßen gesichert, wie bei der bisherigen Corona Warn App. 
Außerdem sollen erst ab einem gewissen Threshhold von Aufzeichnungen (z.B. 5 Personen) die Daten auf den Server hochgeladen werden, sowie lediglich die Anzahl der Begegnungen und nicht die einzelnen RPI‘s der Geräte. 
Diese Aggregation der Daten stellt sicher, dass keine einzelnen RPI's oder Personen über die Masten hinweg nachverfolgt und damit identifiziert werden können. 
Abb. \ref{mast_func} zeigt den groben Ablauf eines zentralen Masten. \\ 
\begin{figure}[h]
	\centering
	\includegraphics[width=0.45\textwidth]{"Mast_functionality"}
	\caption{Funktionsweise eines zentralen Masten}
	\label{mast_func}
\end{figure}
\\
Wie bereits erwähnt scannt ein Mast seine Umgebung nach Endgeräten und zeichnet alle Begegnungen innerhalb eines Zeitintervalls von etwa 15 Minuten (Lebensdauer eines RPI's) auf. 
Die Aufzeichnung erfolgt analog zur bisherigen Begegnungsaufzeichnung der Corona Warn App über BLE. 
Es werden demnach nur die RPI's ausgetauscht und keinerlei sensible Nutzerdaten. 
Nach der Aufzeichnung überprüft der Mast, ob es sich bei der Begegnung nicht um ein bereits bekanntes Gerät handelt und erhöht dementsprechend den Counter für die Begegnungsanzahl um eins. 
Nach Ablauf des 15 Minuten Intervalls lädt der Mast die gesammelte Anzahl an Begegnungen auf einen zentralen Server hoch, setzt seine Aufzeichnungsdaten zurück und  fängt wieder von vorne mit seiner Begegnungsaufzeichnung an. 

Da ein Mast unter Umständen einen größeren Bereich abdecken muss als ein normaler Nutzer, sollte das Signal (z.B. mit Repeatern) unter Umständen verstärkt werden, falls dies mit BLE möglich.
Andernfalls müsste ein Ort in mehrere Bereiche unterteilt werden, die dann von einzelnen Masten aufgezeichnet werden können. \\
Hier wird auch deutlich, dass sich die Aufzeichnungsradien der einzelnen Masten möglichst nicht überschneiden sollte, da sonst ein Smartphone von mehreren Masten gleichtzeitig aufgenommen wird.
Eine mögliche Lösung für dieses Problem wäre, die BLE Signale der Masten mit bestimmten Tags zu versehen, sodass ein Smartphone erkennt, falls es sich im Radius von mehreren Masten aufhällt und dementsprechend nur einem Masten seine Daten sendet. 

Da die Installation zentraler Masten an jedem belebteren Platz ein aufwändiges und kostspieliges Unterfangen sein kann schlagen wir vor, dass registrierte/zertifizierte mobile Endgeräte ebenfalls als zentraler Mast fungieren können. 
Hierfür müssen jedoch Einschränkungen gelten, sodass die Funktion als zentraler Mast auf beispielsweise einem Smartphone nicht missbraucht wird.
Folgende Regeln und Restriktionen sollten für alle Masten gelten:
\begin{itemize}
	\item Masten können selbständig von Dritten erworben oder im Falle eines Smarphones, welches als zentraler Mast fungieren soll, die App installiert werden. 
	Jedoch muss vor der Nutzung des Features das Gerät von einer offizienllen Behörde autorisiert werden. Damit darf dann etwa ein Restaurantbetreiber, über eine zentrale Behörde registriert, seine Einrichtung mit seinem Smartphone aufzeichnen. 
	\item Zentrale Masten müssen währende ihrer Nutzung dauerhaft ihren Standort aktivieren und können nur im Umfeld ihres registrierten Bereiches (z.B. Ort des Restaurants) als Mast fungieren. 
	Verlässt ein Mast den für ihn zugewiesenen Aufzeichnungsbereich sollt dieser seine Aufzeichnung automatisch stoppen. Dies verhindert, dass das Feature für unerlaubte Zwecke benutzt wird.
	\item Verletzung der Eigenschaften sollte von anderen Nutzern gemeldet werden können. 
	Beispiele für Verletzung der Eigenschaften wäre etwa eine Person, die in einem unerlaubten Gebiet (z.B. Wohngebiet) einen Masten betreibt oder ein Mast, der viele Begegnungen aufzeichnet, obwohl sich keine Person im Umkreis befindet.  
\end{itemize}

Masten, die ihre Begegnungen aufgezeichnet haben, laden diese auf einen zentralen Server hoch. 
Dieser verarbeitet die gesammelten Daten aller Masten und erstellt die Density Map, welche dann vom Nutzer auf seinem Endgerät abgerufen werden kann.
Abb. \ref{density_creation} zeigt einen grafischen Ablauf zu Erstellung der Density Map mittels zentraler Masten. \\
\begin{figure}[h]
	\centering
	\includegraphics[width=0.45\textwidth]{"Density_Map_Overview"}
	\caption{Übersicht: Erstellen der Density Map}
	\label{density_creation}
\end{figure} 
\\
Wie in Kapitel 2 beschrieben besitzen die RPI's eine Lebensdauer von etwa 15 Minuten. 
Dementsprechend muss beachtet werden, dass Smartphones auch  während des Aufenthaltes im Umkreis eines Masten ihren RPI regelmäßig ändern, wodurch ein Endgerät vom Mast als zwei unterschiedliche Begegnungen erkannt werden kann. 
Der Mast zeichnet dementsprechend zu viele Begegnungen auf und muss seine gemessene Anzahl an Begegnungen nach unten korrigieren, um die Tatsächliche Begegnungsanzahl herauszufinden. 
Wie stark der Wert nach unten korrigiert werden muss hängt hierbei von der Durchschnittlichen Verweildauer eines Smartphones am Masten ab. 
Beträgt die durchschnittliche Verweildauer beispielsweise 15 Minuten (was dem Maximalwert entspricht, da sich nach 15min. der Mast zurücksetzt) wird innerhalb der Aufenthaltsdauer am Masten jedes Endgerät auf jeden fall einmal seinen RPI ändern und als neue Begegnung vom Masten aufgezeichnet werden. 
In diesem Fall würde der Mast doppelt so viele Begegnungen aufzeichnen als er in Realität.
Ist die durchschnittliche Verweildauer hingegen kurz im Verhätlnis zum Änderungsintervall der RPI's (z.B. 3 Minuten) ist die Wahrscheinlichkeit,  dass sich in diesem Zeitintervall  der RPI eines Smartphones ändert sehr klein, was weniger mehrfache Aufzeichnungen zur Folge hat. 
Wie stark ein aufgezeichneter Wert nach unten korrigiert werden sollte wird mittels folgender Formel errechnet: 

\centerline{$\text{IDs} = \frac{\text{RPIs}}{(1+p)}$}

Eine Übersicht infwiefern der gemessene Wert mithilfe der Formel korrigiert wird in Abb. \ref{cor_factor} aufgezeichnet. 
Auf der x-Achse befindet sich die durchscnittliche Verweildauer der Smatphones in Mastreichweite in Minuten, während die y-Achse die Anzahl an Smatphones (ID's) angibt.
Wir gehen davon aus, dass der Mast in jedem Fall genau 100 ID's aufgezeichnet hat (blauer Graph). Je nach Verweildauer muss der gemessene Wert nun mittels Formel nach unten korrigiert werden (orangener Graph).\\
\begin{figure}[h]
	\centering
	\includegraphics[width=0.45\textwidth]{"Korrekturfaktor"}
	\caption{Korrekturfaktor in Abhängigkeit der durchschnittlichen Verweildauer}
	\label{cor_factor}
\end{figure} \\

\textit{Ohne Masten:}\\
Beim Ansatz ohne zentrale Masten besteht die Idee darin, dass jedes Smartphone als „mobiler Mast“ arbeitet und die Funktion eines Masten überehmen kann.
Auch bei diesem Ansatz möchten wir znächst die Daten dezentral bündeln und aggregieren, um den bereits erwähnten Privatsphäre- und Sicherheitsrisiken des naiven Ansatzes entgegenzuwirken.  

Begegnen sich zwei Smartphones tauschen diese, wie bisher in der aktuellen Corona Warn App, ihre RPI's und Daten untereinander aus. 
Zusätzlich wird nun eines der beiden Geräte als Mast ausgewählt (Abb. \ref{mobile_mast}). 
Dieses zeichnet die Begegnung der beiden untereinander auf und übernimmt alle bisherigen Aufzeichnungen des anderen Gerätes in diesem Zeitintervall.
Die Aufzeichnungsdaten werden also transitiv bis Ende des Intervalls weitergegeben. 
Nach Ablauf des Zeitintervalls lädt das Smartphone, welches als mobiler Mast ausgewählt wurde, die aggregierte Begegnungsanzahl mit seinem aktuellen Standort auf den zentralen Server.

Bei dieser Methode können natürlich nur Smartphones, welche ihren Standort aktiv haben, als mobiler Mast arbeiten. 
Im besten Fall hat jedes Handy diese Funktion aktiv. Smatphones, die ihren Standort nicht aktiv haben, werden können von anderen Smartphones mit aktiven Standort aufgezeichnet, jedoch selbst nie als mobiler Mast ausgewählt werden.

Anders als bei zentralen Masten kann nun an jedem Ort, an dem sich ein Smartphone mit der Corona Warn App befindet eine Aufzeichnung für die Density-Map erfolgen. 
Die Datenerfassung der Density-Map ist somit nicht mehr von vorinstallierten zentralen Masten abhängig.\\
\begin{figure}[h]
	\centering
	\includegraphics[width=0.45\textwidth]{"Mobile_Mast"}
	\caption{Smartphones als Mobile Masten}
	\label{mobile_mast}
\end{figure} \\
Beim mobilen Ansatz können diverse Probleme auftreten, welche gelöst werden müssen: Zum Beispiel können zwei Smartphones A und B ihre gesamten Daten immer an das jeweilige andere Handy überreichen, bevor ihr 15 Minuten Intervall abgelaufen ist. 
Somit bleiben die gesammelten Daten für unendlich lange Zeit lokal auf A oder B gespeichert, ohne dass sie jemals hochgeladen werden. 
Eine Lösung für dieses Problem wäre immer das Smarthone als mobilen Mast auszuwählen, das näher am „Reset“, also am Ablauf des 15-Minuten Intervalls ist.

Ein weiteres Problem bei mobilen Masten besteht darin, dass sie sich innerhalb des Zeitintervalls über längere Strecken bewegen können während sie ihre Begegnungen aufzeichnen.
In diesem Fall entpsrechen Ort der Aufzeichnung eines anderen Gerätes und Ort des Hochladens der Daten nach Ablauf des Zeitintervalles nicht überein und das Bild der Density-Map wird verzerrt. 
Dieses Problem wird teilweise bereits durch die Corona App selbst gelöst, welche Begegnungen erst dann aufzeichnet, sobald sich Geräte über etwas längere Zeit in Kontakt stehen. 
Bei sich schnell bewegenden Smartphones müssten sich demnach anderer Smartphones mit gleicher Geschwindigkeit in dieselbe Richtung bewegen, damit eine Begegnung aufgezeichnet wird. 
Das Verkürzen des Zeitintervalls für Uploads auf beispielsweise 5 Minuten würde diesem Problem ebenfalls entgegenwirken, da in kürzeren Intervallen die Smartphones sich im Schnitt weniger weit bewegen können.

Eine weitere Lösung, die wir diesbezüglich vorschlagen ist ein sogenannter „Discount Faktor“. 
Dieser gewichtet indirekte Verbindungen eines mobilen Masten schwächer in die Auswertung der Density Map. \\
\begin{figure}[h]
	\centering
	\includegraphics[width=0.45\textwidth]{"Discount_Faktor"}
	\caption{Discount Faktor für Mobile Masten}
	\label{disc_factor}
\end{figure} \\
In Abbildung \ref{disc_factor} begegnen sich die beiden Smartphones $A$ und $B$. $A$ hat bereits die RPI's $X, Y, Z$ und $W$ aufgezeichnet und B die RPI's $X, Z, V, U$. 
Nun tauschen die beiden Smartphones ihre RPI's gemäß funktionsweise der Corona Warn App aus. Außerdem wird $A$ als mobiler Mast ausgewählt und übernimmt die Aufzeichnungen von $B$ für die Density-Map.
Die RPI's $V$ und $U$ wurden nur von $B$ aufgezeichnet. Dies bedeutet, dass sie sich nicht im direkten Umkreis von $A$ befinden können. $B$ muss sich vor der Begegnung mit $A$ also an einem anderen Ort befunden haben und sich zu $A$ bewegt haben (aus Sicht von $A$).
Dementpsrechend werden die Begegnungen RPI's $U$ und $V$ mit dem Discount Faktor $\gamma$ versehen.
Der Discount Faktor kann auf verschiedene Arten realisiert werden. Er könnte beispielsweise einen Zahlenwert im Intervall $(0,1)$ annehmen. 
Somit würden indirekte Begegnungen mit dem Faktor multipliziert und nicht voll gewertet werden. Im Beispiel wären die zwei indirekten Begegnungen $U$ und $V$ mit gewähltem $\gamma = 0.6$ versehen und nur als $2 \cdot 0.6 = 1.2$ Begegnungen aufgezeichnet werden.
Eine andere Lösung wäre beim Auswerten der Density-Map indirekten Begegnungen mit anderen Standorten zu versehen. 
Während alle direkten Begegnungen von $A$ im unmittelbaren Umkreis seines Standortes auf der Map angezeigt werden $U$ und $V$(beim hochladen aggregiert in 2 Begegnungen) in einem größeren Radius um $A$ angezeigt.
In diesem Fall wäre der Discount Faktor nur eine Markierung und müsste keinen numerischen Wert erhalten. 

Der Discount Faktor kann ebenfalls transitiv weitergegeben werden. 
Sollten etwa die beiden indirekten Begegnungen $U$ und $V$ an ein weiteres Smartphone $C$ übergeben werden, so würden sie bei $C$ doppelt mit dem Discount Faktor versehen werden.\\
\\
\textit{Hybridlösung:}\\
Um die Effekte beider Ansätze nutzen zu können und die Effizienz weitmöglichst zu erhöhen schlagen wir eine Hybridlösung aus zentralen und mobilen Masten vor. 
Die Umsetzung hierfür ist relativ simpel. Wie oben beschrieben werden an häufig besuchten Orten und Plätzen zentrale Masten aufgestellt, um Daten für die Density-Map zu sammeln. 
Überall dort, wo jedoch keine solchen zentralen Masten existieren, greift der dezentrale Ansatz und die Mobiltelefone fungieren als mobile Masten. 
Der im infrastrukuturbasierenden Ansatz beschriebene spezielle Tag für RPI's der Masten kann hier verwendet werden, damit ein Smartphone erkennt, ob er sich zurzeit im Umkreis eines zentralen Masten befindet oder nicht. 
Mittels Hybridlösung könnte man die Vorteile beider Ansätze vereinen und die Nachteile ausgleichen. Zentrale Masten können zuverlässiger einen festen Ort aufzeichnen, da für diese Funktion optimiert sind und sich dauerhaft an diesem Ort befinden.
Mobile Masten hingegen können an jedem beliebigen Ort eine Ansammlung vieler Smartphones und damit vieler Menschen erkennen und auch eine Bewegung großer Menschenmassen aufzeichnen.
Sie sind demnach flexibler, was den Standort angeht.\\
\\
Das letzte zentrale Problem, was nicht nur unsere Lösungsansätze sondern auch die Corona Warn App im allgemeinen stark betrifft ist, dass nicht jede Person ein Smartphone besitzt und die  Corona Warn App auf seinem Smartphone installiert hat. 
Dementsprechend liegen die aufgezeichneten Werte durch die Masten stets unter der tatsächlichen Anzahl der Personen, die sich am entsprechenden Ort befinden. Laut aktuellem Stand (Mai 2021) benutzen etwa 27,8mio (34\%). 
Unter der Annahme, dass die Nutzer der App auf die Standorte bezogen gleichverteilt sind bedeutet dies, dass man die gemessene Anzahl etwa verdreifachen muss, um die tatsächliche Anzahl an Personen zu erhalten.\\

\subsubsection{Simulation}
\text{ }\\
Um die im Lösungsansatz erarbeiteten theoretischen Überlegungen zu verifizieren und deren Umsetzbarkeit nachzuweisen haben wir eine Simulation der Masten programmiert.
Eine reale Umsetzung des Features mit echten Masten und Smartphones erwies sich als zu umfangreich und im Rahmen unseres Forschungsprojektes zeitlich nicht umsetzbar. 
Die Simulation beschränkt sich auf die Umsetzung mittels zentraler Masten. Als Grundlage diente uns hierfür die Simulation "COVID-19 spread simulator“ von Miguel Ángel Durán \cite{Simulation}. \\
\\
\textit{Motivation:} \\
Unser Ansatz der festen Masten basiert darauf, Handys beim Eintritt in einen Mastbereich zu registrieren und ihren RPI zu speichern.\\
Dadurch, dass der RPI periodisch wechselt, werden zwangsläufig Handys doppelt registriert, sollten sie sich zum Zeitpunkt des Wechsels in der Reichweite des Mastes befinden. \\
Die gemessene Anzahl an Handys weicht somit immer zu einem bestimmten Teil von der realen Menge ab. Diesen Wert korrigieren wir nach unten, abhängig von der durchschnittlichen Verweildauer.\\
Die Wahrscheinlichkeit p einer doppelten Registrierung ist definiert durch

\centerline{$p = t_V / t_Z$}

wobei $t_Z$ die Dauer eines Zyklus ist und $t_V$ die durchschnittliche Verweildauer während dieses Zyklus. Die Berechnung der Verweildauer wird in Abschnitt "Aufbau" beschrieben. Da doppelt-registrierte Handys einfach gezählt werden sollen, muss die Hälfte des ermittelten Anteils trotzdem gezählt werden, womit sich die folgende Formel ergibt:

\centerline{Korrigierte Anzahl = Gemessene Anzahl * $(1 – p/2)$}

Abb. \ref{old_form} zeigt den durch diese Formel berechneten Unterschied zwischen dem realen und gemessenen Wert, abhängig von der durchschnittlichen Verweildauer.

\begin{figure}[h]
	\centering
	\includegraphics[width=0.45\textwidth]{"Alte Formel Graph"}
	\caption{Anteil realer Wert am gemessen, abhängig von der durchschnittlichen Verweildauer}
	\label{old_form}
\end{figure}

Das Ziel der Simulation ist es, festzustellen, wie nah der auf diese Weise berichtigte Wert an der tatsächlichen Zahl von Handys liegt.\\

\textit{Aufbau:}\\
Um die Formel zu verifizieren, modifizieren wir die Simulation "COVID-19 spread simulator“.\\
Es werden 1000 Handys simuliert, die als Kreise mit zufälliger Bewegung und Position über ein festgelegtes Gebiet wandern. Auf diesem Gebiet sind 4 Masten, dargestellt durch größere Kreise, welche permanent speichern, welche Kreise in ihrer Reichweite liegen und wie lange dies der Fall ist. \\
Jeder Kreis hat eine sich periodisch verändernde, temporäre ID (RPI) und eine individuelle, permanente ID. Dies ermöglicht es uns, das Zählen von RPI’s zu simulieren und gleichzeitig die korrekte Anzahl zu wissen.\\
Die durchschnittliche Verweildauer wird immer aktuell berechnet nach Ablauf des Zyklus eines Mastes. Dazu wird während des Zyklus mit jeder ID eine Zeitdauer abgespeichert und aus diesem der Durchschnitt ermittelt.  Zur Laufzeit der Simulation speichern die Masten eintretende Kreise neu ab und zählen ab diesem Zeitpunkt einen Counter hoch. Am Ende eines Zyklus werden diese Counter addiert und durch die gemessene Anzahl der Kreise geteilt.\\
Je höher die durchschnittliche Verweildauer $t_V$ des Zyklus ist, desto höher ist die Wahrscheinlichkeit p, dass ein einziges Handy doppelt registriert wird.\\
Zu Verifizierungszwecken haben wir die durchschnittliche Verweildauer mit Hilfe der permanenten IDs berechnet, was einen genaueren Wert liefert. Benutzt man stattdessen die RPIs, verringert sich die durchschnittliche Verweildauer und der korrigierte Wert wird ungenauer. Dies entspricht jedoch der Anwendung in der Praxis, da aus Datenschutzgründen nicht die IDs und ihre Verweildauer gemessen werden.\\
Die Simulation bietet zudem die Möglichkeit, Social Distancing durchzuführen. Dies führt dazu, dass ein Großteil der Kreise keine Geschwindigkeit mehr haben, und nur einige wenige sich überhaupt bewegen.\\

\begin{figure}[h]
	\centering
	\includegraphics[width=0.45\textwidth]{"Simulation"}
	\caption{Darstellung der durchgeführten Simulation}
	\label{sim}
\end{figure}

\textit{Ergebnis:}
Die gesammelten Daten belegen, dass die durch die Formel berechnete Anzahl (orangener Graph) wesentlich näher an der tatsächlichen Zahl (blauer Graph) liegt als die gemessene Menge (grüner Graph). Es fällt jedoch auf, dass in fast allen Fällen die berechnete Zahl immer noch größer ist als die tatsächliche. Nur in einzelnen Fällen wird die Zahl zu niedrig abgeschätzt.\\
Abb. \ref{mast_data_plotted_old} stellt die gesammelten Daten in einem Schaubild dar, ohne das Social Distancing benutzt wird.\\
Hierbei ist der rote Graph die durchschnittliche Verweildauer, der grüne Graph der gemesser Wert, der orangene Graph der korrigierte Wert und der blaue Graph der reale Wert.\\
Mit Social Distancing (Abb. \ref{mast_data_plotted_social_distancing_old}) ist die durchschnittliche Verweildauer wesentlich höher und der angeglichene Wert näher an dem gemessen Wert liegt als in Abb. \ref{mast_data_plotted_old}.

\begin{figure}[h]
	\centering
	\includegraphics[width=0.5\textwidth]{"Mast_Data_Plotted_old"}
	\caption{Ergebnisse ohne Social Distancing}
	\label{mast_data_plotted_old}
\end{figure}

\begin{figure}[h]
	\centering
	\includegraphics[width=0.5\textwidth]{"Mast_Data_Plotted_Social_Distancing_old"}
	\caption{Ergebnisse mit Social Distancing}
	\label{mast_data_plotted_social_distancing_old}
\end{figure}

Hinzu kommt die Beobachtung, dass je höher die durchschnittliche Verweildauer, desto geringer die Differenz zwischen korrigierten und realen Wert. Die Genauigkeit unserer Abschätzung wird demnach besser, je höher die Verweildauer ist.\\

\textit{Schlussfolgerung:}
Aus dem Ergebnis lässt sich die Erkenntnis gewinnen, dass die bisherige Formel nicht optimal ist. Daher haben wir einen neuen Ansatz zur Berechnung gewählt.\\
Die Wahrscheinlichkeit p einer doppelten Registrierung bleibt gleich. Wir gehen davon aus, dass sich mit dieser und der korrekten Anzahl der IDs eines Zyklus die gemessene Anzahl berechnen lässt:\\
\centerline{RPIs = IDs * (1 + p) }
Nach Umformung ergibt sich somit eine neue Formel zur Angleichung unseres gemessen Wertes:\\
\centerline{IDs = RPIs/ (1+p)}\\

\textit{Simulation neuer Formel:}
Um die neue Formel zu testen, benutzen wir dieselbe Simulation wie zuvor, mit der Berechnung des korrigierten Wertes als einzigen Unterschied.\\
Mi Hilfe der neuen Formel ergeben sich folgende Werte:

\begin{figure}[h]
	\centering
	\includegraphics[width=0.5\textwidth]{"Mast_Data_Plotted"}
	\caption{Neue Ergebnisse ohne Social Distancing}
	\label{mast_data_plotted}
\end{figure}

\begin{figure}[h]
	\centering
	\includegraphics[width=0.5\textwidth]{"Mast_Data_Plotted_Social_Distancing"}
	\caption{Neue Ergebnisse mit Social Distancing}
	\label{mast_data_plotted_social_distancing}
\end{figure}


\subsection{Bewertung der Hauptfeatures}

\section{Zusammenfassung und Ausblick}

\section*{Danksagung}

Dankschön an meine Mama und mein Papa und mein Hund Bello, der immer so lieb mit mir gekuschelt hat! Danke an unseren Sponsor
RAID: Shadow Legends und NordVPN!

\begin{thebibliography}{00}
\bibitem{Simulation} Miguel Ángel Durán, COVID-19 spread simulator, 
https://github.com/midudev/covid-19-spread-simulator
\end{thebibliography}
\vspace{12pt}

\end{document}
