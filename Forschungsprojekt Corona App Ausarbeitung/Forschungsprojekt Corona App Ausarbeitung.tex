\documentclass[conference]{IEEEtran}
\IEEEoverridecommandlockouts
% The preceding line is only needed to identify funding in the first footnote. If that is unneeded, please comment it out.
\usepackage{cite}
\usepackage{amsmath,amssymb,amsfonts}
\usepackage{algorithmic}
\usepackage{graphicx}
\usepackage{textcomp}
\usepackage{xcolor}
\usepackage[ngerman]{babel}
\def\BibTeX{{\rm B\kern-.05em{\sc i\kern-.025em b}\kern-.08em
    T\kern-.1667em\lower.7ex\hbox{E}\kern-.125emX}}
\begin{document}

\title{Entwurf und Analyse von Crowd-Sensing Mechanismen für mobile Corona-Warn-Applikationen\\
{\footnotesize \textsuperscript{*}Note: Sub-titles are not captured in Xplore and
should not be used}
}

\author{\IEEEauthorblockN{Gabriel Bonnet}
\IEEEauthorblockA{\textit{Bsc. Science Softwaretechnik} \\
\textit{Universität Stuttgart}\\
Stuttgart, Germany}
\and
\IEEEauthorblockN{Kai Braun} 
\IEEEauthorblockA{\textit{Bsc. Science Softwaretechnik} \\
\textit{Universität Stuttgart}\\
Stuttgart, Germany}
\and
\IEEEauthorblockN{Hannes Deichmann}
\IEEEauthorblockA{\textit{Bsc. Science Softwaretechnik} \\
\textit{Universität Stuttgart}\\
Stuttgart, Germany}
\and
\IEEEauthorblockN{Timm Marquardt}
\IEEEauthorblockA{\textit{Bsc. Science Informatik} \\
\textit{Universität Stuttgart}\\
Stuttgart, Germany}
\and
\IEEEauthorblockN{\textbf{Betreuer:} Dr. rer. Nat. Frank Dürr}
\IEEEauthorblockA{\textit{Institut für Parallele und Verteilte Systeme} \\
\textit{Universität Stuttgart}\\
Stuttgart, Germany \\
frank.duerr@ipvs.uni-stuttgart.de}
\and
\IEEEauthorblockN{\textbf{Prüfer: } Prof. Dr. rer. Nat. Kurt Rothermel}
\IEEEauthorblockA{\textit{Institut für Parallele und Verteilte Systeme} \\
\textit{Universität Stuttgart}\\
Stuttgart, Germany \\
kurt.rothermel@ipvs.uni-stuttgart.de}
}

\maketitle

\begin{abstract}
TODO
\end{abstract}

\begin{IEEEkeywords}
Corona, Corona-Warn-App, Density Map, Risikoberechnung
\end{IEEEkeywords}

\section{Einleitung}
Seit nun mehr als eineinhalb Jahren breitet sich das Virus auf der Erde aus und zwingt die Weltbevölkerung in bis dahin noch nie erlebte Ausnahmezustände.
Laut derzeitigem Stand sind es etwa 173 Millionen Infizierte und 3,7 Millionen Todesopfer \cite{CoronaZahlenWorld}, wobei die Dunkelziffer weitaus höher Werte vermutlen lässt. 
In Deutschland wurdem dem Robert Koch Institut insgesamt 3.7 Millionen infizierte und etwa 90.000 Todesopfer gemeldet \cite{CoronaZahlenDe}. 
Insbesondere stellten dabei das extrem hohe Ansteckungsrisiko und der lange infektiöse Zeintraum von 14 Tagen, in denen ein Infizierter andere anstecken kann, ohne selber Symptome zu zeigen, eine große Herausforderung dar.

Um die Ausbreitung des Viruses weitmöglichst einzudämmen wurden im Land zahlreiche vorkehrungen, wie etwa Kontaktbeschränkungen, Maskenpflicht oder Lockdowns, getroffen. 
Mittlerweile befinden sich auch verschiedene Impfstoffe gegen das Virus im Umlauf, welche eine baldige Rückkehr zur Normalität erhoffen.
Vor der Erforschung und Herstellung der Impfstoffe gilt jedoch die Kontaktverfolgung und Beschränkung als effektivste Methode das Virus einzudaämmen. 
Hierbei wird bei einer infizierten Person versucht, alle Kontakte und damit potenzielle angesteckte Personen, nachzuverfolgen und zu isolieren, damit die Infektionsketten des Viruses unterbrochen werden.
So war man beispielsweise verpflichtet in Resaurant oder Bars seine Kontaktdaten zu hinterlegen, damit man im Falle einer potenziellen Begegnung mit einem Infizierten in der Einrichtung benachrichtigt werden kann.
Eine solche manuelle Kontakverfolgung kann sich jedoch als ein sehr schwieriges und aufwändiges Unterfangen herausstellen, da man sich alle Kontakte innerhalb des langen Zeitraumes von 2 Wochen merken und wieder ins Gedächtnis rufen muss.
Aus diesem Grund wurde vom Robert Koch Institut eine App entwickelt, die diese Kontaktverfolgung automatisieren sollte: Die Corona Warn App \cite{}. Ein zentraler Bestandspunkt der App ist die Privatsphäre, die mittels dezentraler Kontaktverfolgung und anomynen Datenautausches garantiert ist.

In der Zeit der Herausgabe der App gab es viele Diskussionen und Kritik rund um die App. So wurde sie hinsichtlich der Privatphäre öfters kritisiert und musste nachgebessert werden \cite{}. 
Andere Stimmen behaupteten, die App sei überhaupt nicht effektiv und würde nicht nennenswert zur Eindämmung der Pandemie beitragen \cite{}. 

Das Ziel unserer Arbeit ist es, Erweiterungsmöglichkeiten der bisherigen Corona Warn App zu finden und diese zu beurteilen. 
Einen besonderen Fokus legen wir hierbei auf die Crowdsensing Mechanismen der Corona Warn App, da hier von einer Großen Menge an Nutzern über die Smartphones anonymisierte Kontaktdaten untereindander ausgetauscht und verarbeitet werden.
Unsere Erweiterungen sollen zum einen dazu beitragen, die Ausbreitung des Coronaviruses noch effektiver einzudämmen, zum anderen aber auch weitere nützliche Features bieten, die im Hinblick auf das Virus und für die Nutzer interessant und hilfreich wären.
Die Erweiterungen wurden in Haupt- und Nebenfeatures unterteilt. In den Nebenfeatures finden sich eine Reihe von kleineren oder weniger aufwändigen Erweiterungen, wohingegen wir bei den zwei großen Hauptfeatures, der \textbf{erweiterten Risikoberechnung} und der \textbf{Density-Map} noch tiefer in den Aufbau die Funktionsweise der jeweiligen Features hineingehen.
Alle vorgeschlagenen Erweiterungen werden in ihren jeweiligen Abschnitten zusätzlich noch hinsichtlich Sicherheit, Privatsphäre, Nutzen und Aufwand beurteilt. 

Der Rest der Arbeit ist folgendermaßen aufgebaut: In Abschnitt \ref{rel_work} befidnen sich bestehende Arbeiten. Insbesodere gehen wir genauer auf die Funktion der bisherigen Corona Warn App ein. 
In Abschnitt \ref{Nebenfeatures} werden dann eine Reihe von Nebenfeatures beschrieben und beurteilt, bevor es in Abschnitt \ref{Hauptfeatures} weiter mit den beiden Hauptfeatures \textit{Erweiterte Risikoberechnung} und \textit{Density-Map} weitergeht. 
In Abschnitt \ref{conclusion} befindet sich unsere abschließende Zusmmenfassung und ein Ausblick. 

\section{Bestehende Arbeiten} \label{rel_work}

In diesem Abschnitt wird zunächst auf die Corona Lage, sowie auf die funktionsweise der aktuellen Corona Warn App eingegangen, auf die unsere Arbeit basiert. 
Des weiteren wir auf bestehende Arbeiten in Richtung Crowd-Sensing, insbesondere mittels des BLE Features, eingegangen. 

\subsection{Corona Lage}
Bei Corona oder COVID-19 handelt es sich um eine Krankheit, die durch die Coronaviren SARS-CoV ausgelöst wird. 
Anfang 2020 waren diese Viren verantwortlich für die folgende weltweite Corona Pandemie, die aktuell zu weltweit 3 500 000 Toten geführt hat. 
Corona kann symptomatisch sowie asymptomatisch auftreten, wobei Husten und Atemprobleme zu typischen Symptomen zählen.
Übertragen wird Corona durch Schleimhäute, die durch Tröpfcheninfektion oder Aerosole gereizt werden können.
Bei der Tröpfcheninfektion handelt es sich um Infektionen auf kurze Distanz. Durch zum Beispiel Niesen werden Tröpfchen in die Luft geschleudert, 
die innerhalb von 1-2 Metern auf den Boden sinken und in diesem Radius auch Infektionen auslösen können. 
Die Aerosole dagegen sind leichter und können somit länger in der Luft schweben, 
was dazu führt, dass in geschlossenen Räumen eine Ansteckungsgefahr bis zu vier Meter um die infektiöse Person besteht. 
Der Ansteckungszeitraum beträgt 10-20 Tage und sorgt dafür, dass potenziell infektiöse Personen bis zu 14 Tage in Quarantäne kommen.
Mittlerweile wurden Impfstoffe gegen das Virus entwickelt und weltweit 1,4 Milliarden Impfdosen verabreicht, um das Coronavirus einzudämmen

Die Pandemie schränkt das Alltagsleben der gesamten Bevölkerung ein. 
Durch verschiedene Lockdowns war es nicht mehr möglich, nach 20 Uhr das Haus ohne driftigen Grund zu verlassen und sich im öffentlichen Raum ohne Maske aufzuhalten. 
Aufgrund der hohen Übertragungsrate ist es wichtig, Kontakte nachzuverfolgen und Infektionsketten zu unterbrechen.
Nicht nur um die Anzahl Toter zu verringern, sondern auch um die Rückkehr in den Alltag zu ermöglichen, sollte Corona eingedämmt werden.
Aus diesen Gründen wurde nach einer Corona Warn App verlangt, die ermöglichen soll, dass Kontakte automatisch nachverfolgt werden können und gefährdete Personen gewarnt werden.

\subsection{Corona Warn App}
Die deutsche \glqq Corona Warn App\grqq{} (CWA) wurde im Auftrag des Robert Koch Institutes von SAP, der deutschen Telekom und weiteren Unternehmen entwickelt \cite{CWA}. 
Ihr Hauptziel ist die automatisierte und anonyme Kontaktverfolgung mittels Smartphones, um möglichst viele Infektionsketten zu unterbrechen und die Ausbreitung des Coronavirus weitmöglichst zu verlangsamen. 
Zum aktuellen Zeitpunkt ist die App seit etwa einem dreiviertel Jahr aktiv und  wurde knapp 30 Millionen Mal in den App Stores von Apple und Google heruntergeladen (Stand Mai 2021) \cite{Downloads}.

Die Funktion der Corona Warn App ist aus Nutzersicht relativ simpel. Der Nutzer lädt sich die App auf sein Smartphone und aktiviert seine Bluetooth-Funktion. 
Falls er nun mit einer weiteren Person, die ebenfalls die CWA installiert hat, in Kontakt kommt, werden anonym Daten ausgetauscht und die Begegnung auf den jeweiligen Endgeräten gespeichert. 
Sofern man sich nun mit dem Coronavirus infiziert hat, erhält man von einer Teststelle zusammen mit dem positiven Testergebnis einen Code, welchem man in seine App eintragen kann. 
Alle Kontaktpersonen werden dementsprechend informiert, ohne zu wissen, von wem diese Mitteilung letztendlich stammt. 

Die Corona Warn App unterscheidet prinzipiell zwischen zwei Risikostufen, die mithilfe eines grünen bzw. roten Bildschirms auf der Startseite der App angezeigt werden. 
Ein grüner Bildschirm bedeutet hierbei keine/kaum Risikobegegnungen und damit ein vermindertes Risiko, während ein roter Bildschirm ein erhöhtes Risiko anzeigt.

Der Corona Warn App zugrunde liegendes Feature ist das \glqq Exposure Notification Framework", 
welches in Kooperation von Google und Apple für Corona Apps entwickelt wurde und das Bluetooth Low Energy (BLE) Feature nutzt, 
um die Begegnungsdaten der Smartphones anonym untereinander auszutauschen. \\
\\
\textit{Exposure Notification Framework:}\\
Jeden Tag wird vom Framework ein sogenannter Temporary Exposure Key (TEK) generiert. Dieser hält 24 Stunden und ist ein zufällig ausgewählter Schlüssel. 
Aus diesem TEK werden Rolling Proximity Identifiers (RPI) abgeleitet. 
Diese RPI's haben eine Lebensdauer von 10-15 Minuten, bevor sie gelöscht werden und ein neuer RPI aus dem TEK erstellt wird. Wichtig ist hierbei, 
dass sich aus den RPI's nicht der TEK ableiten lässt. 
Die RPI's werden nun unter den Smartphones im Falle einer Begegnung mithilfe von BLE ausgetauscht. Neben den RPI's werden zudem verschlüsselt Informationen über die Begegnungsdauer, sowie dem ungefähren Abstand ausgetauscht.
Falls sich nun eine Person mit dem Coronavirus infiziert, werden dessen TEK (nun Diagnosis Keys genannt) der letzten 14 Tage auf einen zentralen Server hochgeladen. 
Jedes Smartphone hat Zugriff auf diesen zentralen Server und kann die Diagnosis Keys aller Infizierten Personen  herunterladen. 
Mithilfe der Diagnosis Keys können dann die gespeicherten RPI's aus den Begegnungen abgeglichen und im Falle einer Übereinstimmung die Begegnungsdauer und der Abstand entschlüsselt werden. 
Zusammen mit dem Transmission Risk Level (TRL) bestimmt die Corona Warn App aus diesen Informationen das Infektionsrisiko der Person. \\
\\
\textit{Transmission Risk Level (TRL):}\\
Das TRL wird in der CWA lokal berechnet und auf dem CWA Server zu jedem Diagnosis Key abgespeichert, um diesem ein Übertragungs Risiko (TRL) zuzuordnen. 
Dieses Risiko Level beruht auf der Annahme, dass eine infizierte Person ein Übertragungspotenzial hat das anfangs niedrig ist, 
dann auf ein Maximum steigt und innerhalb von 14 Tagen wieder abnimmt. Um daraus ein Risiko Level abzuleiten müssen verschiedene Zeitpunkte bekannt sein. 
Darunter zählen der Symptombeginn, das Infektiös sein, das angesteckt worden sein und drei weiter Zeitpunkte die beim Testen aufgezeichnet werden müssen, 
Zeitpunkt des Testens, Zeitpunkt des Ergebnis bekommen und Zeitpunkt des Hochladens. Aus diesen Werten ist es möglich den TRL so zu bestimmen, 
dass er das Übertragungspotenzial widerspiegelt(Quelle CWADoku).
Allerdings werden diese Zeitpunkte noch nicht bereitgestellt und es wird somit auf einen Standardwert zurückgegriffen. 
Weshalb im Verlauf des Dokuments das Feature TRL Update vorgestellt wird, dass eine Lösung zeigt die fehlenden Daten zu sammeln.

\subsection{Crowd-Sensing und Bluetooth Low Energy}
Die allermeisten technischen Geräte besitzen in der heutigen Zeit die Möglichkeit, sich mit dem Internet zu verbinden und Daten untereinander auszutauschen.
Seien es Smartphones, Laptops oder weitere IoT-Geräte, sie alle besitzen die Funktion die Umgebungdaten mittels Sensoren zu sammeln und diese Daten über das Internet weiterzugeben.
Mobile Crowdsensing beschreibt den Vorgang, ein großes System solcher mobilen Endgeräte aufzubauen, um Sensordaten im großen Maßstab zu sammeln und auswerten zu können \cite{MCS}.
Die Vorteile des Crowd-Sensings liegen auf der Hand: Anstatt Daten im großen Stile aufwändig über ein zentrales Gerät zu sammeln kann die Vielzahl an bereits existierenden externen Geräten genutzt werden, um diese Daten dezentral zu sammeln und verarbeiten.
Somit muss lediglich die Auswertung der Daten auf einem zentralen Server stattfinden, auf denen die einzelnen Geräte ihre Daten senden. 

Bezüglich Crowd-Sensing existieren  bereits umfangreiche Arbeiten aus verschiedensten Anwendungsbereichen. 
So werden beispielsweise Smartphones verwendet, um die Luftverschmutzung in verschiedenen Stadtteilen von Großstädten oder in Gebäuden zu messen \cite{AirQuality}. 
Die Standortbasierten Daten werden dann auf einen Server geladen, damit Nutzer die aktuelle Belastung in den Regionen (oder auch Räumen in Gebäuden) abfragen kann. Eine ähnliche Umsetzung gibt es hinsichtlich der Lärmbelästigung \cite{NoiseSense}. 

Im Verkehrssektor existieren ebenfalls eine Vielzahl von Lösungen, um mithilfe von Crowd Sensing Verkehrsdaten zu sammeln, analysieren und die Lage zu verbessern. 
Saremi et. Al. \cite{TrafficRegulators} Benutzen unter anderem GPS Daten von Smartphones oder Fahrzeugen, um Informationen über die Existenz von Ampeln an Kreuzungen zu extrahieren. 

Anhand dieser Beispiele lassen sich auch die Probleme des Crowd-Sensings herauskristalisieren. Wie etwa wird die Integrität der gegebenen Daten sichergestellt?
Für das Beispiel der Luftverschmutzung könnte beispielsweise der Fall auftreten, dass ein Smartphone bewusst falsche Messdaten an den zentralen Server sendet, ohne dass dieser die Inkorrektheit der Daten feststellen kann.
Die Corona Warn App stellt die Datenintegrität dadurch her, dass Nutzer einen vom Gesundheitsamt bereitgestellten Code eingeben müssen, damit man als Infizierter verifiziert ist \cite{CWA}.
Ein weiteres Problem stellt die Privatheit der Daten dar. Werden beim Crowd-Sensing sensible Nutzerdaten, wie der Standort mitgeliefert, könnte damit unter Umständen eine Menge von Nutzerdaten auf einem zentralen Server gesammelt und missbraucht werden.
Auch in diesem Fall bietet die Corona Warn App Schutzt, indem nur anonyme TEK's auf den zentralen Server geladen werden und sensiblere Daten, wie die Begegnungsdauer und Entfernung dezentral und verschlüsselt auf den einzelnen Smarphones vorliegen.\\
\\
\textit{BLE Crowd Sensing:}
Auch im Bereich BLE existieren Crowd Sensing Anwendungen zum Erfassen von Daten und Erstellen von Density-Maps. 
Tschuhiya et al. \cite{BLESensorDevices} beschäftigen sich mit der allgemeinen Nutzbarkeit des BLE Features zum Erstellen von Maps. 
Anstelle von Smartphones werden hier Arduinos verwendet, um verschiedene Sensordaten, wie Druck, Luftfeuchtigkeit, Lichtintensität usw. zu messen. 
Diese Daten werden dann per BLE an die Smartphones weitergegeben, wo sie ausgewertet und dann auf einen Server hochgeladen wurden. 

In \cite{CICN} werden Smartphones und vorinstallierte Raspberry Pies mit CICN (Community Information-Centric Networking)  verwendet, um eine IoT Sensoren Map innerhalb geschlossener Gebäuden, wie etwa Shopping Malls, zu erstellen. 
Die Raspberry Pies werden in den einzelnen Räumen vorinstalliert und messen Sensordaten, wie Temperatur oder Luftfeuchtigkeit. 
Außerdem werden sie mit BLE Beacons versehen, damit Nutzer per Smartphone automatisch die Daten auswerten und eine Übersichtskarte des Gebäudes erstellen kann.  

Basalamah \cite{BLETags} beschäftigt sich mit Bluetooth Low Energy Tags zum Aufzeichnen von Bewegungsströmen größerer Menschenmassen. 
Die BLE Tags stoßen regelmäßig Signale aus, die sich mithilfe von Smartphones empfangen lassen. 
Dadurch kann der Standort und Status der einzelnen Tags gezielt nachverfolgt werden. 
Im Versuch wurden Pilger beim Haddsch (Pilgerfahrt nach Mekka) mit diesen BLE Tags ausgestattet und deren Bewegung mittels Smartphones, welche an bestimmten Orten als Aufzeichnungsmast agierten, nachverfolgt. 
An diesem Ansatz ist unsere Density-Map Lösung mittels zentraler Masten stark angelehnt, jedoch bestand die Idee hier mittels der Smartphones gezielt einzelne Tags zu identifizieren und daraus folgend die Bewegungsströme der Pilger nachzuvollziehen. 
Unser Ansatz hingegen ist es, per BLE möglichst anonym die Daten zu erfassen und damit eine Density-Map zu erstellen. 
Außerdem verwenden wir keine speziellen BLE Tags sondern die Smartphones selbst zum Datenaustausch. 

Ein Feature, das unserer Density-Map am nächsten kommt sind vermutlich die live Stoßzeiten des Unternehmens Google \cite{GoogleStosszeiten}.  
In dessen Suchmaschine lässt sich zu fasst jeder öffentlichen Einrichtung, sowie Restaurants oder Einzelhandel Läden neben den Öffnungszeiten die aktuelle Besucherzahl anzeigen. 
Zudem wird noch die Durchschnittliche Aufenthaltsdauer von Personen, sowie eine Abschätzung der Auslastung über den Tag verteilt mitgegeben. 
Die Besucherzahlen werden nicht als konkreter numerischer Wert angegeben, sondern mithilfe von Balkendiagrammen grafisch angezeigt, sodass man lediglich die Auslastung mit anderen Uhr-/Tageszeiten vergleichen kann, ohne den konkreten Wert zu kennen. 
Das Live Stoßzeiten Feature wird jedoch nur als Diagramm für jede Einrichtung und nicht als Karte realisiert.
Für die Umsetzung ihres Features benutzt Google voraussichtlich die personalisierten Nutzerdaten seiner User, wie beispielsweise den Standortverlauf der Google Maps Anwendung \cite{GoogleStandortverlauf}. 
Auf deren Webseite gibt Google keine weiteren Informationen darüber Preis, lediglich wie das Feature funktioniert. 

Zusammenfassend lässt sich sagen, dass es eine Fülle von Arbeiten gibt, welche sich mit dem Thema Crowd Sensing beschäftigen. 
Die meisten verwenden jedoch personalisierte Daten der Nutzer, wie den Standort, um entsprechende Karten zu generieren. 
Location-Privacy ist ein aktuelles Forschungsgebiet und es existieren eine Reihe von Lösungen, die dieses Problem behandeln \cite{Location}. 
Die Corona Warn App hingegen verfolgt den Ansatz der kompletten Pseudonymisierung der Daten, bei dem der Nutzer zu jeder Zeit hinter einem zufällig generierten RPI verborgen ist. 
Auch bei der Nutzung von BLE wurden sensible Daten gezielt mitgegeben, um konkrete Standorte nachzuverfolgen, anstatt die Anonymität des BLE features gezielt auszunutzen. 
Das Erstellen einer Density-Map mittels anonymer Nutzerdaten und BLE scheint es nach unserem Wissen demnach nicht zu geben. 

\section{Lösungsvorschläge} \label{Nebenfeatures}
In diesem Abschnitt werden die Erweiterungen der Corona Warn App präsentiert, die im Verlauf der Forschungsarbeit von uns entwickelt und analysiert wurden. 
Ausgenommen sind die beiden Hauptfeatures \textbf{Erweiterte Risikoberechnung} und \textbf{Density-Map}, welche separat in Abschitt \ref{} und \ref{} beschrieben sind.
Da das Thema Corona derzeit hochaktuell und präsent ist, wurden einige der vorgestellten Erweiterungen im Verlauf unserer Arbeit von anderen Parteien bereits in ähnlicher Form umgesetzt.
Auch die Corona Warn App befindet sich kontinuierlich en Entwicklung wodurch die von uns hier vorgeschlagenen Features in der akutellen Version der App bereits vorhanden sind oder in naher Zukunft erscheinen sollen. 

\subsection{Bewertungskriterien}

Um einen möglichst einheitlichen Vergleich der Features zu garantieren, haben wir sechs Bereiche festgelegt, in denen die Features beschrieben werden.\\
Diese Bereiche sind wie folgt aufgeteilt:\\
\textit{Privatsphäre:} Wie stark greift das Feature in die Privatsphäre ein?\\
\textit{Nutzen:} Welchen effektiven Nutzen bringt das Feature und lohnt sich der Aufwand für den effektiven Nutzen?\\
\textit{Umsetzbarkeit:} Wie realistisch ist es, dieses Feature zu implementieren in Anbetracht von Datenschutz, Programmierung oder anderen Hindernissen?\\
\textit{Akzeptanz:} Da ein möglichst hohe Nutzung der App in der Bevölkerung erzielt werden soll, ist es wichtig, dass das Feature entweder einen Anreiz zur Nutzung schafft oder zumindest diesen nicht schwächt. Die Frage ist also, ob die neue Funktion Skepsis oder Missbilligung beim Nutzer auslöst.\\
\textit{Sicherheit:} Wie gut lässt sich das implementierte Feature vor Missbrauch und Angriffen schützen?\\
\textit{Daten:} Welche Daten werden mindestens benötigt, dass das Feature funktioniert?\\


\subsection{Übersicht aller Erweiterungen}

Im Folgenden werden alle Erweiterungen, welche uns zu Beginn der Forschungsarbeit als tiefergehendes Feature möglich erschienen, aufgezählt und nach unseren in Abschnitt III.A genannten Kriterien beschrieben.\\

\textit{Impfpass:} \\ 
In der aktuellen Situation ist es wichtig, nachweisen zu können, dass man geimpft wurde. Dies könnte man entweder dadurch regeln, einen analogen Impfnachweis immer bei sich zu tragen, oder durch einen digiatalen Impfpass in der Corona-Warn-App.\\
Größter Vorteil eines in der App integrierten Nachweises wäre die Digitalisierung des Verwaltungswesens und dadurch eine Vereinfachung des Organisationsaufwands.\\ Anstatt mehrere Unterlagen mit sich herumzutragen, hat man diese stattdessen gesammelt in einer App.\\
Digitalisierung ist zudem ein positiv konnotierter Begriff, welcher einen Anreiz für die Bevölkerung schafft, die App zu nutzen. Dies ist immer ein relevanter Nutzen, da es das Ziel ist, die Corona-Warn-App von 100\% der Bevölkerung nutzen zu lassen. \\
Implementiert man zusätzlich dazu eine Art Krankenakte, in der man Merkmale wie Gewicht und Vorerkrankungen einträgt, liese sich dadurch die Impfpriorisierung und die Reihenfolge in der Bevölkerung besser organisieren.\\
Benötigte Daten für einen generellen Impfpass wären:\\
- Name \\
- Krankenkasse\\
- Durchgeführte Impfungen\\
Für die Einschätzung der Impfpriorisierung wäre zusätzlich erforderlich:\\
- Alter\\
- Vorerkrankungen \\
- Gewicht \\
- Beruf/Arbeitsumfeld\\
Bei einem unverbindlichen Impfpass wären diese Angaben nicht wirklich nutzbar außer für den Endnutzer, da man auf die Eigenverantwortung des einzelnen Nutzers angewiesen wäre, die Daten also einfach gefälscht werden können.\\
Stattdessen müsste man, um die Daten als sicher zu garantieren, die Bearbeitung des Impfpass beschränken, sodass nur offizielle Stellen wie Krankenhäuser oder Arztpraxen diesen bearbeiten können.\\
Somit wird Richtigkeit der Daten gewährt, welche dann für das Krankenwesen genutzt werden können.\\

\textit{Gamification:}\\
Eine Möglichkeit, das Problem einer zu niedrigen Nutzung der App zu adressieren, ist Gamification. Dadurch versuchen wir, den Anreiz zu steigern, die App herunterzuladen und auch aktiv zu benutzen.\\ 
Gamification der App kann es in verschiedenen Varianten geben. Zum Beispiel lassen sich simple Spielmechaniken wie persönliche Highscores einbauen, welche man mit anderen Nutzern der App vergleichen kann.  Dafür bieten sich Statistiken an, die sich bereits aus der Nutzung der App erstellen lassen. Dazu zählen die getroffenen Personen an einem Tag, die durchschnittliche Dauer einer Begegnung oder der ermittelte Risikofaktor. Ziel der Highscores wäre es dann, diese Werte so möglich wie gering zu halten oder im Vergleich zu anderen Nutzern bessere Werte zu haben.\\
Vorstellbar sind auch daraus resultierende optische Veränderungen der App durch neue Farben oder ein anderes Design.\\
Dieses Feature richtet sich vermutlich an eine jüngere Zielgruppe, die mit solchen Mechaniken meist mehr vertraut ist als ältere Generationen.\\
Es könnte sogar sein, dass dieses Feature als Verharmlosung der ernsten Thematik interpretiert wird, was in einem gegensätzlichen Effekt zur eigentlichen Intention resultieren würde.\\
Eine Alternative zu Highscores und anderen virtuellen Errungenschaften wäre eine Art Belohnung in der Realität, zum Beispiel durch eine in der App durchgeführte Lotterie.\\
Dies ist vermutlich attraktiver für alle Generationen, jedoch schwieriger umzusetzen, da hier Betrug oder mehrmalige Teilnahme verhindert werden muss. Vorstellbar wäre auch eine Verbindung mit den ermittelten Statistiken, sodass Leute mit guten Werten eine höhere Gewinnchance haben.

Eine Umsetzung dieses Features wurde im Laufe dieser Forschungsarbeit bereits in den USA, als Lotterie im Rahmen der Impfkampagne eingeführt \cite{Lotto}.
Jeder Geimpfte hat dabei eine Chance den Hauptgewinn von einer Millionen US-Dollar zu gewinnen.\\

\textit{Ortsbezogene Informationen:}\\
Ein Problem das im Verlauf der Pandemie auftritt ist die undurchsichtige und inkonsistente Informationsgebung 
hinsichtich der beschlossenen Maßnahmen und Einschränkungen an die sich die Bevölkerung richten muss. 
Viele dieser Maßnahmen wurden nicht auf Bundesebene beschlossen, sondern unabhängig von den Budnesländern, je nach Ausmaß der Krankheitsfälle im jeweiligen Bundesland.
Selbst bei der Bundesweiten Notbremse bezogen sich die Regelungen auf die Inzidenz der einzelnen Landkreise. \cite{Notbremse}  
Dies führte zu einer großen Verwirrung welche derzeitigen Gesetze und Verordnungen gelten, zumal diese teilweise im Tagesrythmus geändert wurden.

Auf Grund dessen wäre ein Feature hilfreich, das die aktuellen Verordnungen abhängig vom jeweiligen Landkreis darstellt und einen Überblick verschafft.
Dabei können die Nutzer unabhängig von ihrem Standort auf die Daten eines jeweiligen Landkreises zugreifen,
um auf einfachem Weg die offiziellen Informationen zu erhalten. 

Dieses Feature existiert bereits in inoffizieller Ausführung, als \glqq Darf ich das? \grqq{} -App von der InTradeSys GmbH. \cite{DarfIchDas}\\

\textit{Integration von Applikationen:}\\
Es existieren viele verschiedene Plattformen, die fast täglich neue Nachrichten zur aktuellen Corona-Situation herausgeben \cite{RKI}, \cite{BMG}, Daten zur Eindämmung der Pandemie sammeln \cite{RKIDatenspende}, \cite{Gedaechtnistest} oder Neuigkeiten zu Forschungsarbeiten mit Bezug zu Corona veröffentlichen \cite{GesundheitsforschungNews}.
In diesem Feature werden Nachrichten, Datensammlungen, Umfragen und Forschungsarbeiten als \textit{Informationen} zusammengefasst.
Die CWA ist der perfekte Ort all diese Informationen der Plattformen zu vereinen. 
Da sie von der Bundesregierung in Auftrag gegeben wurde, besitzt sie ein hohes maß an Integrität und Datenschutz steht im Vordergrund.

Es ist Ziel dieses Features die CWA so zu erweitern, dass sie Anlaufstelle für jede Person wird, die relevante und qualitativ hochwertige Informationen zu Corona sucht.
Dabei steht im Vordergrund, dass die Informationen geprüft sein sollen, um die Verbeitung von falschen Informationen zu vermeiden. 
Dazu dürfen nur vertrauenswürdige Quellen verwendet werden, wie zum Beispiel die Bundesregierung, das Robert-Koch-Institut oder Forschungsinstitute. 
Die CWA soll so erweitert werden, dass die Informationen zwar weiterhin auf den eigenen Webseiten der Herausgeber veröffentlicht werden, jedoch zusätzlich in die CWA eingebunden werden.

Durch das Zusammenführen verschiedenster Informationen wird eine sehr große Reichweite erzeugt, die gerade für die Datensammlung und Forschung große Relevanz hat.
Desweiteren würde die CWA durch die zusätzlichen Informationen attraktiver, was einen positiven Effekt auf die Kontaktverfolgung hätte.

In Version 1.13 der CWA wurde bereits eine freiwillige Datenspende in Zusammenarbeit mit dem Robert-Koch-Institut eingefügt \cite{freiwilligeDatenspende}. 
\section{Erweiterte Risikoberechnung} \label{Erweiterte Risikoberechnung}

%Aktuelle Risikoberechnung
Die Risikoberechnung der aktuellen Corona Warn App zeigt einem Nutzer zwei Risiken an, 
das erhöhte Risiko mit einem roten Bildschirm und das geringe Risiko mit einem grünen Bildschirm. 
Generell wird angenommen, dass wegen der Inkubationszeit, Tests beziehungsweise Begegnungen innerhalb der letzten 14 Tage stattgefunden haben.
Um dieses zu berechnen wird das \textit{Übertragunspotenzial} aus Begegnungsdauer und Begegnungsnähe mit einer Person, 
die ein positives Testergebnis in die CWA eingetragen hat, mit einbezogen. Und zum anderen die Wahrscheinlichkeit, 
dass eine Person infiziert ist, die im folgenden als \textit{Positiv-Begegnungs-Wahrscheinlichkeit} bezeichnet wird.
Wenn das Übertragunspotenzial sich insgesamt zu einer Begegnungsdauer von 13 Minuten in weniger als 8 Metern Abstand aufsummiert, 
wird der Bildschirm rot, ein erhöhtes Risiko angezeigt. Die Positiv-Begegnungs-Wahrscheinlichkeit ist in der aktuellen CWA immer 1, 
da es sich in jedem Fall um Begegnungen mit infizierten Personen handelt.
Diese Berechnung hätte einen hohen Nutzen, wenn mehr Personen die App herunterladen würden und mehr Tests zur Verfügung gestellt werden könnten. (Quelle 123)
Je mehr Tests durchgeführt werden und auch in die CWA eingetragen werden, umso genauer wird die aktuelle Risikoberechnung. 
Die aktuelle Risikoberechnung der CWA ist in Abbildung \ref{Risikoberechnung_Abbildung} zu sehen.

\begin{figure*}[h]
	\centering
	\includegraphics[width=0.9\textwidth]{"Aktuelle_Risikoberechnung"}
	\caption{Aktuelle Risikoberechnung \cite{Risikoberechnung}}
	\label{Risikoberechnung_Abbildung}
\end{figure*}

%Definitionen
Mit der Risikoberechnung wird das Übertragungspotenzial und die Positiv-Begegnungs-Wahrscheinlichkeit der Begegnungen ermittelt,
um damit die Gefahr infiziert worden zu sein, abzuschätzen.
Aufgrund der aktuellen Test-Möglichkeiten und Downloadzahlen sollte die Risikoberechnung jedoch angepasst werden.
Dafür wird die Risikoberechnung der CWA erweitert, 
um damit einen Risikowert zu berechnen, der die Gefahr besser widerspiegelt.

Dabei wird zwischen zwei Personengruppen, die jeweils in 3 Untergruppen unterteilt werden, unterschieden, 
da die Positiv-Begegnungs-Wahrscheinlichkeit von Person zu Person stark variieren kann.
Es wird unterschieden zwischen \textit{CWA-User}, die von der CWA somit aufgezeichnet werden und \textit{Not-CWA-User}, 
die die CWA nicht aktiv nutzen und somit nicht aufgezeichnet werden. Die CWA nicht aktiv nutzen heißt, dass keine TEKs geteilt werden.
Die CWA-User werden in drei Untergruppen aufgeteilt: 
\textit{CWA-Positive} sind alle Personen, die einen positiven Test in die CWA eingetragen haben,
\textit{CWA-Likely-Positive} sind alle Personen, die mindestens eine Begegnung mit CWA-Positive hatten und einen roten Bildschirm auf der CWA angezeigt bekommen und
\textit{CWA-Likely-Negative} sind alle Personen, die meist keine Begegnung mit CWA-Positive hatten und somit einen grünen Bildschirm auf der CWA angzeigt bekommen.
Ebenso wird die NOT-CWA-User in drei Untergruppen aufgeteilt: \textit{Not-CWA-Positive} sind alle Personen, die ein positives Test Ergebnis haben, 
dieses aber nicht in die CWA eingetragen haben (Es ist möglich, dass sie die CWA trotzdem passiv nutzen), \textit{Hidden-Positive} sind alle Personen, die infiziert wurden, 
jedoch kein positives Testergebnis nachweisen können und \textit{Hidden-Negative} sind alle Personen, die nicht infiziert sind.

%Probleme und Lösungsvorschläge
Bei näherer Betrachtung der Risikoberechnung der CWA sind uns mehrere Kritikpunkte aufgefallen.
Zum einen fließen nur Begegnungen mit CWA-Positive Personen in diese mit ein.
Deshalb sollen alle Personengruppen miteinbezogen werden.
Zum anderen wird die Dunkelziffer nicht berücksichtigt. Diese soll im Folgenden durch die Begegnungsanzahl betrachtet werden.
Desweiteren soll die geringe Nutzerzahl der CWA zum Hochkorrigieren der Begegnungsanzahl genutzt werden.
Als letzten Kritikpunkt wird die Granulierung und Transparenz der Risikoberechnung aufgezeigt. 
Aus diesem Grund soll der Risikowert in vier unterschiedliche Risikostufen (grün, gelb, orange, rot) aufgeteilt werden, dessen Zustandekommen mit angezeigt wird.

Für die erweiterte Risikoberechnung werden die Features Familienfeature, Feature TRL Update und
Feature PK-Server benötigt.

Das Familienfeature verhindert das mehrmalige Aufzeichnen von vertrauenswürdigen Personen,
damit die Dunkelziffer besser miteinberechnet werden kann.
Mit dem Feature TRL Update ist es möglich den Krankheitsverlauf in die Risikoberechnung mit einfließen zu lassen,
um zu beachten, wie infektiös eine Person ist.
Dabei ist dies nicht mit dem im Abschnitt \ref{Erweiterte Risikoberechnung} definierten Begriff Übertragungspotenzial zu verwechseln.
Das Feature PKServer wird benötigt, um Begegnungen mit CWA-Likely-Positive Personen zu erkennen und diese in der Risikoberechnung höher zu gewichten.

Diese Features werden in Abschnitt 1 bis 3 vorgestellt.
Im Abschnitt 4 werden Probleme und Lösungsansätze der aktuellen Risikoberechnung diskutiert und die drei Summanden, aus denen der Risikowert zusammengesetzt ist, genauer erläutert.
Im 5. Abschnitt wird die erweiterte Risikoberechnung bewertet.
\\
\subsubsection{Familienfeature}
\label{Familienfeature}
Im Folgenden wird das Familienfeature vorgestellt, welches später dafür genutzt wird die Risikoberechnung innerhalb der CWA zu verbessern.
Das Familienfeature ermöglicht es Gruppen von Verwandten bzw. Freunden zu erstellen und mit diesen Temporary-Exposure-Keys zu teilen.
Mithilfe der geteilten TEKs ist es möglich von Freunden bzw. Verwandten empfangene Rolling Proximity Identifier bei der Risikoberechnung gesondert zu behandeln.
Dies geschieht, indem bei der Risikoberechnung die Begegnungs RPIs mit den TEKs aus den einzelnen Gruppen abgeglichen werden und nur die RPIs mit einbezogen werden, die keinen passenden TEK in irgendeiner Gruppe haben.
Insbesondere heißt dies, dass Begegnungen mit Freunden bzw. Verwandten nicht in R\textsubscript{CWA-Likely-Positive}, Abschnitt (CWA-Likely-Positive) und R\textsubscript{CWA-Likely-Negative}, Abschnitt (RLikely-Negative), miteinberechnet werden. 

Das Austauschen von TEKs kann allerdings Eingriffe in die Privatsphäre zur Folge haben.
Erkrankt eine Person aus einer Gruppe an Corona und gibt ihre TEKs als Diagnosis Keys frei, ist es für die anderen Gruppenangehörigen möglich mit hoher Wahrscheinlichkeit die erkrankte Person zu bestimmen.
Aus diesem Grund sollten Gruppen nur mit vertrauenswürdigen Personen, wie engen Freunden oder Verwandten, gegründet werden.
In Abbildung \ref{Familienfeature_Abbildung} ist eine Übersicht des Familienfeatures gegeben, welches im Folgenden beschrieben wird:

\begin{figure}[h]
	\centering
	\includegraphics[width=0.48\textwidth]{"Familiengruppe"}
	\caption{Familienfeature}
	\label{Familienfeature_Abbildung}
\end{figure}

Als Servergruppe wird eine Gruppe auf einem speziell für das Familienfeature angelegten Server bezeichnet.
Eine Servergruppe besitzt eine für den Server einmalige ID, einen private signing key und eine Liste mit verschlüsselten TEKs.
Die Liste der verschlüsselten TEKs ist beim Anlegen einer neuen Servergruppe leer.

Jeder Nutzer der CWA kann beliebig viele lokale Gruppen in der eigenen CWA erstellen.
Innerhalb einer lokalen Gruppe wird ein private signing key, ein private encryption key und eine Servergruppen ID gespeichert.
Die Verwendung des private signing keys, private encryption keys und der Servergruppen ID wird in den nächsten Absätzen genauer erläutert.
Der private signing key und insbesondere private encryption key einer Gruppe müssen geheim gehalten werden, damit die Privatsphäre keines Mitglieds verletzt wird.
Es gibt zwei verschiedene Möglichkeiten den private signing key, private encryption key und die Servergruppen ID einer lokalen Gruppe hinzuzufügen.
Dafür kann man entweder eine neue Servergruppe anlegen oder einer bereits existierenden Servergruppe beitreten.

Entscheidet man sich dazu eine neue Servergruppe anzulegen, so wird lokal in der CWA ein private signing key und ein private encryption key generiert.
Der private signing key wird zusammen mit einer Anfrage, eine neue Servergruppe zu erstellen, an den Server gesendet.
Ein Verlust des private signing keys kann dazu führen, dass ein Angreifer willkürlich TEKs hochladen und herunterladen kann, die dieser Gruppe zugeordnet werden.
Allerdings reicht der private signing key nicht aus, um die TEKs von Gruppenmitgliedern lesen zu können, da diese mit dem private encryption key verschlüsselt sind.
Der Server erstellt auf die Anfrage hin eine neue Servergruppe mit einer noch nicht verwendeten ID und schickt die ID an den Anfragesteller zurück.
Der lokal erstelle private signing key, private encryption key, sowie die erhaltene Servergruppen ID werden der Gruppe zugehörig abgespeichert.

Im Falle der zweiten Möglichkeit, einer bereits existierenden Servergruppe beizutreten, benötigt man ein Mitglied der Servergruppe.
Das Mitglied der Servergruppe teilt den private signing key, private encryption key und die Servergruppen ID durch zum Beispiel einen QR-Code.
Diese müssen dann nur noch abgespeichert werden.

An dieser Stelle ist anzumerken, dass jedes Mitglied neue Personen zur Gruppe hinzufügen kann, ohne, dass andere Mitglieder etwas davon mitbekommen.
Aus diesem Grund wird nochmal betont Gruppen nur mit vertrauenswürdigen Personen zu teilen, um die Privatsphäre zu schützen.
Sollte eine Person unbefugten Zugriff auf eine Gruppe haben, ist es nicht möglich, dieser Person den Zugriff zu entziehen.
In diesem Fall muss die lokale Gruppe gelöscht werden, damit keine weiteren TEKs geteilt werden.

Hat man ein oder mehrere Gruppen lokal erstellt und die zugehörigen private signing keys, private encyrption keys und Servergruppen IDs hinzugefügt, werden automatisch alle 24h die eigenen TEKs auf den Server hochgeladen und die TEKs der anderen Mitglieder heruntergeladen.
Damit außer der Gruppenmitglieder niemand Zugriff auf die TEKs hat, wird dabei folgendes Verfahren angewandt:

Die TEKs werden mithilfe des private encryption keys einer Gruppe und eines sicheren, symmetrischen Verschlüsselunsverfahrens verschlüsselt und dann mit dem private signing key der Gruppe und einem sicheren, symmetrischen Signaturverfahren signiert.
Dies wird zusammen mit der Servergruppen ID der Gruppe an den Server gesendet.
Der Server nutzt den private signing key zugehörig zur mitgeschickten ID, um die TEKs zu validieren.
Sind die TEKs validiert, werden sie der Liste der verschlüsselten TEKs zusammen mit ihrem Hochladedatum hinzugefügt.
Die verschlüsselten TEKs werden nach 14 Tagen aus der Liste der verschlüsselten TEKs gelöscht.
Dies geschieht iterativ für jede lokale Gruppe.

Um die TEKs der Gruppenmitglieder einer Gruppe herunterzuladen, muss eine Anfrage an den Server gestellt werden.
Diese Anfrage beinhaltet die Servergruppen ID der Gruppe und das Datum des letzten Downloads.
Mithilfe des private signing keys wird die Anfrage validiert und die verschlüsselten TEKs vom Server an die CWA gesendet, die nach dem Datum des letzten lokalen Downloads hochgeladen wurden.
Die verschlüsselten TEKs können dann mit dem private encryption key der Gruppe entschlüsselt werden.
Auch hier geschieht dies iterativ für jede lokale Gruppe.\\


\subsubsection{Feature PKServer}

Das Feature PKServer ist Voraussetzung für die in Abschnitt (Risikoberechnung) eingeführte erweiterte Risikoberechnung.
In der aktuellen CWA werden nur Begegnungen mit CWA-Positive Personen in der Risikoberechnung betrachtet.
Jedoch sind CWA-Likely-Positive Personen mit einer erhöhten Wahrscheinlichkeit infiziert und ansteckend, da sie Kontakt zu CWA-Positive Personen hatten.
Diese sollen mithilfe dieses Features in die Risikoberechnung in Abschnitt (R-CWA-Likely-Positive) mit einbezogen werden.

Damit diese in die Risikoberechnung mit einfließen können, benötigen wir die TEKs der CWA-Likely-Positive Personen.
Ähnlich wie beim CWA Server sollen diese TEKs auch auf einem Server gesammelt und von da aus verteilt werden.
Dieser Server wird in diesem Dokument als der Potenziell Kranke Server (PKServer) bezeichnet.
Wird in der CWA ein erhöhtes Risiko mit einem Risikowert $\geq$ 13 erkannt und durch einen roten Bildschirm angezeigt, so sollen die TEKs der letzten 14 Tage, verbunden mit dem zum roten Bildschirm führenden Risikowert, auf den PKServer hochgeladen werden.
Dies kann entweder durch das freiwillige bestätigen einer Option oder als Integration in die freiwillige Datenspende implementiert werden.
Aufgrund der Inkubationszeit werden TEKs auf dem PKServer, die älter als 14 Tage sind, gelöscht.

Die CWA lädt regelmäßig alle neuen TEKs mit den zugehörigen Risikowerten vom PKServer herunter und leitet daraus die RPIs ab.
Die Verwendung der RPIs und Risikowerte in der Risikoberechnung wird im Abschnitt (CWA-Likely-Positive) beschrieben.
Das Ableiten der RPIs aus den heruntergeladenen TEKs funktioniert genau so, wie das Ableiten der RPIs aus den Diagnosis Keys auf dem CWA Server.
Der mit einem TEK verbundene Risikowert wird dazu genutzt die Wahrscheinlichkeit zu ermitteln, ob der Teilende des TEKs mit Corona infiziert ist.

Anders als beim Teilen von TEKs als Diagnosis Keys mit dem CWA Server im Falle eines positiven Testergebnisses wird kein eindeutiger, autorisierter Schlüssel benötigt, um TEKs auf den PKServer hochladen zu können.
Dies wirft verschiedene Probleme hervor, die zwar im Folgenden genauer erläutert, jedoch nicht gelöst, werden.

Die vorgeschlagene Implementierung dieses Features sieht vor, dass ein TEK nicht mehrmals innerhalb 14 Tage auf den PKServer hochgeladen wird.
Dies führt dazu, dass nur der erste Risikowert, der zu einem roten Bildschirm führt, zusammen mit den TEKs auf den PKServer hochgeladen werden kann.
Da die Risikowerte in direkter Verbindung mit der Risikoberechnung stehen, kann dies zu einer ungenaueren Risikoberechnung führen.
Sollten bereits auf dem PKServer existierende TEKs erneut hochgeladen werden, so müssen diese abgelehnt werden, weil nicht bestätigt wird, ob das erneute Hochladen von der gleichen Person getätigt wird.
Würden diese nicht abgelehnt werden, wäre es ein leichtes für einen Angreifer die Risikowerte der TEKs zu verändern, da der Lesezugriff auf die TEKs und Risikowerte auf dem PKServer nicht eingeschränkt ist.

Es ist des weiteren möglich falsche TEKs auf den PKServer hochzuladen, da weder die Existenz eines TEKs, noch das Vorhanden sein eines roten Bildschirms, überprüft wird.
Dies ermöglicht Denial of Service Angriffe mit verhältnismäßig geringem Aufwand und Ressourcen.\\

\subsubsection{Feature TRL Update}
In diesem Feature wird das volle Potenzial des TRLs genutzt. Dafür wird festgelegt, dass das TRL den ursprünglichen Standardwert von vier nicht unterschreiten darf, da mit zu geringer Sicherheit gesagt werden kann, dass eine Infizierte Person nicht mehr infektiös ist. Die Implementierung dafür existiert bereits in der CWA, jedoch existieren die benötigten Daten noch nicht, wie im Abschnitt (TRL) genauer beschrieben. Durch Umfragen, die beim Eingeben eines positiven Corona Tests aktiviert werden, sollen die Infizierten ihren Infektionszeitpunkt und den Symptom-Start-Zeitpunkt angeben können. Es wird angenommen, dass beim Eingeben des Testergebnisses der Zeitpunkt des Testens, der Zeitpunkt des Ergebnis Erhaltens und der Zeitpunkt des Eingebens in der App gespeichert werden. Eine Umfrage könnte folgende Fragen beinhalten:

\begin{itemize}
	\item Wann haben Sie in den letzten 14 Tagen Symptome verspürt?
	\item Wann wurden Sie vermutlich infiziert?
\end{itemize}

Im Falle, dass es nicht möglich ist, die Zeitpunkte des Testens und  des Rückmeldens im Testergebnis zu codieren, kann man folgende Fragen anhängen:

\begin{itemize}
	\item Wann haben Sie sich testen lassen?
	\item Wann haben Sie das positive Ergebnis bekommen?
\end{itemize} 
	
Mit diesen Daten kann dann der Zeitpunkt des Infektiös seins kalkuliert werden und das Übertragungspotenzial mit dem TRL für jeden Diagnosis Key abgeschätzt werden, damit Nutzer besser gewarnt werden können.\\

\subsubsection{Risikoberechnung}


In diesem Abschnitt werden die drei Summanden R\textsubscript{CWA-Positive}, R\textsubscript{CWA-Likely-Positive} und R\textsubscript{CWA-Likely-Negative} beschrieben, aus denen sich der Risikowert zusammensetzt.

\begin{figure*}[h]
	\centering
	\includegraphics[width=0.9\textwidth]{"RiskCalculation"}
	\caption{erweiterte Risikoberechnung}
	\label{ErweiterteRisikoberechnung_Abbildung}
\end{figure*}

In Abschnitt (Probleme und Lösungsansätze) werden Probleme und Lösungsansätze der aktuellen Risikoberechnung genauer erläutert.
Danach werden in Abschnitt (sfndao- ABC) die drei Summanden näher beschrieben, wobei die angepasste Grafik in Abbildung \ref{ErweiterteRisikoberechnung_Abbildung} dies veranschaulicht.

\paragraph{Probleme und Lösungsansätze}

Das Hauptproblem der Risikoberechnung der aktuellen CWA ist, dass diese zu wenige relevante Faktoren mit einberechnet, denn nur CWA-Positive Personen werden mit einbezogen.
Alle anderen Personengruppen werden ignoriert. Der Risikowert ist somit geringer als er eigentlich sein sollte. Im Folgenden werden weitere Faktoren beschrieben, die die erweiterte Risikoberechnung beachtet:

Mit der Dunkelziffer lässt sich die Anzahl an tatsächlich Infizierten abschätzen und sie sorgt dafür, dass CWA-Likely-Negative Personen, unter einbezug der positiven Not-CWA-User, 
zu einem Risiko führen können, welches in der neuen Risikoberechnung nicht ignoriert wird.
Die Downloadzahl wird genutzt, um den Risikowert nach oben zu korrigieren, da die Not-CWA-User nicht als Begegnung aufgezeichnet werden.
Begegnungen mit CWA-Likely-Positive Personen, die eine höhere Wahrscheinlichkeit haben infiziert zu sein, da sie direkten Kontakt zu Infizierten hatten, 
werden in der neuen Berechnung stärker als die CWA-Likely-Negative gewichtet.
Außerdem ist die Risikoberechnungsanzeige nicht transparent genug, zwar ist die Anzahl an Begegnungen mit CWA-Positive Personen sichtbar, 
jedoch sieht man weder den Grad des Übertragungspotenzials, noch wie der Risikowert zustande gekommen ist.

Dafür wurde die Berechnung des Risikowertes in 3 Summanden aufgeteilt, damit die unterschiedlichen Personengruppen und Faktoren mit einbezogen werden können.

\centerline{\text{ }}
\centerline{\text{Neuer Risikowert := }}
\centerline{\text{R\textsubscript{CWA-Positive} + R\textsubscript{CWA-Likely-Positive} + R\textsubscript{CWA-Likely-Negative}}}
\centerline{\text{ }}

Der erste Summand ist R\textsubscript{CWA-Positive} und berechnet das Risiko, indem nur Begegnungen mit CWA-Positive Personen einbezogen werden. 
Dabei wird jede Begegnung gewichtet, je nach dem wie hoch das Übertragungspotenzial der Begegnung ist. 
Der zweite Summand R\textsubscript{CWA-Likely-Positive} schätzt das Risiko von Begegnungen mit Likely-Positive Personen, sprich roten Bildschirmen bzw. 
CWAs mit einem Risikowert von über 13, aber keinem positiven Test Ergebnis, ab.
Außerdem gibt es noch den dritten Summanden R\textsubscript{CWA-Likely-Negative}, der die Begegnungen mit Personen aus der Personengruppe CWA-Likely-Negativ mit einberechnet,
wobei auch die Personengruppe der Not-CWA-User mit betrachtet wird.
Der Neue Risikowert wird anschließend in eine Bildschirmfarbe umgewandelt, die als höhe des Infektionsrisikos interpretiert werden kann.

Zudem sollte es eine geringere Abstufung der Risikostufen geben, denn nur grün und rot wecken den Eindruck, dass man bei einem grünen nicht gefährdet wäre. 
Ab einem gewissen Risiko sollte auch ein gelber sowie ein orangener Bildschirm angezeigt werden, damit die Nutzer ihr Verhalten besser regulieren können. \\
Dabei werden anstatt der ursprünglichen zwei Farben nun vier Farben Rot, Orange, Gelb und Grün verwendet.
\\

$0\le Risikowert < 5 \rightarrow \text{grüner Bildschirm}$

$5\le Risikowert < 9 \rightarrow \text{gelber Bildschirm}$

$9\le Risikowert < 13 \rightarrow \text{orangener Bildschirm}$

$13 \le Risikowert  \rightarrow \text{roter Bildschirm}$
\\
Ein weiteres Problem der aktuellen Risikoberechnung ist die geringe Transparenz des Zustandekommens des Risikowertes.
Ein transparenter Risikowert macht es für Nutzer leichter nachvollziehbar, wie sie sich bei einem erhöhten Risiko Verhalten sollten und fördert die aktive Nutzerzahl.
Damit die Risikoberechnung transparenter für die Nutzer wird, werden die Anteile der einzelnen Summanden zum farbigen Bildschirm angegeben, 
sowie die Begegnungsdauer und Begegnungsnähe zu CWA-Positive Nutzern.

\paragraph{R\textsubscript{CWA-Positive}}

Das Risiko R\textsubscript{CWA-Positive} entspricht der aktuellen Risikoberechnung der CWA (Stand 26.04.2021). 
In dieses Risiko fließen alle Begegnungen mit der Personengruppe CWA-Positive ein. 
Da CWA-Positive Personen ein positives Testergebnis in der CWA eingetragen haben, ist die Positiv-Begegnungs-Wahrscheinlichkeit hier konstant 1. 
Es wird also nur das Übertragungspotential betrachtet. Dafür wird zunächst zu einer Begegnung die Weighted Exposure Time ausgewertet. 
Die Weighted Exposure Time bestimmt die Gewichtung der Begegnungszeit t anhand der geringsten Dämpfung (attenuation) att  wie folgt: 

\centerline{\text{ }}
\centerline{$\text{WeightedExposureTime} := t \cdot \text{attFactor}$}
\centerline{\text{ }}

Wobei der attFactor durch folgende Intervalle aus der Distanz att bestimmt wird:

\centerline{\text{ }}

\text{$att < 63dB:attFactor=0.8$}

\text{$63dB \le att<73dB:attFactor=1.0$}

\text{$73dB \le att<79dB:attFactor=0.1$}

\text{$79dB \le att:attFactor=0$}
\centerline{\text{ }}

Dabei kann man die Werte 63dB bzw. 73dB, als Signalstärke von einem Handy interpretieren, das 3 Meter bzw 8 Meter weit entfernt ist. 
Dabei schwächen Kleidung, Körper oder andere Gegenstände das Signal ab und sorgen dafür, dass die meisten Begegnungen erst auf kleinerer Distanz aufgezeichnet werden (Quelle cwa-risk-assessment).
Aus der Weighted Exposure Time wird dann die Normalised Exposure Time t­n mithilfe des Transmission Risk Levels TRL ermittelt und des daraus resultierenden Transmission Risk Values TRV. 
Das Transmission Risk Level wird in Abschnitt TRL genauer beschrieben und bezieht die Wahrscheinlichkeit mit ein, dass eine Person die Krankheit auch wirklich überträgt. 
Der TRV ergibt sich aus diesem nach der folgenden Vorschrift:

\centerline{\text{ }}

\text{$TRL=I\rightarrow TRV=0.0$}

\text{$TRL=II\rightarrow TRV=0.0$}

\text{$TRL=III\rightarrow TRV=0.6$}

\text{$TRL=IV\rightarrow TRV=0.8$}

\text{$TRL=V\rightarrow TRV=1.0$}

\text{$TRL=VI\rightarrow TRV=1.2$}

\text{$TRL=VII\rightarrow TRV=1.4$}

\text{$TRL=VIII\rightarrow TRV=1.6$}
\centerline{\text{ }}

Mit dem TRV lässt sich dann t\textsubscript{n} bestimmen:

\centerline{\text{ }}
\centerline{$t_n := \text{Weighted Exposure Time}\cdot TRV$}
\centerline{\text{ }}

Zuletzt wird aus den Normalised Exposure Times der einzelnen Begegnungen die Summe gebildet und man erhält damit das Risiko R\textsubscript{CWA-Positive}:

\centerline{\text{ }}
\centerline{ \text{R\textsubscript{CWA-Positive}} $ := \sum t_n$}
\centerline{\text{ }}
\paragraph{R\textsubscript{CWA-Likely-Positive}}
Das Risiko R\textsubscript{CWA-Likely-Positive} ergibt sich aus den Begegnungen mit CWA-Likely-Positive Personen, 
die eine erhöhte Wahrscheinlichkeit aufweisen, infiziert zu sein und deswegen in der Berechnung gesondert behandelt werden. 
Hierbei werden Begegnungen mit Gruppenangehörigen aus dem Familienfeature (Abschnitt Familienfeature) nicht mit einbezogen, 
da diese sonst aufgrund der potenziell langen Begegnungsdauer und 
kurzen Distanz ein hohes R\textsubscript{CWA-Likely-Positive} Risiko hervorrufen könnten und man im Regelfall diesen Personen ohnehin sofort mitteilt, wenn man ansteckend sein könnte. 
Zudem werden Begegnungen mit Diagnosis Keys nicht mit einberechnet, da diese bereits im Summand R\textsubscript{CWA-Positive} betrachtet werden. Dies geschieht, 
indem bei der Risikoberechnung die Begegnungs RPIs mit den Diagnosis Keys auf dem CWA Server abgeglichen werden und nur die RPIs mit einbezogen werden, 
die keinen passenden Diagnosis Keys auf dem CWA Server liegen haben.
Zuerst wird, wie beim Summand R\textsubscript{CWA-Positive}, das Übertragungspotenzial einer Begegnung bestimmt, indem die Weighted Exposure Time ausgewertet wird. 
Aus der Weighted Exposure Time wird danach allerdings nicht die Normalised Exposure Time bestimmt, wie dies beim Summand R\textsubscript{CWA-Positive} der Fall ist, 
da dafür ein positives Testergebnis vorausgesetzt wird, welches CWA-Likely-Positive Personen nicht vorweisen können.
Anstelle davon wird als nächstes aus der Weighted Exposure Time jeder Begegnung die Standardized Exposure Time $t_s$ ermittelt. 
Da die Begegnung der CWA-Likely-Positive Personengruppe angehört, ist es nicht sicher, dass diese auch Corona hat. 
Aus diesem Grund wird die Positiv-Begegnungs-Wahrscheinlichkeit benötigt, welche die Wahrscheinlichkeit angibt, dass die Begegnung infiziert ist. 
Um die Positiv-Begegnungs-Wahrscheinlichkeit zu erhalten wird jedoch erst eine Funktion f benötigt, die diese Wahrscheinlichkeit angibt. 
Die Funktion $f$ nimmt als Input den Risikowert der Begegnung und gibt die Wahrscheinlichkeit zurück, dass diese infiziert ist. 
Der Risikowert wurde zuvor mit dem TEK zusammen vom PKServer (Abschnitt Feature Potenziell Kranke) heruntergeladen. Um f zu bestimmen wird folgendes Verfahren angewandt:
Zunächst werden Risikowert-Intervalle gebildet, da Risikowerte nicht nur natürliche Zahlen als Werte annehmen können. 
Anhand der gesammelten Daten (Abschnitt Feature Potenziell Kranke) wird für jedes Risikowert-Intervall $x_i mit i \in N$ und einer Intervallgröße von 1 die Wahrscheinlichkeit bestimmt, 
dass eine Begegnung, mit einem Risikowert innerhalb $x_i$, infiziert ist. Dabei gilt für jedes Intervall:

\centerline{\text{ }}
\centerline{$x_i := [13+i, 14+i[$}
\centerline{\text{ }}

Es sind somit nur Risikowerte über 13 relevant, da die Risikowerte auf dem PKServer alle $\le 13$ sind. 
Damit die Wahrscheinlicht bestimmt werden kann, dass eine Person mit einem Risikowert innerhalb eines bestimmten Intervalls $x_i$ infiziert ist, 
wird die Anzahl an Personen benötigt, die einen Risikowert innerhalb des Intervalls $x_i$ hat und die Anzahl an Personen, 
die sowohl einen Risikowert innerhalb des Intervalls $x_i$ hat, als auch später an Corona erkrankt ist. 
Dafür werden die TEKs auf dem PKServer und die Diagnosis Keys auf dem CWA Server verwendet. Aus den TEKs auf dem PKServer kann die Anzahl an Personen bestimmt werden,
die einen Risikowert innerhalb des Intervalls $x_i$ hat.
Werden zusätzlich zu den TEKs auf dem PKServer die Diagnosis Keys auf dem CWA Server dazu genommen, 
kann damit die Anzahl an Personen bestimmt werden, die sowohl einen Risikowert innerhalb des Intervalls $x_i$ hat, als auch später an Corona erkrankt ist, 
indem die TEKs auf dem PKServer und den zugehörigen Risikowerten mit den Diagnosis Keys auf dem CWA Server abgeglichen werden. 
Daraus lässt sich für jedes Intervall $x_i$ die Wahrscheinlichkeit, infiziert zu sein, wie folgt bestimmen:

\centerline{\text{ }}
\centerline{$\frac{\text{Anzahl TEKs auf dem PKServer und CWA Server mit Risikowert} \in x_i}{\text{Anzahl TEKs auf dem PKServer mit Risikowert} \in x_i}$}
\centerline{\text{ }}

Zur Anzahl TEKs auf dem PKServer und CWA Server mit Risikowert $\in x_i$ werden nur die TEKs gezählt, 
die auf PKServer und den CWA Server hochgeladen wurden und einen Risikowert innerhalb $x_i$ aufweisen. 

Zur Anzahl TEKs auf dem PKServer mit Risikowert  $\in x_i$  werden alle TEKs auf dem PKServer gezählt, die einen Risikowert innerhalb $x_i$ aufweisen.
Da in der Regel zwischen dem Hochladen eines TEKs auf den TEK Server und dem Hochladen des gleichen TEKs als Diagnosis Key auf den CWA Server einige Tage vergehen und 

Keys auf beiden Servern nur 14 Tage lang gespeichert werden, muss für Bestimmung der Funktion $f$ die richtige Datenmenge genommen werden. 
Die Anzahl TEKs auf dem PKServer mit Risikowert  $\in x_i$ werden ab dem Startzeitpunkt $s_0$, der 14 Tage in der Vergangenheit liegt, bis zu einem Endzeitpunkt $s_1$, 
der 7 Tage nach $s_0$ liegt, gezählt. Die Anzahl TEKs auf dem PKServer und
CWA Server mit Risikowert $\in x_i$ werden ab dem gleichen Startzeitpunkt $s_0$ bis zum Tag der Bestimmung der Funktion $f$ gezählt.
Dabei muss allerdings der TEK im Zeitraum $s_0$ bis $s_1$ auf den PKServer hochgeladen worden sein. 
Verwendet man zur Bestimmung der Funktion f zum Beispiel die TEKs der letzten 7 Tage auf dem PKServer, 
so werden viele dieser TEKs nicht als Diagnosis Keys auf den CWA Server hochgeladen, 
obwohl diese infiziert sind, da noch kein positives Testergebnis aufgrund der geringen zeitlichen Distanz vorliegt. 
Die davon betroffenen TEKs würden die Funktion $f$ negativ beeinflussen.

Nachdem für jedes Intervall $x_i$ die Wahrscheinlichkeit, infiziert zu sein, bestimmt wurde, wird durch ein Regressionsverfahren die Funktion $f$ bestimmt, 
die jedem Risikowert die Positiv-Begegnungs-Wahrscheinlichkeit zuordnet. Hierbei ist besonders zu beachten, 
dass die Funktion $f$ etwa alle 7 Tage mit aktuelleren Daten neu berechnet und für die CWA aktualisiert werden muss. 
Beim Herunterladen von Keys (vom CWA Server oder PKServer) innerhalb der CWA muss dann gleichzeitig die aktualisierte Funktion $f$ heruntergeladen werden. 
Mithilfe der Funktion $f$ kann dann $t_s$ bestimmt werden: 

\centerline{\text{ }}
\centerline{$t_s := \text{Weighted Exposure Time} \cdot f(ContactRS)$}
\centerline{\text{ }}

Hierbei ist ContactRS der Risikowert des Begegneten.
Als nächstes wird aus den Standardized Exposure Times der einzelnen Begegnungen mit CWA-Likely-Positive Personen die Summe gebildet und das Ergebnis auf 13 begrenzt, 
um daraus das Risiko RCWA-Likely-Positive zu erhalten:

\centerline{\text{ }}
\centerline{$R_{\text{CWA-Likely-Positive}}$ := min($\sum t_s$, 13)}
\centerline{\text{ }}

Die Begrenzung des Summanden R\textsubscript{CWA-Likely-Positive} auf 13 hat zur Folge, 
dass CWA-Likely-Positive Personen nur einen Risikowert $\le 13$ bei anderen Personen hervorrufen können. 
Dies verhindert, dass sich eine Kaskade von CWA-Likely-Positive Personen bilden kann. Nimmt man diese Begrenzung nicht vor, 
ist es möglich, dass eine einzelne CWA-Likely-Positive Person bei einer großen Anzahl an Personen einen roten Bildschirm hervorrufen kann, 
die wiederum bei weiteren Personen rote Bildschirme hervorrufen können und so weiter. 
Durch die Begrenzung kann zwar eine CWA-Likely-Positive Person mit einem hohen Risikowert immer noch rote Bildschirme bei anderen hervorrufen, 
jedoch können die Personen, die dadurch einen roten Bildschirm erhalten haben, diesen nur durch sehr lange Begegnungen, z.B. einen ganzen Tag, bei weiteren Personen hervorrufen.\\
\paragraph{R\textsubscript{CWA-Likely-Negative}}
Das Risiko R\textsubscript{CWA-Likely-Negative} (Abbildung 3, grüner Kasten) berechnet alle Begegnungen mit CWA-Likely-Negative und Not-CWA-User Personen mit ein.
Denn aktuell haben nur ein Bruchteil der Infizierten ein positives Test Ergebnis in die App eingetragen. 
Somit muss das Risiko mit den folgenden Hauptpunkten angepasst werden:

\begin{itemize}
	\item Dunkelziffer: Mithilfe der Dunkelziffer ist eine genauere Abschätzung der Anzahl der mit Corona infizierten deutschen Bürger möglich. 
	Denn Covid-19 hat die Eigenschaft nicht immer symptomatisch aufzutreten, es gibt somit viele Infizierte, 
	die nicht aufgezeichnet wurden. Dazu kommt noch, dass nicht alle Personen ihr Testergebnis in die App eintragen.
	Die Dunkelziffer ist somit  ein Faktor, der beschreibt wie viele Personen mehr erkrankt sind, 
	als positiv getestet wurden. Mithilfe der Dunkelziffer kann somit die größe der Gruppe Hidden-Positive geschätzt werden.
	Die Dunkelziffer wurde erstmals durch eine Studie in Kupferzell \cite{Dunkelziffer} nachgewiesen und bestimmt. 
	Da die Datenerhebung dieser Studie vom 20. Mai bis 9. Juni 2020 durchgeführt wurde und sich die Testmöglichkeiten seitdem stark entwickelt haben, 
	ist die damals bestimmte Dunkelziffer von 6 möglicherweise ungenau. 
	\item  AppUsage: Eine Begegnung mit anderen Personen kann nur dann aufgezeichnet werden, wenn diese die Corona Warn App auch installiert haben. 
	Begegnungen mit der Personengruppen Not-CWA-User werden somit nicht aufgezeichnet, sowie auch der Anteil an Infizierten aus dieser Gruppe. 
	Da die derzeitige Nutzung deutlich unter 50\% liegt, muss dieser Faktor in die Risikoberechnung mit einfließen, 
	indem die Anzahl der aufgezeichneten Begegnungen nach oben korrigiert wird. 
	Dazu wird die Downloadzahl der CWA (DownloadCount) und die Einwohnerzahl Deutschlands (Population) genutzt. 
	Aktuell würde die AppNutzung somit ca. $frac{1}{3}$ betragen.\cite{Einwohnerzahl} \cite{Downloads}
\end{itemize}

\centerline{\text{ }}
\centerline{$AppUsage := \frac{DownloadCount}{Population}$}
\centerline{\text{ }}

Wie auch beim Risiko R\textsubscript{CWA-Positive} werden beim Risiko R\textsubscript{CWA-Likely-Negative} die Begegnungen der letzten 14 Tage beachtet.
Denn obwohl sich jemand nur mit Personen aus den Gruppen CWA-Likely-Negative und Not-CWA-User getroffen hat
ist es bei einer hohen Anzahl an Begegnungen wahrscheinlicher infiziert zu werden, als bei einer niedrigen, denn in der Not-CWA-User Gruppe gibt es Personen, die infektiös sind. 
Deswegen wird hierfür die Personengruppe Likely-Negative benötigt. 
Hierbei werden, wie beim Risiko R\textsubscript{CWA-Likely-Positive}, 
Begegnungen mit Gruppenangehörigen aus dem Familienfeature (Abschnitt Familienfeature) und Begegnungen mit Diagnosis Keys, 
sowie zusätzlich Begegnungen mit Likely-Positive (Abschnitt Potenziell Kranke)  nicht mit einberechnet, 
da diese bereits in die Risiken R\textsubscript{CWA-Positive} und R\textsubscript{CWA-Likely-Positive} mit einfließen.

Für dieses Risiko wird hauptsächlich mit einer Positiv-Begegnungs-Wahrscheinlichkeit gerechnet,
da man mit dieser die Gruppen Not-CWA-Positive und Hidden-Positive mit einfließen lassen kann.
Dabei wird das Übertragungspotenzial nicht wie bei den anderen Risiken R\textsubscript{CWA-Positive} und R\textsubscript{CWA-Likely-Positive}  berechnet, 
da die Positiv-Begegnungs-Wahrscheinlichkeit einer CWA-Likely-Negative Person zu gering ist. 
Die Positiv-Begegnungs-Wahrscheinlichkeit mit Begegnungen aus der CWA-Likely-Negative Gruppe wird dafür genutzt, 
um den Einfluss der Gruppen Not-CWA-Positive und Hidden-Positive zu berechnen und wird unter anderem durch die Dunkelziffer und die AppNutzung bestimmt.

Um das Übertragungspotenzial trotzdem miteinfließen zu lassen werden alle Begegnungen gezählt bei denen die Signalstärke att Aunter 73 Dezibel ist,
der Abstand somit kleiner als 8 Meter ist und die Begegnungsdauer zwischen einem LowerBound und einem UpperBound ist. 

Damit nicht jede noch so kurze Begegnungen aufgezeichnet wird,
wird eine LowerBound auf 2 Minuten gesetzt. Bei einer einzigen langen Begegnung werden viele Begegnungen mit einer Begegnungsdauer von ca.
15 Minuten aufgezeichnet, da der RPI alle 15 Minuten gewechselt wird. Um dies zu verhindern wird noch einen UpperBound benötigt.
Dafür soll jede Signalstärke mit einberechnet werden, damit kurze Distanzierungen keinen großen Einfluss haben.
Für diesen Wert wird 14 Minuten geschätzt, damit Verbindungsfehler keinen großen Einfluss auf die Begegnungsanzahl haben.
Jede Begegnung mit Begegnungsdauer zwischen Lower- und UpperBound wird somit gezählt.

\centerline{\text{ }}
\text{$if(att<73dB) \le lowerBound$\{}

\text{\&\&$t(att > 0dB) < upperBound)$}

\text{$Contacts++$}

\text{\}}

\centerline{\text{ }}

Um die Wahrscheinlichkeit zu berechnen, bei welcher Begegnungsanzahl man sich ansteckt, wird zuerst die aktuelle Anzahl infizierter ausgerechnet, 
um damit die Positiv-Begegnungs-Wahrscheinlichkeit auszurechnen. 
Die aktuelle Anzahl positiv getesteter wird mit der Dunkelziffer multipliziert, 
um die Anzahl aktuell Infizierter zu berechnen. Damit wird currentCWAPositive subtrahiert, da das Risiko bereits in R\textsubscript{CWA-Positive} mit einfließt.  

\centerline{\text{ }}
\centerline{$\text{PersonIsInfected} :=$}
\centerline{$\frac{((\text{currentPositiveTested} \cdot \text{Dunkelziffer}) - \text{currentCWAPositive}}{\text{Population}}$}
\centerline{\text{ }}

R\textsubscript{CWA-Likely-Negative}setzt sich aus dieser Positiv-Begegnungs-Wahrscheinlichkeit zusammen und wird auf 13 beschränkt, 
da die Person gewarnt werden soll, allerdings nicht als hoch Infektiös gilt. 
Dabei wird die Anzahl Contacts mit der AppUsage nach oben korrigiert, um die Anzahl wirklicher Begegnungen CorrectContacts besser abzuschätzen. 
Aktuell beträgt die AppUsage ca. $1/3$ (Quelle), somit werden die Contacts um das Dreifache erhöht. 

\centerline{\text{ }}
\centerline{$\text{CorrectContacts} := \frac{Contacts}{AppUsage}$}
\centerline{\text{ }}

Da mit einem Beschränkten Wachstum gerechnet wird und somit die Positive-Begegnungs-Wahrscheinlichkeit subtrahiert werden muss, 
muss mit der Gegenwahrscheinlichkeit gerechnet werden. Denn umso höher die  Infektionswahrscheinlichkeit, 
desto geringer ist die Gegenwahrscheinlichkeit und desto näher ist R\textsubscript{CWA-Likely-Negative} an der Schranke 13. 
Um die Wahrscheinlichkeit zu berechnen infiziert worden zu sein, wird die Wahrscheinlichkeit, dass eine Person nicht infektiös ist, 
mit der Anzahl an getroffenen Personen CorrectContacts exponiert.

\centerline{\text{ }}
\centerline{\text{R\textsubscript{CWA-Likely-Negative}$=$}}
\centerline{\text{$(1 - ( 1-\text{PersonIsInfected} )\text{CorrectContacts}) \cdot 13$}}
\centerline{\text{ }}

Somit ist es möglich ein erhöhtes Risiko in der App angezeigt zu bekommen, obwohl man keinen Diagnosis Key getroffen hat. 
\\
\subsubsection{Bewertung}

Mithilfe der erweiterten Risikoberechnung kann das Infektionsrisiko besser abgeschätzt werden,
da mehr Faktoren betrachtet werden. Weiterhin gilt jedoch: Je mehr Personen die CWA nutzen, desto effektiver ist diese.
Dabei wird auf die Privatsphäre der Nutzer geachtet, indem freiwillige, anonymisierte Datensammlungen verwendet werden.
Da die Datenspende freiwillig ist, benötigt man eine hohe Beteiligung an Nutzern, die Daten teilen.
Zudem ist die Berechnung des Risikos R\textsubscript{CWA-Positive}, welches aus der aktuellen CWA übernommen wurde,
in der erweiterten Risikoberechnung mit eingebunden, wodurch nur ein höherer Risikowert zusammen kommen kann
und somit Personen höchstens früher gewarnt werden. 


Die erweiterte Risikoberechnung bietet einen hohen Grad an Transparenz bezüglich des Zustandekommens und der Bedeutung des Risikowertes. 
Außerdem werden durch die feinere Granulierung der Risikoberechnung und Risikoanzeige auch kleinere Risikostufen wahrgenommen. 
Somit werden schneller Veränderungen des Infektionsrisikos dargestellt, wodurch dem in der CWA angezeigten Risiko mehr vertraut werden kann.

Für die Verwendung des Familienfeatures und des PKServers müssen Sicherheitsmechanismen gegen Denial of Service Angriffe implementiert werden, 
auf die in diesem Dokument nicht weiter eingegangen wird.
Im Vergleich dazu wird bei den im Abschnitt (Feature TRL Update) vorgestellten Umfragen der Besitz eines eindeutigen Schlüssels vorausgesetzt,
der die Teilnahme an einer Umfrage ermöglicht. Der dafür benötigte Schlüssel ist nur im Falle eines positiven Testergebnisses zu erhalten und sorgt dafür, dass kein Spam möglich ist.

Obwohl es 2 weitere Server gibt (PKServer und Familienfeature Server), auf denen Daten zentral gespeichert werden, bleiben die Risikoberechnung und Aufzeichnungen der RPIs dezentral. 
Auf den zusätzlichen Servern werden ausgewählte TEKs gespeichert, wie auf dem CWA Server. 
Zudem werden die Daten auf dem PKServer und dem CWA Server dazu genutzt, die in Abschnitt (R-CWA-LIKELY-POSITIVE) beschriebene Funktion f zentral zu bestimmen.

Durch die Installation von 2 weiteren Server, werden zusätzliche Serverkosten anfallen. Da diese Server einen ähnlichen Aufbau wie der CWA Server haben, 
werden die Kosten für diese in einen ähnlichen Umfang haben. 
\\
\subsection{Density-Map}
Eine Density-Map zeigt an, wo sich bestimmte Datenpunkte auf einer Karte häufen. 
In der Regel findet sie Anwendung bei Bevölkerungsaufzeichnungen. 
Im Sinne der erweiterten Corona Warn App möchten wir dieses Feature benutzen, um in nahezu Echtzeit die Besucherzahl an öffentlichen Orten oder Geschäften aufzuzeichnen und grafisch auf einer Karte darstellen. 
Damit kann eine Nutzer stark besuchte Gegenden identifizieren und meiden. 
Ein ähnliches Feature wird bereits vom Unternehmen Google angeboten. 
Wer in dessen Suchmaschine nach Geschäften in der Umgebung sucht kann neben den Öffnungszeiten meist auch die live Stoßzeiten und durchschnittliche Aufenthaltsdauer ansehen. 
Google verwendet hierfür jedoch personalisierte Daten seiner Nutzer, wie z.B. den Standort um diese Aufzeichnung vorzunehmen. 
Um die Sicherheit und Privatsphäre der Nutzer zu schützen schlagen wir deshalb eine alternative Lösung mithilfe des BLE-Features der Corona Warn App vor. \\
\subsubsection{Lösungsansatz} 
\text{ }\\
Das zentrale Problem, mit dem wir uns beim Erstellen einer Density Map beschäftigen ist, wie die anonymen Daten gebündelt und auf einen zentralen Server geladen werden können, ohne die Sicherheits-/Privacy Aspekte signifikant einzuschränken/verletzen. 
Ein naiver Ansatz wäre beispielsweise, dass jedes Smartphone seine gesammelte Begegnungsanzahl auf einen zentralen Server hochlädt und dieser diese Daten auswertet. 
Jedoch müssten dafür neben der Standorte der einzelnen Smatphones auch die RPI's aller aufgezeichneten Geräte mit hochgeladen werden, um etwa eine doppelte Zählung bestimmter Endgeräte zu verhindern. 
Eine derartige Sammlung aller Kontaktdaten (auch wenn es sich bei diesen Daten um anonyme RPI's handelt) auf einem zentralen Server würde jedoch der grundsätzlichen Philosophie der Corona Warn App widersprechen.
Diese verfolgte in ihrer Kontaktverfolgung einen möglichst dezentralen Ansatz, um die Privatsphäre der Nutzer zu garantieren.
Dementsprechend suchen wir ebebfalls gezielt nach Verfahren, welche weitmöglichst eine dezentrale Lösung verfolgen. 
Hierfür haben wir uns für die Umsetzung 2 Verfahren überlegt, mit welchen wir das Problem angehen: Eine infrastrukuturbasierte Lösung mittels \textbf{zentraler Masten}, welche in Echtzeit die BLE-Geräte in ihrem festen Empfangsgebiet aufzeichnen, sowie eine infrastrukturlose Lösung \textbf{ohne Masten}. \\ \\
\textit{Infrastrukturbasierte Lösung mit Masten: }\\
Die Idee zentraler Masten besteht darin, dass diese an stark besuchten Orten/Plätzen, wie z.B. Supermärkten oder Haltestellen, ihre Umgebung scannen und für ein gewisses Zeitintervall alle erfassten Smartphones aufzeichnen. 
Nach Ablauf des Zeitintervalls wird die Anzahl der Begegnungen auf einen zentralen Server hochgeladen.
Anders als beim naiven Ansatz, bei dem der zentrale Server die Daten aller Smartphones erfassen würde, ist hier die Aufzeichung auf tausende bis zehntausende einzelne Masten verteilt. 
Dadurch gäbe es keinen \glqq Single point of failure", bei dem beispielsweise die Standorte aller Smartphones preisgegeben werden würde. 
Da außerdem die Daten dezentral bei den Masten ausgewertet werden und dann erst auf einen Server hochgeladen, reduziert dies den Datenstrom und die Last des znetralen Servers erheblich. 

Die Begegnungsaufzeichnung erfolgt analog zur Aufzeichnung mobiler Endgeräte untereinander per BLE. 
Somit sind in diesem Aspekt Sicherheit und Privatsphäre gleichermaßen gesichert, wie bei der bisherigen Corona Warn App. 
Außerdem sollen erst ab einem gewissen Threshhold von Aufzeichnungen (z.B. 5 Personen) die Daten auf den Server hochgeladen werden, sowie lediglich die Anzahl der Begegnungen und nicht die einzelnen RPI‘s der Geräte. 
Diese Aggregation der Daten stellt sicher, dass keine einzelnen RPI's oder Personen über die Masten hinweg nachverfolgt und damit identifiziert werden können. 
Abb. \ref{mast_func} zeigt den groben Ablauf eines zentralen Masten. \\ 
\begin{figure}[h]
	\centering
	\includegraphics[width=0.5\textwidth]{"Mast_functionality"}
	\caption{Funktionsweise eines zentralen Masten}
	\label{mast_func}
\end{figure}
\\
Wie bereits erwähnt scannt ein Mast seine Umgebung nach Endgeräten und zeichnet alle Begegnungen innerhalb eines Zeitintervalls von etwa 15 Minuten (Lebensdauer eines RPI's) auf. 
Die Aufzeichnung erfolgt analog zur bisherigen Begegnungsaufzeichnung der Corona Warn App über BLE. 
Es werden demnach nur die RPI's ausgetauscht und keinerlei sensible Nutzerdaten. 
Nach der Aufzeichnung überprüft der Mast, ob es sich bei der Begegnung nicht um ein bereits bekanntes Gerät handelt und erhöht dementsprechend den Counter für die Begegnungsanzahl um eins. 
Nach Ablauf des 15 Minuten Intervalls lädt der Mast die gesammelte Anzahl an Begegnungen auf einen zentralen Server hoch, setzt seine Aufzeichnungsdaten zurück und  fängt wieder von vorne mit seiner Begegnungsaufzeichnung an. 

Da ein Mast unter Umständen einen größeren Bereich abdecken muss als ein normaler Nutzer, sollte das Signal (z.B. mit Repeatern) unter Umständen verstärkt werden, falls dies mit BLE möglich.
Andernfalls müsste ein Ort in mehrere Bereiche unterteilt werden, die dann von einzelnen Masten aufgezeichnet werden können. \\
Hier wird auch deutlich, dass sich die Aufzeichnungsradien der einzelnen Masten möglichst nicht überschneiden sollte, da sonst ein Smartphone von mehreren Masten gleichtzeitig aufgenommen wird.
Eine mögliche Lösung für dieses Problem wäre, die BLE Signale der Masten mit bestimmten Tags zu versehen, sodass ein Smartphone erkennt, falls es sich im Radius von mehreren Masten aufhällt und dementsprechend nur einem Masten seine Daten sendet. 

Da die Installation zentraler Masten an jedem belebteren Platz ein aufwändiges und kostspieliges Unterfangen sein kann schlagen wir vor, dass registrierte/zertifizierte mobile Endgeräte ebenfalls als zentraler Mast fungieren können. 
Hierfür müssen jedoch Einschränkungen gelten, sodass die Funktion als zentraler Mast auf beispielsweise einem Smartphone nicht missbraucht wird.
Folgende Regeln und Restriktionen sollten für alle Masten gelten:
\begin{itemize}
	\item Masten können selbständig von Dritten erworben oder im Falle eines Smarphones, welches als zentraler Mast fungieren soll, die App installiert werden. 
	Jedoch muss vor der Nutzung des Features das Gerät von einer offizienllen Behörde autorisiert werden. Damit darf dann etwa ein Restaurantbetreiber, über eine zentrale Behörde registriert, seine Einrichtung mit seinem Smartphone aufzeichnen. 
	\item Zentrale Masten müssen währende ihrer Nutzung dauerhaft ihren Standort aktivieren und können nur im Umfeld ihres registrierten Bereiches (z.B. Ort des Restaurants) als Mast fungieren. 
	Verlässt ein Mast den für ihn zugewiesenen Aufzeichnungsbereich sollt dieser seine Aufzeichnung automatisch stoppen. Dies verhindert, dass das Feature für unerlaubte Zwecke benutzt wird.
	\item Verletzung der Eigenschaften sollte von anderen Nutzern gemeldet werden können. 
	Beispiele für Verletzung der Eigenschaften wäre etwa eine Person, die in einem unerlaubten Gebiet (z.B. Wohngebiet) einen Masten betreibt oder ein Mast, der viele Begegnungen aufzeichnet, obwohl sich keine Person im Umkreis befindet.  
\end{itemize}

Masten, die ihre Begegnungen aufgezeichnet haben, laden diese auf einen zentralen Server hoch. 
Dieser verarbeitet die gesammelten Daten aller Masten und erstellt die Density Map, welche dann vom Nutzer auf seinem Endgerät abgerufen werden kann.
Abb. \ref{density_creation} zeigt einen grafischen Ablauf zu Erstellung der Density Map mittels zentraler Masten. \\
\begin{figure}[h]
	\centering
	\includegraphics[width=0.45\textwidth]{"Density_Map_Overview"}
	\caption{Übersicht: Erstellen der Density Map}
	\label{density_creation}
\end{figure} 
\\
Wie in Kapitel 2 beschrieben, besitzen die RPI's eine Lebensdauer von etwa 15 Minuten. 
Dementsprechend muss beachtet werden, dass Smartphones auch  während des Aufenthaltes im Umkreis eines Masts ihren RPI regelmäßig ändern, wodurch ein Endgerät vom Mast als zwei unterschiedliche Begegnungen erkannt werden kann. 
Der Mast zeichnet dementsprechend zu viele Begegnungen auf und muss seine gemessene Anzahl an Begegnungen nach unten korrigieren, um die Tatsächliche Begegnungsanzahl herauszufinden. 
Wie stark der Wert nach unten korrigiert werden muss hängt hierbei von der Durchschnittlichen Verweildauer eines Smartphones am Masten ab. 
Beträgt die durchschnittliche Verweildauer beispielsweise 15 Minuten (was dem Maximalwert entspricht, da sich nach 15min. der Mast zurücksetzt) wird innerhalb der Aufenthaltsdauer am Masten jedes Endgerät auf jeden Fall einmal seinen RPI ändern und als neue Begegnung vom Mast aufgezeichnet werden. 
In diesem Fall würde der Mast doppelt so viele Begegnungen aufzeichnen als es in der Realität sind.
Ist die durchschnittliche Verweildauer hingegen kurz im Verhältnis zum Änderungsintervall der RPI's (z.B. 3 Minuten) ist die Wahrscheinlichkeit,  dass sich in diesem Zeitintervall der RPI eines Smartphones ändert, sehr klein, was weniger mehrfache Aufzeichnungen zur Folge hat. 
Wie stark ein aufgezeichneter Wert nach unten korrigiert werden sollte wird mit folgender Formel errechnet: 
\begin{equation}
	\text{IDs} = \frac{\text{RPIs}}{(1+p)}
\end{equation}
Dabei ist \glqq RPIs\grqq{} die gemessene Anzahl an RPI's des Masten und \glqq IDs\grqq{}  die voraussichtlich tatsächliche Anzahl an Endgeräten.
Die Wahrscheinlichkeit $p$ einer doppelten Registrierung ist definiert durch:

\centerline{$p = t_V / t_Z$}
wobei $t_Z$ die Dauer eines 15-Minuten Zyklus ist und $t_V$ die durchschnittliche Verweildauer während dieses Zyklus.
Vorausgesetzt wir kennen die \glqq IDs\grqq{} können wir mittels $p$ ausrechnen, wie viele der ID's doppelt gezählt werden mit: IDs$\cdot p$. 
Daraus ergibt sich für die gemessenen \glqq RPIs\grqq{} unter Annahme, dass die ID's bekannt sind, folgende Formel: 
\centerline{$\text{RPIs} = \text{IDs} + \text{IDs} \cdot p $}
Diese Formel nach \glqq IDs\grqq{} umgeformt ergibt die oben genannte Formel zur Berechnung der tatsächlichen Anzahl an Endgeräten.

Eine Übersicht infwiefern der gemessene Wert mithilfe der Formel korrigiert wird in Abb. \ref{cor_factor} aufgezeichnet. 
Auf der x-Achse befindet sich die durchschnittliche Verweildauer der Smartphones in Mastreichweite in Minuten, während die y-Achse die Anzahl an Smatphones (ID's) angibt.
Wir gehen davon aus, dass der Mast in jedem Fall genau 100 ID's aufgezeichnet hat (blauer Graph). Je nach Verweildauer muss der gemessene Wert nun mittels Formel nach unten korrigiert werden (orangener Graph).\\
\begin{figure}[h]
	\centering
	\includegraphics[width=0.45\textwidth]{"Korrekturfaktor"}
	\caption{Korrekturfaktor in Abhängigkeit der durchschnittlichen Verweildauer}
	\label{cor_factor}
\end{figure} \\

Eine Alternative zu dieser Formel wurde zu Beginn der Forschungsarbeit erarbeitet und basiert darauf, den Anteil an doppelten Begegnungen von der gemessenen Anzahl abzuziehen. Der Anteil wird dabei an Hand der Wahrscheinlichkeit einer doppelten Registrierung berechnet, welche wiederum von der durchschnittlichen Verweildauer abhängt.\\
Bei einer Wahrscheinlichkeit \textit{p} einer doppelten Registrierung müssen \textit{p/2} Begegnungen von den RPI's abgezogen werden, um so den realen Wert zu erhalten.\\
Diese alternative Formel ist somit wie folgt definiert:

\centerline{$\text{IDs} = \text{RPIs} * (1-p/2)$}
Beide Formeln werden im Abschnitt \textit{Simulation} evaluiert und miteinander verglichen.\\

\textit{Ohne Masten:}\\
Beim Ansatz ohne zentrale Masten besteht die Idee darin, dass jedes Smartphone als „mobiler Mast“ arbeitet und die Funktion eines Masten überehmen kann.
Auch bei diesem Ansatz möchten wir znächst die Daten dezentral bündeln und aggregieren, um den bereits erwähnten Privatsphäre- und Sicherheitsrisiken des naiven Ansatzes entgegenzuwirken.  

Begegnen sich zwei Smartphones tauschen diese, wie bisher in der aktuellen Corona Warn App, ihre RPI's und Daten untereinander aus. 
Zusätzlich wird nun eines der beiden Geräte als Mast ausgewählt (Abb. \ref{mobile_mast}). 
Dieses zeichnet die Begegnung der beiden untereinander auf und übernimmt alle bisherigen Aufzeichnungen des anderen Gerätes in diesem Zeitintervall.
Die Aufzeichnungsdaten werden also transitiv bis Ende des Intervalls weitergegeben. 
Nach Ablauf des Zeitintervalls lädt das Smartphone, welches als mobiler Mast ausgewählt wurde, die aggregierte Begegnungsanzahl mit seinem aktuellen Standort auf den zentralen Server.

Bei dieser Methode können natürlich nur Smartphones, welche ihren Standort aktiv haben, als mobiler Mast arbeiten. 
Im besten Fall hat jedes Handy diese Funktion aktiv. Smatphones, die ihren Standort nicht aktiv haben, werden können von anderen Smartphones mit aktiven Standort aufgezeichnet, jedoch selbst nie als mobiler Mast ausgewählt werden.

Anders als bei zentralen Masten kann nun an jedem Ort, an dem sich ein Smartphone mit der Corona Warn App befindet eine Aufzeichnung für die Density-Map erfolgen. 
Die Datenerfassung der Density-Map ist somit nicht mehr von vorinstallierten zentralen Masten abhängig.\\
\begin{figure}[h]
	\centering
	\includegraphics[width=0.45\textwidth]{"Mobile_Mast"}
	\caption{Smartphones als Mobile Masten}
	\label{mobile_mast}
\end{figure} \\
Beim mobilen Ansatz können diverse Probleme auftreten, welche gelöst werden müssen: Zum Beispiel können zwei Smartphones A und B ihre gesamten Daten immer an das jeweilige andere Handy überreichen, bevor ihr 15 Minuten Intervall abgelaufen ist. 
Somit bleiben die gesammelten Daten für unendlich lange Zeit lokal auf A oder B gespeichert, ohne dass sie jemals hochgeladen werden. 
Eine Lösung für dieses Problem wäre immer das Smartphone als mobilen Mast auszuwählen, das näher am „Reset“, also am Ablauf des 15-Minuten Intervalls ist.

Ein weiteres Problem bei mobilen Masten besteht darin, dass sie sich innerhalb des Zeitintervalls über längere Strecken bewegen können während sie ihre Begegnungen aufzeichnen.
In diesem Fall entpsrechen Ort der Aufzeichnung eines anderen Gerätes und Ort des Hochladens der Daten nach Ablauf des Zeitintervalles nicht überein und das Bild der Density-Map wird verzerrt. 
Dieses Problem wird teilweise bereits durch die Corona App selbst gelöst, welche Begegnungen erst dann aufzeichnet, sobald sich Geräte über etwas längere Zeit in Kontakt stehen. 
Bei sich schnell bewegenden Smartphones müssten sich demnach anderer Smartphones mit gleicher Geschwindigkeit in dieselbe Richtung bewegen, damit eine Begegnung aufgezeichnet wird. 
Das Verkürzen des Zeitintervalls für Uploads auf beispielsweise 5 Minuten würde diesem Problem ebenfalls entgegenwirken, da in kürzeren Intervallen die Smartphones sich im Schnitt weniger weit bewegen können.

Eine weitere Lösung, die wir diesbezüglich vorschlagen ist ein sogenannter „Discount Faktor“. 
Dieser gewichtet indirekte Verbindungen eines mobilen Masten schwächer in die Auswertung der Density Map. \\
\begin{figure}[h]
	\centering
	\includegraphics[width=0.45\textwidth]{"Discount_Faktor"}
	\caption{Discount Faktor für Mobile Masten}
	\label{disc_factor}
\end{figure} \\
In Abbildung \ref{disc_factor} begegnen sich die beiden Smartphones $A$ und $B$. $A$ hat bereits die RPI's $X, Y, Z$ und $W$ aufgezeichnet und B die RPI's $X, Z, V, U$. 
Nun tauschen die beiden Smartphones ihre RPI's gemäß funktionsweise der Corona Warn App aus. Außerdem wird $A$ als mobiler Mast ausgewählt und übernimmt die Aufzeichnungen von $B$ für die Density-Map.
Die RPI's $V$ und $U$ wurden nur von $B$ aufgezeichnet. Dies bedeutet, dass sie sich nicht im direkten Umkreis von $A$ befinden können. $B$ muss sich vor der Begegnung mit $A$ also an einem anderen Ort befunden haben und sich zu $A$ bewegt haben (aus Sicht von $A$).
Dementpsrechend werden die Begegnungen RPI's $U$ und $V$ mit dem Discount Faktor $\gamma$ versehen.
Der Discount Faktor kann auf verschiedene Arten realisiert werden. Er könnte beispielsweise einen Zahlenwert im Intervall $(0,1)$ annehmen. 
Somit würden indirekte Begegnungen mit dem Faktor multipliziert und nicht voll gewertet werden. Im Beispiel wären die zwei indirekten Begegnungen $U$ und $V$ mit gewähltem $\gamma = 0.6$ versehen und nur als $2 \cdot 0.6 = 1.2$ Begegnungen aufgezeichnet werden.
Eine andere Lösung wäre beim Auswerten der Density-Map indirekten Begegnungen mit anderen Standorten zu versehen. 
Während alle direkten Begegnungen von $A$ im unmittelbaren Umkreis seines Standortes auf der Map angezeigt werden $U$ und $V$(beim hochladen aggregiert in 2 Begegnungen) in einem größeren Radius um $A$ angezeigt.
In diesem Fall wäre der Discount Faktor nur eine Markierung und müsste keinen numerischen Wert erhalten. 

Der Discount Faktor kann ebenfalls transitiv weitergegeben werden. 
Sollten etwa die beiden indirekten Begegnungen $U$ und $V$ an ein weiteres Smartphone $C$ übergeben werden, so würden sie bei $C$ doppelt mit dem Discount Faktor versehen werden.\\
\\
\textit{Hybridlösung:}\\
Um die Effekte beider Ansätze nutzen zu können und die Effizienz weitmöglichst zu erhöhen schlagen wir eine Hybridlösung aus zentralen und mobilen Masten vor. 
Die Umsetzung hierfür ist relativ simpel. Wie oben beschrieben werden an häufig besuchten Orten und Plätzen zentrale Masten aufgestellt, um Daten für die Density-Map zu sammeln. 
Überall dort, wo jedoch keine solchen zentralen Masten existieren, greift der dezentrale Ansatz und die Mobiltelefone fungieren als mobile Masten. 
Der im infrastrukuturbasierenden Ansatz beschriebene spezielle Tag für RPI's der Masten kann hier verwendet werden, damit ein Smartphone erkennt, ob er sich zurzeit im Umkreis eines zentralen Masten befindet oder nicht. 
Mittels Hybridlösung könnte man die Vorteile beider Ansätze vereinen und die Nachteile ausgleichen. Zentrale Masten können zuverlässiger einen festen Ort aufzeichnen, da für diese Funktion optimiert sind und sich dauerhaft an diesem Ort befinden.
Mobile Masten hingegen können an jedem beliebigen Ort eine Ansammlung vieler Smartphones und damit vieler Menschen erkennen und auch eine Bewegung großer Menschenmassen aufzeichnen.
Sie sind demnach flexibler, was den Standort angeht.\\
\\
Das letzte zentrale Problem, was nicht nur unsere Lösungsansätze sondern auch die Corona Warn App im allgemeinen stark betrifft ist, dass nicht jede Person ein Smartphone besitzt und die  Corona Warn App auf seinem Smartphone installiert hat. 
Dementsprechend liegen die aufgezeichneten Werte durch die Masten stets unter der tatsächlichen Anzahl der Personen, die sich am entsprechenden Ort befinden. Laut aktuellem Stand (Mai 2021) benutzen etwa 27,8mio (33\%). \cite{Downloads}
Unter der Annahme, dass die Nutzer der App auf die Standorte bezogen gleichverteilt sind bedeutet dies, dass man die gemessene Anzahl etwa verdreifachen muss, um die tatsächliche Anzahl an Personen zu erhalten.\\

\subsubsection{Simulation}

Um die im Lösungsansatz erarbeiteten theoretischen Überlegungen zu verifizieren und deren Umsetzbarkeit nachzuweisen. haben wir eine Simulation des mastenbasierten Ansatzes implementiert und evaluiert.\\
Eine reale Umsetzung des Features mit echten Masten und Smartphones erwies sich als zu umfangreich und im Rahmen unseres Forschungsprojektes zeitlich nicht umsetzbar.\\
Die Simulation beschränkt sich auf die Umsetzung mittels zentraler Masten. Als Grundlage diente uns hierfür die Simulation "COVID-19 Spread Simulator“ von Miguel Ángel Durán \cite{Simulation}. \\
\\
Wie bereits in Abschnitt IV.B beschrieben, existiert eine bestimme Wahrscheinlichkeit einer doppelten Registrierung innerhalb eines Mastes, wodurch der gemessene Wert stets angeglichen werden muss, um den realen Wert zu approximieren.\\
Im ersten Teil unserer Simulationen testen wir die Formel\\
\centerline{$\text{IDs} = \text{RPIs} * (1-p/2)$}
und im zweiten Teil dieses Abschnitts die Formel
\centerline{$\text{IDs} = \frac{\text{RPIs}}{(1+p)}$}

Das Ziel der Simulation ist es, festzustellen, wie nah der durch beide Formeln berichtigte Wert an der tatsächlichen Anzahl an Handys liegt.\\

\textit{Aufbau:}\\
Wir simulieren 1000 Handys, in Abbildung 8 als Punkte dargestellt, mit zufälliger Bewegung und Position über ein festgelegtes Gebiet wandern. Die Geschwindigkeit und Richtung ihrer Bewegung wird durch ein Random-direction-Model bestimmt.\\
Auf diesem Gebiet sind vier Masten, in Abbildung 8 dargestellt durch größere Kreise, welche persistent speichern, welche Handys in ihrer Reichweite liegen und wie lange dies der Fall ist.\\
Jeder Kreis hat eine sich periodisch verändernde, temporäre ID (RPI) und eine individuelle, permanente ID. Dies ermöglicht es uns, das Zählen von RPI’s zu simulieren und gleichzeitig mit der korrekten Anzahl zu vergleichen.\\
Die durchschnittliche Verweildauer wird immer aktuell berechnet nach Ablauf des Zyklus' eines Mastes. Dazu wird während des Zyklus zu jeder ID eine Zeitdauer gemessen und aus diesen der Durchschnitt ermittelt.\\
Zur Laufzeit der Simulation speichern die Masten eintretende mobile Objekte ab und zählen ab dem Zeitpunkt des Eintritts einen Counter hoch, welcher die Dauer des Aufenthalts im Radius des Mastes misst. Am Ende eines Zyklus werden diese Counter addiert und durch die gemessene Anzahl der Handys geteilt.\\
Zwischenzeitliches Verlassen des Radius des Mastes setzt den Counter nicht zurück. Stattdessen wird nach Wiedereintritt an Hand der ID der alte Counter weitergeführt.\\
Je höher die durchschnittliche Verweildauer $t_V$ des Zyklus ist, desto höher ist die Wahrscheinlichkeit p, dass ein einziges Handy doppelt registriert wird.\\
Zu Verifizierungszwecken haben wir die durchschnittliche Verweildauer mit Hilfe der permanenten IDs berechnet, was den tatsächlichen Wert liefert.\\
Benutzt man stattdessen die RPIs, verringert sich die durchschnittliche Verweildauer, da Handys doppelt gezählt werden können und dadurch fälschlicherweise mit mehreren Countern gerechnet wird, die dabei geringe Zeitdauern liefern. Die durchschnittliche Verweildauer nimmt also ab.\\
Durch eine geringere Verweildauer nimmt auch die Genauigkeit des korrigierten Wertes ab. Da das Ziel unserer Simulation die Abschätzung der Güte unseres angeglichenen Wertes ist, rechnen wir mit der tatsächlichen Verweildauer, um Nebenfaktoren zu minimieren.\\ 
Die Berechnung der durchschnittlichen Verweildauer anhand von RPI's würde jedoch bei einer Umsetzung in der Praxis verwendet werden, da aus Datenschutzgründen nicht die IDs und ihre Verweildauer gemessen werden.\\
Die Simulation bietet zudem die Möglichkeit, Social Distancing durchzuführen. Um den Effekt dieser Maßnahme darzustellen, wurde die Geschwindigkeit von Handys zeitweise auf Null reduziert. Dies führt dazu, dass nur einige wenige Kreise sich überhaupt bewegen, die Anzahl an Ein- und Austritten bei Masten also stark reduziert ist.\\

%\begin{figure}[h]
%	\centering
%	\includegraphics[width=0.45\textwidth]{"Simulation"}
%	\caption{Darstellung der durchgeführten Simulation}
%	\label{sim}
%\end{figure}

\textit{Ergebnisse:}

Abb. \ref{mast_data_plotted_old} stellt die gesammelten Daten in einem Schaubild dar, ohne das Social Distancing benutzt wird.\\
Hierbei ist der rote Graph die durchschnittliche Verweildauer, der grüne Graph der gemessene Wert, der orangene Graph der korrigierte Wert und der blaue Graph der reale Wert.\\
Mit Social Distancing (Abb. \ref{mast_data_plotted_social_distancing_old}) ist die durchschnittliche Verweildauer wesentlich höher und der angeglichene Wert liegt näher an dem gemessen Wert als in Abb. \ref{mast_data_plotted_old}.

\begin{figure}[h]
	\centering
	\includegraphics[width=0.5\textwidth]{"Mast_Data_Plotted_old"}
	\caption{Ergebnisse ohne Social Distancing}
	\label{mast_data_plotted_old}
\end{figure}

\begin{figure}[h]
	\centering
	\includegraphics[width=0.5\textwidth]{"Mast_Data_Plotted_Social_Distancing_old"}
	\caption{Ergebnisse mit Social Distancing}
	\label{mast_data_plotted_social_distancing_old}
\end{figure}

\textit{Schlussfolgerung:}\\
Die gesammelten Daten belegen, dass die durch die Formel berechnete Anzahl (orangener Graph) wesentlich näher an der tatsächlichen Zahl (blauer Graph) liegt als die gemessene Menge (grüner Graph). Es fällt jedoch auf, dass in fast allen Fällen die berechnete Zahl immer noch größer ist als die tatsächliche. Nur in einzelnen Fällen wird die Zahl zu niedrig abgeschätzt.\\
Hinzu kommt die Beobachtung, dass je höher die durchschnittliche Verweildauer, desto geringer die Differenz zwischen korrigierten und realen Wert. Die Genauigkeit unserer Abschätzung wird demnach besser, je höher die Verweildauer ist.\\

Aus dem Ergebnis lässt sich die Erkenntnis gewinnen, dass die bisherige Formel nicht optimal ist, da der berechnete Wert stets zu groß ist, sich grundlegend jedoch im Vergleich zum gemessenen Wert ein Wert näher an der realen Anzahl berechnen lässt.

\textit{Simulation zweiter Formel:}
Um die Formel $\text{IDs} = \frac{\text{RPIs}}{(1+p)}$
zu testen, benutzen wir dieselbe Simulation wie zuvor, mit der Berechnung des korrigierten Wertes als einzigen Unterschied.\\
Mit Hilfe dieser Formel ergeben sich die in Abbildung \ref{mast_data_plotted} und Abbildung \ref{mast_data_plotted_social_distancing} gezeigten Werte.

\begin{figure}[h]
	\centering
	\includegraphics[width=0.5\textwidth]{"Mast_Data_Plotted"}
	\caption{Neue Ergebnisse ohne Social Distancing}
	\label{mast_data_plotted}
\end{figure}

\begin{figure}[h]
	\centering
	\includegraphics[width=0.5\textwidth]{"Mast_Data_Plotted_Social_Distancing"}
	\caption{Neue Ergebnisse mit Social Distancing}
	\label{mast_data_plotted_social_distancing}
\end{figure}

\textit{Ergebnisse:}\\
Wie auch im ersten Abschnitt lässt sich feststellen, dass der durch die Formel korrigierte Wert immer näher am realen Wert liegt als der gemessene Wert. Die Approximation ist also auch hier generell erfolgreich.\\
Zusätzlich dazu sind die korrigierten Werte in dieser Simulation in gleichen Anteilen etwas zu hoch oder zu niedrig, und nicht standardmäßig zu hoch abgeschätzt, wie dies in der vorherigen Simulation der Fall war.\\
Der durchschnittliche Unterschied zwischen korrigiertem und realen Wert ist zudem niedriger mit dieser Formel als bei der Verwendung der anderen Formel.\\


\textit{Schlussfolgerung:}\\
Durch den geringeren Unterschied zwischen korrigiertem und realen Wert ist die zweite Formel besser als die Formel der ersten Simulation für die Erfüllung unseres Ziels, die Anzahl von individuellen Handys innerhalb eines Mastes zu berechnen.\\

\subsubsection{Bewertung}
Das Feature der Density-Map erzielt im Hinblick auf die Eindämmung des Coronavirus keinen besonders hohen Nutzen im Vergleich zu anderen vorgestellten Erweiterungen.\\
Die Hauptidee hinter der Implementierung dieser Erweiterung innerhalb der Corona Warn App ist die, Orte großer Menschenmassen bewusst zu meiden und damit die Ausbreitung des Viruses einzudämmen.\\
Durch einen vorausschauenderen Lebensalltag oder Kontaktbeschränkungen seitens der Regierung kann dieser Effekt auch erreicht werden.\\
Jedoch ist die Density-Map ein Feature, welches auch außerhalb der Pandemiezeit effektiv genutzt werden kann und einen Nutzen bringt.\\
Gerade in Zeiten, in denen keine größeren Einschränkungen gelten, ist es vermutlich umso wichtiger, weiterhin bewusst Abstand zu größeren Ansammlungen zu halten, um der Ausbreitung einer weiteren Infektionswelle vorzubeugen.\\
Deshalb trägt die Density-Map vor allem in Zeiten der Lockerung der Beschränkungsmaßnahmen einen guten Teil dazu bei, diesen Zustand zu erhalten.\\
Unabhängig vom Kontext der Pandemie hat die Erstellung einer Density-Map, welche auf der anonymisierten Verfolgung von Kontakten basiert, ebenfalls viele Verwendungen.\\
So lässt sich zum Beispiel messen, wie viele Plätze in einem Zug bereits belegt sind, oder ob ein öffentlicher Platz wie ein Spielplatz überfüllt ist. Die Auswertung der gesammelten Daten und Korrigierung mittels der in dieser Arbeit ermittelten Formel wäre auch in diesen Bereichen anwendbar und hat somit Relevanz über den Zeitraum der Pandemie hinaus.\\

Auch die Tatsache, dass bereits ein vergleichbares Feature von Google existiert, stellt den Nutzen des Features natürlich in Frage.\\
Jedoch wird das Feature dem Ziel, eine im Hinblick auf Privatsphäre fortgeschrittenere Alternative zu entwickeln, gerecht. 
Die Density-Map sammelt in allen vorgestellten Lösungsansätzen keine sensiblen Nutzerdaten und verfolgt einen weitestgehend dezentralen Ansatz.\\
Trotzdem ist logisch, dass man bei einem Feature, welches die Ansammlung größerer Menschenmassen tracken möchte, nicht zu 100\% anonym ist.\\
Vielmehr greift hier der Effekt der K-Anonymity, dass man nur eine von vielen Personen ist, sowie der Fakt, dass die eigene Identität hinter einem temporären RPI verborgen ist.\\
Trotzdem ist es gerade im infrastrukturlosen Ansatz ohne zentrale Masten möglich, Ansammlungen von Menschen im privaten Raum (wie etwa Partys) zu tracken.\\

Auch im Hinblick auf Akzeptabilität könnte das Feature auf Probleme stoßen.\\
Unwissende Nutzer, die sich nicht mit der Umsetzung des Features auskennen, könnten davon ausgehen, dass man mithilfe der Density-Map dauerhaft getrackt werden kann.\\
Ein ähnliches Phänomen gab es bereits, als die Corona Warn App das erste Mal vorgestellt wurde.\\
Dagegen hilft eine klare Aufklärung darüber, wie das Feature im Konkreten funktioniert, sowie eine freiwillige Teilnahme am Benutzen der Density-Map.\\

Im Aspekt der Sicherheit stellt die Density-Map kein höheres Risiko als die bisherige Corona Warn App dar.\\ 
Es existieren keine sensiblen Nutzerdaten, die durch eine Angriff auf das System preisgegeben werden könnten. Durch die dezentrale Auswertung und Aggregation der Daten existiert außerdem kein Single Point of Failure.\\
Es könnte natürlich passieren, dass einzelne Masten gehackt werden, wodurch beispielsweise die Begegnungsaufzeichnung für diesen Ort verfälscht werden könnte.\\
Dies stellt jedoch weder ein Risiko für alle anderen Masten dar, noch für einen Nutzer, der sich in Nähe eines gehackten Masten befindet, da wie bereits genannt nur anonyme und verschlüsselte Daten mit dem Masten ausgetauscht werden.

Für die Umsetzung des Density-Map Features ergibt sich ein größerer Aufwand zur Installation der zentralen Masten. Um eine gute deutschlandweite Abdeckung zu erreichen, müssten hierfür vermutlich zehn- bis hunderttausende Geräte zu Verfügung gestellt werden, die als zentrale Masten fungieren könnten.\\
Beim infrastrukturlosen Ansatz würde dieser Punkt jedoch wegfallen.\\

Alles in Allem stellt das Density-Map Feature eine gute Erweiterung der Corona Warn App, sowohl im Hinblick auf die Eindämmung des Virus, als auch für eine unabhängige Nutzung außerhalb der Pandemie, dar.\\
Dabei schafft sie es, das BLE-Feature der App erweitert zu nutzen und dabei den bisher gesetzten Privatsphärestandards gerecht zu werden. 

\section{Zusammenfassung und Ausblick} \label{conclusion}

\section*{Danksagung}

\begin{thebibliography}{00}
\bibitem{CoronaZahlenWorld} https://coronavirus.jhu.edu/map.html (07.06.2021)
\bibitem{CoronaZahlenDe} https://www.rki.de/DE/Content/InfAZ/N/Neuartiges\_Coronavirus/Fallzahlen.html (07.06.2021)
\bibitem{CWA} https://www.rki.de/DE/Content/InfAZ/N/Neuartiges\_Coronavirus/WarnApp/Warn\_App.html (09.06.2021)
\bibitem{Simulation} Miguel Ángel Durán, COVID-19 spread simulator, 
https://github.com/midudev/covid-19-spread-simulator
\bibitem{MCS} R. K. Ganti, F. Ye and H. Lei, "Mobile crowdsensing: current state and future challenges," in IEEE Communications Magazine, vol. 49, no. 11, pp. 32-39, November 2011, doi: 10.1109/MCOM.2011.6069707.
\bibitem{AirQuality} J. Dutta, F. Gazi, S. Roy and C. Chowdhury, "AirSense: Opportunistic crowd-sensing based air quality monitoring system for smart city," 2016 IEEE SENSORS, 2016, pp. 1-3, doi: 10.1109/ICSENS.2016.7808730.
\bibitem{NoiseSense} Z. Qin and Y. Zhu, "NoiseSense: A Crowd Sensing System for Urban Noise Mapping Service," 2016 IEEE 22nd International Conference on Parallel and Distributed Systems (ICPADS), 2016, pp. 80-87, doi: 10.1109/ICPADS.2016.0020.
\bibitem{TrafficRegulators} F. Saremi and T. Abdelzaher, "Combining Map-Based Inference and Crowd-Sensing for Detecting Traffic Regulators," 2015 IEEE 12th International Conference on Mobile Ad Hoc and Sensor Systems, 2015, pp. 145-153, doi: 10.1109/MASS.2015.18.
\bibitem{BLESensorDevices} Y. Tsuchiya and N. Ishikawa, "Mobile Crowdsensing from Sensor Devices Using Bluetooth Low Energy (BLE)," 2018 International Conference on Computational Science and Computational Intelligence (CSCI), 2018, pp. 800-805, doi: 10.1109/CSCI46756.2018.00160.
\bibitem{CICN} T. You, M. Umair and Y. Hong, "Mobile crowd sensing based on CICN," 2017 International Conference on Information and Communication Technology Convergence (ICTC), 2017, pp. 1272-1275, doi: 10.1109/ICTC.2017.8190919.
\bibitem{BLETags} A. Basalamah, "Sensing the Crowds Using Bluetooth Low Energy Tags," in IEEE Access, vol. 4, pp. 4225-4233, 2016, doi: 10.1109/ACCESS.2016.2594210.
\bibitem{GoogleStosszeiten} "Google Stoßzeiten Feature" \\ https://support.google.com/business/answer/6263531/sto\%C3\%9Fzeiten-wartezeiten-und-besuchsdauer?hl=de (04.06.2021)
\bibitem{LocationPrivacy} Marius Wernke, Pavel Skvortsov, Frank Dürr, Kurt Rothermel,  "A classification of location privacy attacks and approaches", in Springer, November 2012
\bibitem{GoogleStandortverlauf} "Google Standortverlauf" \\ https://support.google.com/maps/answer/6258979/google-maps\%E2\%80\%91zeitachse-computer?co=GENIE.Platform\%3DDesktop\&hl=de (04.06.2021)
\bibitem{RKI} "Robert-Koch-Institut" \\ https://www.rki.de/DE/Home/homepage\_node.html (09.06.2021)
\bibitem{BMG} "Bundesministerium für Gesundheit (BMG)" \\ https://www.bundesgesundheitsministerium.de (09.06.2021)
\bibitem{RKIDatenspende} “RKI - 2020 - Mit Daten von Fitnessarmbändern und Smartwatches mehr über die Verbreitung des Coronavirus erfahren“ RKI.de \\ https://www.rki.de/DE/Content/Service/Presse/Pressemitteilungen/2020/04\_2020.html (09.06.2021)
\bibitem{Gedaechtnistest} “Aufruf zum bundesweiten Gedächtnistest: Was macht „Corona“ mit unserem Kopf?“ DZNE.de \\https://www.dzne.de/aktuelles/pressemitteilungen/presse/aufruf-zum-bundesweiten-gedaechtnistest-was-macht-corona-mit-unserem-kopf (09.06.2021)
\bibitem{GesundheitsforschungNews} "Corona: Aktuelles aus der Forschung" \\https://www.gesundheitsforschung-bmbf.de/de/corona-aktuelles-aus-der-forschung-10857.php (09.06.2021)
\bibitem{Dunkelziffer} https://www.rki.de/DE/Content/Gesundheitsmonitoring/Studien/cml-studie/Factsheet\_Kupferzell.html (09.06.2021)
\bibitem{Einwohnerzahl} https://www.destatis.de/DE/Themen/Gesellschaft-Umwelt/Bevoelkerung/Bevoelkerungsstand/\_inhalt.html (09.06.2021
\bibitem{Downloads} https://www.rki.de/DE/Content/InfAZ/N/Neuartiges\_Coronavirus/WarnApp/Archiv\_Kennzahlen/Kennzahlen\_21052021.pdf?\_\_blob=publicationFile (09.06.2021)
\bibitem{Lotto} https://www.tagesschau.de/ausland/amerika/usa-ohio-impfung-lotterie-101.html (09.06.2021)
\bibitem{Notbremse} https://www.bundesgesundheitsministerium.de/fileadmin/Dateien/3\_Downloads/Gesetze\_und\_Verordnungen/GuV/B/4\_BevSchG\_BGBL.pdf (09.06.2021)
\bibitem{DarfIchDas} https://www.darfichdas.info/ (09.06.2021)
\bibitem{Risikoberechnung} "Corona-Warn-App Dokumentation" \\https://raw.githubusercontent.com/corona-warn-app/cwa-documentation/master/images/risk\_calculation/risk\_calculation\_enf\_v2\_overview.pdf (09.06.2021)
\bibitem{freiwilligeDatenspende} "Corona-Warn-App: Freiwillige Datenspende für verbesserte Wirksamkeit" \\https://www.bundesregierung.de/breg-de/themen/coronavirus/corona-warn-app-update-1-13-1873182 (09.06.2021)
\end{thebibliography}
\vspace{12pt}

\end{document}
