\documentclass[conference]{IEEEtran}
\IEEEoverridecommandlockouts
% The preceding line is only needed to identify funding in the first footnote. If that is unneeded, please comment it out.
\usepackage{cite}
\usepackage{amsmath,amssymb,amsfonts}
\usepackage{algorithmic}
\usepackage{graphicx}
\usepackage{textcomp}
\usepackage{xcolor}
\usepackage[ngerman]{babel}
\def\BibTeX{{\rm B\kern-.05em{\sc i\kern-.025em b}\kern-.08em
    T\kern-.1667em\lower.7ex\hbox{E}\kern-.125emX}}
\begin{document}

\title{Entwurf und Analyse von Crowd-Sensing Mechanismen für mobile Corona-Warn-Applikationen\\
{\footnotesize \textsuperscript{*}Note: Sub-titles are not captured in Xplore and
should not be used}
}

\author{\IEEEauthorblockN{Gabriel Bonnet}
\IEEEauthorblockA{\textit{Bsc. Science Softwaretechnik} \\
\textit{Universität Stuttgart}\\
Stuttgart, Germany}
\and
\IEEEauthorblockN{Kai Braun} 
\IEEEauthorblockA{\textit{Bsc. Science Softwaretechnik} \\
\textit{Universität Stuttgart}\\
Stuttgart, Germany}
\and
\IEEEauthorblockN{Hannes Deichmann}
\IEEEauthorblockA{\textit{Bsc. Science Softwaretechnik} \\
\textit{Universität Stuttgart}\\
Stuttgart, Germany}
\and
\IEEEauthorblockN{Timm Marquardt}
\IEEEauthorblockA{\textit{Bsc. Science Informatik} \\
\textit{Universität Stuttgart}\\
Stuttgart, Germany}
\and
\IEEEauthorblockN{\textbf{Betreuer:} Dr. rer. Nat. Frank Dürr}
\IEEEauthorblockA{\textit{Institut für Parallele und Verteilte Systeme} \\
\textit{Universität Stuttgart}\\
Stuttgart, Germany \\
frank.duerr@ipvs.uni-stuttgart.de}
\and
\IEEEauthorblockN{\textbf{Prüfer: } Prof. Dr. rer. Nat. Kurt Rothermel}
\IEEEauthorblockA{\textit{Institut für Parallele und Verteilte Systeme} \\
\textit{Universität Stuttgart}\\
Stuttgart, Germany \\
kurt.rothermel@ipvs.uni-stuttgart.de}
}

\maketitle

\begin{abstract}
TODO
\end{abstract}

\begin{IEEEkeywords}
Corona, Corona-Warn-App, Density Map, Risikoberechnung
\end{IEEEkeywords}

\section{Einleitung}
This document is a model and instructions for \LaTeX.
Please observe the conference page limits. 

\section{Bestehende Arbeiten}

\subsection{Corona Lage}
\subsection{Corona Warn App}
\subsection{Bluetooth Low Energy und Density Maps}

\section{Lösungsvorschläge}


\subsection{Übersicht aller Erweiterungen}

\textit{Impfpass:} - In der aktuellen Situation ist es wichtig, nachweisen zu können, dass man geimpft wurde. Dies könnte man entweder dadurch regeln, einen analogen Impfnachweis immer bei sich zu tragen, oder durch einen digiatalen Impfpass in der Corona-Warn-App.\\
Größter Vorteil eines in der App integrierten Nachweises wäre die Digitalisierung des Verwaltungswesens und dadurch eine Vereinfachung des Organisationsaufwands.\\ Anstatt mehrere Unterlagen mit sich herumzutragen, hat man diese stattdessen gesammelt in einer App.\\
Digitalisierung ist zudem ein positiv konnotierter Begriff, welcher einen Anreiz für die Bevölkerung schafft, die App zu nutzen. Dies ist immer ein relevanter Nutzen, da es das Ziel ist, die Corona-Warn-App von 100\% der Bevölkerung nutzen zu lassen. \\
Implementiert man zusätzlich dazu eine Art Krankenakte, in der man Merkmale wie Gewicht und Vorerkrankungen einträgt, liese sich dadurch die Impfpriorisierung und die Reihenfolge in der Bevölkerung besser organisieren.\\
Benötigte Daten für einen generellen Impfpass wären:\\
- Name \\
- Krankenkasse\\
- Durchgeführte Impfungen\\
Für die Einschätzung der Impfpriorisierung wäre zusätzlich erforderlich:\\
- Alter\\
- Vorerkrankungen \\
- Gewicht \\
- Beruf/Arbeitsumfeld\\
Bei einem unverbindlichen Impfpass wären diese Angaben nicht wirklich nutzbar außer für den Endnutzer, da man auf die Eigenverantwortung des einzelnen Nutzers angewiesen wäre, die Daten also einfach gefälscht werden können.\\
Stattdessen müsste man, um die Daten als sicher zu garantieren, die Bearbeitung des Impfpass beschränken, sodass nur offizielle Stellen wie Krankenhäuser oder Arztpraxen diesen bearbeiten können.\\
Somit wird Richtigkeit der Daten gewährt, welche dann für das Krankenwesen genutzt werden können.\\ \\
\textit{Gamification:}\\
Eine Möglichkeit, das Problem einer zu niedrigen Nutzung der App zu adressieren, ist Gamification. Dadurch versuchen wir, den Anreiz zu steigern, die App herunterzuladen und auch aktiv zu benutzen.\\ 
Gamification der App kann es in verschiedenen Varianten geben. Zum Beispiel lassen sich simple Spielmechaniken wie persönliche Highscores einbauen, welche man mit anderen Nutzern der App vergleichen kann.  Dafür bieten sich Statistiken an, die sich bereits aus der Nutzung der App erstellen lassen. Dazu zählen die getroffenen Personen an einem Tag, die durchschnittliche Dauer einer Begegnung oder der ermittelte Risikofaktor. Ziel der Highscores wäre es dann, diese Werte so möglich wie gering zu halten oder im Vergleich zu anderen Nutzern bessere Werte zu haben.\\
Vorstellbar sind auch daraus resultierende optische Veränderungen der App durch neue Farben oder ein anderes Design.\\
Dieses Feature richtet sich vermutlich an eine jüngere Zielgruppe, die mit solchen Mechaniken meist mehr vertraut ist als ältere Generationen.\\
Es könnte sogar sein, dass dieses Feature als Verharmlosung der ernsten Thematik interpretiert wird, was in einem gegensätzlichen Effekt zur eigentlichen Intention resultieren würde.\\
Eine Alternative zu Highscores und anderen virtuellen Errungenschaften wäre eine Art Belohnung in der Realität, zum Beispiel durch eine in der App durchgeführte Lotterie.\\
Dies ist vermutlich attraktiver für alle Generationen, jedoch schwieriger umzusetzen, da hier Betrug oder mehrmalige Teilnahme verhindert werden muss. Vorstellbar wäre auch eine Verbindung mit den ermittelten Statistiken, sodass Leute mit guten Werten eine höhere Gewinnchance haben.\\
 
\subsection{Bewertungskriterien}

Um einen möglichst einheitlichen Vergleich der Features zu garantieren, haben wir sechs Bereiche festgelegt, nach denen die Features beurteilt werden.\\
Diese Bereiche sind wie folgt aufgeteilt:\\
\textit{Privatsphäre:} Wie stark greift das Feature in die Privatsphäre ein?\\
\textit{Nutzen:} Welchen effektiven Nutzen bringt das Feature und lohnt sich der Aufwand für den effektiven Nutzen?\\
\textit{Umsetzbarkeit:} Wie realistisch ist es, dieses Feature zu implementieren in Anbetracht von Datenschutz, Programmierung oder anderen Hindernissen?\\
\textit{Akzeptanz:} Da wir eine möglichst hohe Nutzung der App in der Bevölkerung erzielen wollen, ist es wichtig, dass das Feature entweder einen Anreiz zur Nutzung schafft oder zumindest diesen nicht schwächt. Die Frage ist also, ob die neue Funktion Skepsis oder Missbilligung beim Nutzer auslöst.\\
\textit{Sicherheit:} Wie gut lässt sich das implementierte Feature vor Missbrauch und Angriffen schützen?\\
\textit{Daten:} Welche Daten werden mindestens benötigt, dass das Feature funktioniert?\\
\subsection{Bewertung Nebenfeatures}

\section{Hauptfeatures}
\subsection{Erweiterte Risikoberechnung}
\subsubsection{Familienfeature}
\subsubsection{TRL update}
\subsubsection{Roter Summand}
\subsubsection{Grüner Summand}

\subsection{Density Map}
\subsubsection{Lösungsansatz}
\subsubsection{Simulation}

\textit{Motivation:} \\
Unser Ansatz der festen Masten basiert darauf, Handys beim Eintritt in einen Mastbereich zu registrieren und ihren RPI zu speichern.\\
Dadurch, dass der RPI periodisch wechselt, werden zwangsläufig Handys doppelt registriert, sollten sie sich zum Zeitpunkt des Wechsels in der Reichweite des Mastes befinden. \\
Die gemessene Anzahl an Handys weicht somit immer zu einem bestimmten Teil von der realen Menge ab. Diesen Wert korrigieren wir nach unten, abhängig von der durchschnittlichen Verweildauer.\\
Die Wahrscheinlichkeit p einer doppelten Registrierung ist definiert durch

\centerline{$p = t_V / t_Z$}

wobei $t_Z$ die Dauer eines Zyklus ist und $t_V$ die durchschnittliche Verweildauer während dieses Zyklus. Die Berechnung der Verweildauer wird in Abschnitt "Aufbau" beschrieben. Da doppelt-registrierte Handys einfach gezählt werden sollen, muss die Hälfte des ermittelten Anteils trotzdem gezählt werden, womit sich die folgende Formel ergibt:

\centerline{Korrigierte Anzahl = Gemessene Anzahl * $(1 – p/2)$}

Figur 1 zeigt den durch diese Formel berechneten Unterschied zwischen dem realen und gemessenen Wert, abhängig von der durchschnittlichen Verweildauer.

\begin{figure}[h]
	\centering
	\includegraphics[width=0.45\textwidth]{"Alte Formel Graph"}
	\caption{Anteil realer Wert am gemessen, abhängig von der durchschnittlichen Verweildauer}
\end{figure}

Das Ziel der Simulation ist es, festzustellen, wie nah der auf diese Weise berichtigte Wert an der tatsächlichen Zahl von Handys liegt.\\

\textit{Aufbau:}\\
Um die Formel zu verifizieren, modifizieren wir die Simulation "COVID-19 spread simulator“ von Miguel Ángel Durán \cite{Simulation}.\\
Es werden 200 Handys simuliert, die als Kreise mit zufälliger Bewegung und Position über ein festgelegtes Gebiet wandern. Auf diesem Gebiet sind 4 Masten, dargestellt durch größere Kreise, welche permanent speichern, welche Kreise in ihrer Reichweite liegen und wie lange dies der Fall ist. \\
Jeder Kreis hat eine sich periodisch verändernde, temporäre ID (RPI) und eine individuelle, permanente ID. Dies ermöglicht es uns, das Zählen von RPI’s zu simulieren und gleichzeitig die korrekte Anzahl zu wissen.\\
Die durchschnittliche Verweildauer wird immer aktuell berechnet nach Ablauf des Zyklus eines Mastes. Dazu wird während des Zyklus mit jeder ID eine Zeitdauer abgespeichert und aus diesem der Durchschnitt ermittelt.  Zur Laufzeit der Simulation speichern die Masten eintretende Kreise neu ab und zählen ab diesem Zeitpunkt einen Counter hoch. Am Ende eines Zyklus werden diese Counter addiert und durch die gemessene Anzahl der Kreise geteilt.\\
Je höher die durchschnittliche Verweildauer $t_V$ des Zyklus ist, desto höher ist die Wahrscheinlichkeit p, dass ein einziges Handy doppelt registriert wird.\\
Zu Verifizierungszwecken haben wir die durchschnittliche Verweildauer mit Hilfe der permanenten IDs berechnet, was einen genaueren Wert liefert. Benutzt man stattdessen die RPIs, verringert sich die durchschnittliche Verweildauer und der korrigierte Wert wird ungenauer. Dies entspricht jedoch der Anwendung in der Praxis, da aus Datenschutzgründen nicht die IDs und ihre Verweildauer gemessen werden.\\
Die Simulation bietet zudem die Möglichkeit, Social Distancing durchzuführen. Dies führt dazu, dass ein Großteil der Kreise keine Geschwindigkeit mehr haben, und nur einige wenige sich überhaupt bewegen.\\

\begin{figure}[h]
	\centering
	\includegraphics[width=0.45\textwidth]{"Simulation"}
	\caption{Darstellung der durchgeführten Simulation}
\end{figure}

\textit{Ergebnis:}
Die gesammelten Daten belegen, dass die durch die Formel berechnete Anzahl (orangener Graph) wesentlich näher an der tatsächlichen Zahl (blauer Graph) liegt als die gemessene Menge (grüner Graph). Es fällt jedoch auf, dass in fast allen Fällen die berechnete Zahl immer noch größer ist als die tatsächliche. Nur in einzelnen Fällen wird die Zahl zu niedrig abgeschätzt.\\
Fig. 3 stellt die gesammelten Daten in einem Schaubild dar, ohne das Social Distancing benutzt wird.\\
Hierbei ist der rote Graph die durchschnittliche Verweildauer, der grüne Graph der gemesser Wert, der orangene Graph der korrigierte Wert und der blaue Graph der reale Wert.\\
Mit Social Distancing ist die durchschnittliche Verweildauer wesentlich höher und der angeglichene Wert näher an dem gemessen Wert liegt als in Fig.3.

\begin{figure}[h]
	\centering
	\includegraphics[width=0.5\textwidth]{"Mast_Data_Plotted_old"}
	\caption{Ergebnisse ohne Social Distancing}
\end{figure}

\begin{figure}[h]
	\centering
	\includegraphics[width=0.5\textwidth]{"Mast_Data_Plotted_Social_Distancing_old"}
	\caption{Ergebnisse mit Social Distancing}
\end{figure}

Hinzu kommt die Beobachtung, dass je höher die durchschnittliche Verweildauer, desto geringer die Differenz zwischen korrigierten und realen Wert. Die Genauigkeit unserer Abschätzung wird demnach besser, je höher die Verweildauer ist.\\

\textit{Schlussfolgerung:}
Aus dem Ergebnis lässt sich die Erkenntnis gewinnen, dass die bisherige Formel nicht optimal ist. Daher haben wir einen neuen Ansatz zur Berechnung gewählt.\\
Die Wahrscheinlichkeit p einer doppelten Registrierung bleibt gleich. Wir gehen davon aus, dass sich mit dieser und der korrekten Anzahl der IDs eines Zyklus die gemessene Anzahl berechnen lässt:\\
\centerline{RPIs = IDs * (1 + p) }
Nach Umformung ergibt sich somit eine neue Formel zur Angleichung unseres gemessen Wertes:\\
\centerline{IDs = RPIs/ (1+p)}\\

\textit{Simulation neuer Formel:}
Um die neue Formel zu testen, benutzen wir dieselbe Simulation wie zuvor, mit der Berechnung des korrigierten Wertes als einzigen Unterschied.\\
Mi Hilfe der neuen Formel ergeben sich folgende Werte:

\begin{figure}[h]
	\centering
	\includegraphics[width=0.5\textwidth]{"Mast_Data_Plotted"}
	\caption{Neue Ergebnisse ohne Social Distancing}
\end{figure}

\begin{figure}[h]
	\centering
	\includegraphics[width=0.5\textwidth]{"Mast_Data_Plotted_Social_Distancing"}
	\caption{Neue Ergebnisse mit Social Distancing}
\end{figure}


\subsection{Bewertung der Hauptfeatures}

\section{Zusammenfassung und Ausblick}

\section*{Danksagung}

Dankschön an meine Mama und mein Papa und mein Hund Bello, der immer so lieb mit mir gekuschelt hat! Danke an unseren Sponsor
RAID: Shadow Legends und NordVPN!

\cite{b1}
\begin{thebibliography}{00}
\bibitem{Simulation} Miguel Ángel Durán, COVID-19 spread simulator, 
https://github.com/midudev/covid-19-spread-simulator
\end{thebibliography}
\vspace{12pt}

\end{document}
